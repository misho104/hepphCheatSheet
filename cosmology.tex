% Time-Stamp: <2026-01-14 01:49:16>


\documentclass[CheatSheet]{subfiles}

\begin{document}
\summarystyle
\section{Cosmology}
\paragraph{FLRW metric} With a scale factor normalized by $a(t_0)=1$,\\
\begin{minipage}{0.6\textwidth}
\begin{equation}
  \dd s^2 = \dd t^2-a^2(t)\left[
  \frac{\dd r^2}{1-K r^2}+r^2\dd\theta^2+r^2\sin^2\theta\dd\phi^2
\right]
\end{equation}\vspace{-2em}
\begin{alignat*}{2}
  &\text{comoving coordinate~~}&& \vc r=(r\si\theta\co\phi,r\si\theta\co\phi,r\co\theta),\\
  &\text{proper coordinate~~}&& \vc x(t)=a(t)\vc r,\\
  &\text{comoving distance~~}&& \chi_{AB}=\int_{r_A}^{r_B}\frac{\dd r}{\sqrt{1-Kr^2}},\\
  &\text{proper distance~~}&& d_{AB}(t)=a(t)\chi_{AB}.
\end{alignat*}
\end{minipage}
\begin{minipage}{0.39\textwidth}
\hfill\includegraphics[width=0.99\textwidth]{figs/expansion.pdf}
\end{minipage}

\vspace{0.5em}

Ricci tensor and scalar are given by
\begin{alignat}{3}
 &R_{00}=\tensor{R}{^0_0}=\frac{3\ddot a}{a}, \qquad&
 &R_{0i}=R_{i0}=\tensor{R}{^0_i} = \tensor{R}{^i_0}=0, \qquad&
 &R_{ij}\neq 0,\notag
\\
 &\tensor{R}{^i_j} = \delta^i_j\left(\frac{\ddot a}{a} + \frac{2\dot a^2}{a^2} + \frac{2K}{a^2}\right);\qquad&
 &R = 6\left(\frac{\ddot a}{a} + \frac{\dot a^2}{a^2} + \frac{K}{a^2}\right).
\end{alignat}


\paragraph{Particle density}For a massless particle, with $L^\pm_n=\pm\mathop{\text{\texttt{PolyLog}}}(n,\pm\ee^{\mu/T})$ and arrows denoting $\mu\to0$,
\begin{alignat}{4}
 n\w{MB}    &=\frac{\ee^{\mu/T}}{\pi^2}gT^3 &\quad&\to\frac{1}{\pi^2}gT^3,&\qquad
 \rho\w{MB} &=3Tn\w{MB}&\quad& \to \frac{3}{\pi^2}gT^4,\\
 n\w{BE}&=\frac{L^+_3}{\pi^2}gT^3 &&\to \frac{\zeta_3}{\pi^2}gT^3,&
 \rho\w{BE}&=\frac{3L^+_4}{\pi^2}gT^4&&\to\frac{\pi^2}{30}gT^4,\\
 n\w{FD}&=\frac{L^-_3}{\pi^2}gT^3&&\to\frac34\frac{\zeta_3}{\pi^2}gT^3,&
 \rho\w{FD}&=\frac{3L^-_4}{\pi^2}gT^4&&\to\frac78\frac{\pi^2}{30}gT^4,
\end{alignat}

For massive particle, with $x=m/T$ and $K_n(x)=\mathop{\text{\texttt{BesselK}}}(n,x)$,
\begin{alignat}{3}
 n\w{MB}&=g\ee^{\mu/T}\cdot\frac{T^3}{2\pi^2}x^2K_2(x)\qquad
&\stackrel{x\gg1}\longrightarrow&~
g\ee^{\mu/T}\frac{T^3}{(2\pi)^{3/2}}x^{3/2}\ee^{-x},\\
\rho\w{MB}&=\left(3+\frac{xK_1(x)}{K_2(x)}\right)Tn\w{MB}
&\stackrel{x\gg1}\longrightarrow&~
\left(m+\frac{3}{2}T+\frac{15T^2}{8m}\right)n\w{MB},\qquad&
 p\w{MB}&=T n\w{MB}.
\end{alignat}


\detailstyle

\clearpage

\subsection{FLRW metric}
Two conventions are known for FLRW \GRAY{(\RUSSIAN{Фр\'{и}дман}-Lema\^{i}tre-Robertson-Walker)} metric:
\begin{align}
  \dd s^2
&= \dd t^2-a^2(t)\left[\frac{\dd r^2}{1- K r^2}+ r^2\dd\theta^2+ r^2\sin^2\theta\dd\phi^2\right]
&&\text{$[r]=\text{(length)}$, $a$ is unitless with $a(t_0)=1$}
\\
&= \dd t^2-R^2(t)\left[\frac{\dd \tilde r^2}{1-\tilde K \tilde r^2}+\tilde r^2\dd\theta^2+\tilde r^2\sin^2\theta\dd\phi^2\right]
&&\text{$[R]=\text{(length)}$, $\tilde r$ is unitless, $\tilde K=\{0,\pm1\}$}
\end{align}
related by a rescaling, $R(t)/a(t)=R(t_0)\equiv R_0$, i.e., 
$r = \tilde r R_0$ and $K = \tilde K/R_0^2$.
The curvature radius is given by $6K/a^2$ and a spherical, flat, and hyperspherical universe are respectively given by $K>0$, $K=0$, and $K<0$.

FLRW metric can have several forms. For $\{K>0,K=0,K<0\}$,
\begin{align}
\dd s^2
&= \dd t^2-a^2(t)\left(\frac{\dd r^2}{1- K r^2}+ r^2\dd\Omega\right)
&\dd\Omega &= \dd\theta^2+\sin^2\theta\dd\phi^2,
\\&= \dd t^2-a^2(t)\left[\dd\vc r^2 + \frac{K(\vc r\cdot\dd\vc r)^2}{1-K\|\vc r\|^2}\right]
&\vc r&=(r\sin\theta\cos\phi,\sin\theta\sin\phi,\cos\phi)
\label{eq:FLRW2}
\\&= \dd t^2-\left[\frac{a(t)}{1 + (K/4)\rho^2}\right]^2(\dd\rho^2 + \rho^2\dd\Omega)
     &\rho&=R_0 \tilde \rho:=\frac{2r}{1+\sqrt{1-Kr^2}}=\frac{2\tilde r R_0}{1+\sqrt{1-\tilde K\tilde r^2}}
\label{eq:FLRW3}
\\&= \dd t^2-\left[\frac{R(t)}{1 + (\tilde K/4){\tilde\rho}^2}\right]^2(\dd{\tilde\rho}^2 + {\tilde\rho}^2\dd\Omega)
\\
&=
\dd t^2-R^2(t)\Bigl(\dd{\tilde\chi}^2 + \{\sin\tilde\chi,\tilde\chi,\sinh\tilde\chi\}^2\dd\Omega\Bigr)
&\dd\chi&=R_0\dd\tilde\chi = \frac{\dd r'}{\sqrt{1-K r'^2}}
\quad\text{[comoving distance]}
\\
&=a^2(t)\Bigl(
\dd\eta^2-\dd{\chi}^2 - R_0^2\{\sin\tilde\chi,\tilde\chi,\sinh\tilde\chi\}^2\dd\Omega
\Bigr)
&\dd\eta& := \frac{\dd t'}{a(t')}
\quad\text{[conformal time]}.
\end{align}
Explicitly, $\chi$ is given by
\begin{equation}
 \chi
  = \int_0^r\frac{\dd r'}{\sqrt{1-K r'^2}}
  = \int_0^{\tilde r}\frac{R_0\dd \tilde r'}{\sqrt{1-\tilde K \tilde r'^2}}
  = R_0\{\sin^{-1}\tilde r,\tilde r,\sinh^{-1}\tilde r\}
  = 2R_0\left\{\tan^{-1}\frac{\tilde \rho}{2}, \frac{\tilde\rho}{2},\tanh^{-1}\frac{\tilde \rho}{2}\right\}.
\end{equation}

The Christoffel symbol, Riemann tensor, Ricci tensor, and Ricci scalar are given by
\begin{align*}
 \tensor{\Gamma}{^n_i_j} &= \frac{g^{nk}}{2}(g_{jk,i}+g_{ik,j}-g_{ij,k}),
 &\tensor{R}{_i_j_k^l} &=
 \tensor{\Gamma}{^l_{jk,i}}
 -\tensor{\Gamma}{^l_{ik,j}}
 +\tensor{\Gamma}{^a_j_k}\tensor\Gamma{^l_a_i}
 -\tensor{\Gamma}{^a_i_k}\tensor\Gamma{^l_a_j}\footnotemark,
& {R}_{ij}&=\tensor{R}{_i_k_j^k},
& R &= g^{ij}R_{ij}.
\end{align*}
\footnotetext{Overall sign is convention-dependent.}%
%
We can obtain their explicit forms for the FLRW metric from, for example, Eq.~\eqref{eq:FLRW2} or \eqref{eq:FLRW3}, where $g_{00}=1$, $g_{0i}=g_{i0}=0$, and others depend on the choice of spatial coordinates.
Christoffel symbols are given by (for $a,b,c=1,2,3$)
\begin{equation}
  \tensor{\Gamma}{^0_{ab}}=\frac{-\dot a}{a}\tilde g_{ab},\quad
  \tensor{\Gamma}{^0_{00}}=\tensor{\Gamma}{^0_{0b}}=0,\quad
  \tensor{\Gamma}{^c_{00}}=0,\quad
  \tensor{\Gamma}{^c_{b0}}=\tensor{\Gamma}{^c_{0b}}=\frac{\dot a}{a}\delta^c_b;
  \quad
  \tensor{\Gamma}{^c_{ab}}\text{: coordinate dependent.}
\end{equation}
The Riemann tensor is given by, for both choices of the spacial coordinates,
\begin{equation}
  \tensor{R}{_{ijk}^l}=
    \frac{K+\dot a^2}{a^2} \left(g_{ik}\delta_j^l-g_{jk}\delta_i^l\right)
  + \frac{K+\dot a^2-a\ddot a}{a^2}\left(
  \delta_i^l \delta_{j=k=0}
  -\delta_j^l \delta_{i=k=0}
  +g_{jk} \delta_{i=l=0}
  -g_{ik} \delta_{j=l=0}
  \right)
\end{equation}
and hence (for $a,b=1,2,3$)
\begin{equation}
  \tensor{R}{_{00}}=
    \frac{3\ddot a}{a},\quad
  \tensor{R}{_{ab}}=
    \frac{2K+2\dot a^2+a \ddot a}{a^2} g_{ab},\quad
  R = \frac{6(K+\dot a^2+a \ddot a)}{a^2}.
\end{equation}
\subsection{Particle cosmology}
\subsubsection{Particle distribution and density}
In kinematic equilibrium, the particle number density, energy density, and pressure (i.e., (momentum)$\times$(flux per time) against a ``wall'') are calculated from distributions
\begin{alignat}{3}
 f\w{MB}(\vc k)&=\frac{g}{\ee^{(E-\mu)/T}},\qquad&
 f\w{BE}(\vc k)&=\frac{g}{\ee^{(E-\mu)/T}-1},\qquad&
 f\w{FD}(\vc k)&=\frac{g}{\ee^{(E-\mu)/T}+1},
\end{alignat}
\begin{alignat}{3}
 n   &=\int\ddP{k}f(\vc k),\qquad&
 \rho&=\int\ddP{k}Ef(\vc k),\qquad&
 p   &=\int\ddP{k}k_zv_z f(\vc k)
      =\int\ddP{k}\frac{k^2\cos^2\theta}{E} f(\vc k),
\end{alignat}
where we, in principle, take $\mu<m$ for Bose--Einstein statistics.
It is also useful to define the difference of particle and antiparticle number densities, assuming their chemical potential is opposite in sign:
\begin{equation}
  n_\Delta \deq n_+ - n_- = \int\ddP{k}\Bigl[f(\vc k,\mu) - f(\vc k,-\mu)\Bigr].
\end{equation}
\paragraph{Maxwell–Boltzmann}
For Maxwell--Boltzmann (MB) distribution, they are given explicitly by, with $\mu\deq \mu/T$ (understood) and $a\deq m/T$,
\begin{equation}
  n\w{MB} = \frac{gT^3\ee^{\mu}}{2\pi^2}a^2K_2(a),\quad
  %n\w{\Delta;MB} = g\sinh{\nu}\cdot\frac{T^3}{\pi^2}x^2K_2(x),\quad
  \rho\w{MB} = \left(3+\frac{aK_1(a)}{K_2(a)}\right)T n\w{MB},\quad
  p\w{MB} = T n\w{MB}.
\end{equation}
In special cases,
\begin{align}
  &\text{[massless: $m\ll T$]}
  &&n\w{MB} = \frac{gT^3\ee^{\mu}}{\pi^2}\left(1-\frac14a^2+\Order(a^4)\right),&&
  \rho\w{MB} = 3Tn\w{MB}\left(1+\frac16a^2+\Order(a^4)\right),
\\&\text{[low-temperature: $m\gg T$]}
  &&n\w{MB} \approx \ee^{(\mu-m)/T}\left({\frac{m T}{2\pi}}\right)^{3/2},&&
  \rho\w{MB} \approx \left(m+\frac32T\right)n\w{MB}.
\end{align}

\paragraph{Bose–Einstein and Fermi–Dirac}


For Fermi--Dirac or Bose--Einstein distributions, the results are not expressed analytically, even with $\mu=0$. Namely,
with $a=m/T$, $\mu=\mu/T$ (understood), and $\eta=\mu-a$,
\begin{align}
  &
  n^{\mathrm{BE}}_{\mathrm{FD}}
  = \int\frac{\dd^3\vc k}{(2\pi)^3}\frac{g}{\ee^{(E-\mu)/T}\mp1}
  = \frac{gT^3}{2\pi^2}\int_a^\infty\dd x\frac{x\sqrt{x^2-a^2}}{\ee^{x-\mu}\mp1}
  = \frac{gT^3}{2\pi^2}\int_0^\infty\dd x\frac{(x+a)\sqrt{x^2+2ax}}{\ee^{x-\eta}\mp1},
\\&
  \rho^{\mathrm{BE}}_{\mathrm{FD}}
  = \int\frac{\dd^3\vc k}{(2\pi)^3}\frac{gE}{\ee^{(E-\mu)/T}\mp1}
  = \frac{gT^4}{2\pi^2}\int_a^\infty\dd x\frac{x^2\sqrt{x^2-a^2}}{\ee^{x-\mu}\mp1}
  = \frac{gT^4}{2\pi^2}\int_0^\infty\dd x\frac{(x+a)^2\sqrt{x^2+2ax}}{\ee^{x-\eta}\mp1},
\\&
  p^{\mathrm{BE}}_{\mathrm{FD}}
  = \int\ddP{k}\frac{gk^2\cos^2\theta/E}{\ee^{(E-\mu)/T}\mp1}
  = \frac{gT^4}{6\pi^2}\int_a^\infty\dd x\frac{(x^2-a^2)^{3/2}}{\ee^{x-\mu}\mp1}
  = \frac{gT^4}{6\pi^2}\int_0^\infty\dd x\frac{(x^2+2ax)^{3/2}}{\ee^{x-\eta}\mp1}.
\end{align}
One method to evaluate these integrals is to utilize the expansion formula,
\begin{equation}
  \frac{1 - \ee^{-Nx}}{\ee^x\mp1}=\sum_{n=1}^N (\pm1)^{n+1}\ee^{-nx},\quad
  \frac{1 - \ee^{+Nx}}{\ee^x\mp1}=\sum_{n=1}^N (-1)(\pm1)^{n}\ee^{(n-1)x},
\end{equation}
where the former (latter) should be used for $x>0$ ($x<0$). Namely, if $\mu<a$,
\begin{equation}
  \int_a^\infty\dd x\frac{x\sqrt{x^2-a^2}}{\ee^{x-\mu}\mp1}
=\sum_{n=1}^\infty (\pm1)^{n+1} \int_a^\infty\dd x\,{x\sqrt{x^2-a^2}} \ee^{-n(x-\mu)}
=\sum_{n=1}^\infty (\pm1)^{n+1}\frac{a^2K_2(a n)}{n}\ee^{\mu  n},
\end{equation}
If $\mu<0$, we can use polylogarithm
\begin{equation}
  \mathop{\mathrm{Li}_s}(z) \deq \sum_{n=1}^\infty \frac{z^n}{n^s} \quad (|z|<1);\qquad
  \mathop{\mathrm{Li}_1}(z) = -\log(1-z)
\end{equation}
to evaluate the series expansion
\begin{equation}
   n^{\mathrm{BE}}_{\mathrm{FD}}\Big|_{\mu<0}
   = \frac{gT^3}{2\pi^2}\sum_{n=1}^N (\pm1)^{n+1}\left(\frac{2}{n^3}-\frac{a^2}{2n}+\Order(a^4)\right)\ee^{\mu  n}
   = \frac{gT^3}{2\pi^2}\left(\pm2\mathop{\mathrm{Li}_3}(\pm\ee^\mu)\mp\frac{a^2}{2}\mathop{\mathrm{Li}_1}(\pm\ee^\mu)\right).
\end{equation}
However, this method does not work for $\mu>0$.

If we are interested in the asymmetry, at $\Order(\mu)$,
\begin{equation}
  (n_\Delta)^{\mathrm{BE}}_{\mathrm{FD}}
  \approx
  \frac{gT^3}{2\pi^2}\cdot2\mu\int_a^\infty\dd x
  \frac{\ee^x x\sqrt{x^2-a^2}}{\left(\ee^x\mp1\right)^2}
  =
  \frac{gT^3}{2\pi^2}\cdot2\mu\int_a^\infty\dd x
  \frac{2 x^2- a^2}{\left(e^x\mp1\right) \sqrt{x^2-a^2}},
\end{equation}
which are often compared with $n^{\mathrm{eq}}\deq n|_{\mu=0}$. Namely,
\begin{equation}
  \frac{1}{\mu}\left(\frac{n_\Delta}{n^{\mathrm{eq}}}\right){}^{\mathrm{BE}}_{\mathrm{FD}}
  =
  \frac{\displaystyle
  \int_a^\infty\!\!
  \frac{(4 x^2-2a^2)\dd x}{\left(e^x\mp1\right) \sqrt{x^2-a^2}}
  }{
  \displaystyle\int_a^\infty\!\!\dd x\frac{x\sqrt{x^2-a^2}}{\ee^x\mp1}
  }
=  \frac{\displaystyle
  \int_0^\infty\!\!
  \frac{(4x^2+8ax+2a^2)\dd x}{\left(e^{x+a}\mp1\right) \sqrt{x^2+2ax}}
  }{
  \displaystyle\int_0^\infty\!\!\dd x\frac{(x+a)\sqrt{x^2+2ax}}{\ee^{x+a}\mp1}
  }
=\frac{\pi ^2}{3 \zeta_3}\times\begin{cases}\displaystyle
1-\frac{6 a}{\pi^2}+\Order(a^2)\\\displaystyle
\frac23 + \left(\frac{\log4}{9\zeta_3}-\frac{2}{\pi ^2}\right)a^2 + \Order(a^3).
\end{cases}
\end{equation}


\TODO{merge the above (new) with the below (old)}

Analytical expressions are available for the massless limit as
\begin{align}
    &\text{[massless: $m\ll T$]}
  &&
  \frac{n\w{FD}   }{T^3} = \frac{g L_3^-}{\pi^2}+\Order(x),\quad
  \frac{\rho\w{FD}}{T^4} = \frac{3g L_4^-}{\pi^2}+\Order(x^2),\quad
  \frac{p\w{FD}   }{T^4} = \frac{g L_4^-}{\pi^2}+\Order(x^2),\\
   &\text{[massless: $m\ll T$; $\mu<0$]}
  &&
  \frac{n\w{BE}   }{T^3} = \frac{g L_3^+}{\pi^2}+\Order(x),\quad
  \frac{\rho\w{BE}}{T^4} = \frac{3g L_4^+}{\pi^2}+\Order(x^2),\quad
  \frac{p\w{BE}   }{T^4} = \frac{g L_4^+}{\pi^2}+\Order(x^2),
\end{align}
where $L^\pm_n=\pm\mathop{\text{\texttt{PolyLog}}}(n,\pm\ee^{\mu/T})$.
Note we may need thermal masses for proper discussion, especially for the Bose--Einstein case, which is well-defined only for $\mu<m$.

The difference $n_\Delta$ can be expressed under the assumption $\mu/T\ll1$:
\begin{equation}
 [|\mu|\ll T]\qquad
 \left(n_\Delta{}\right){}^{\mathrm{BE}}_{\mathrm{FD}}
 =\int_{m}^\infty
 \frac{\dd E}{\pi^2} \frac{gE\sqrt{E^2-m^2}\ee^{E/T}}{(\ee^{E/T}\mp1)^2}\frac{\mu}{T}
 +\Order\left(\frac{\mu^3}{T^3}\right)
 =g\mu T^2\xi_\mp(x)
 +\Order\left(\frac{\mu^3}{T^3}\right),
\end{equation}
or rather,
\begin{equation}
 [|\mu|\ll T]\qquad
 \left(\frac{n_\Delta{}}{n^{\text{eq}}}\right){}^{\mathrm{BE}}_{\mathrm{FD}}
 =
 \frac{
 \displaystyle\int_{x}^\infty \dd k\, {2k\sqrt{k^2-x^2}\ee^{k}/(\ee^{k}\mp1)^2}
 }{
 \displaystyle\int_{x}^\infty \dd k\, {k\sqrt{k^2-x^2}/(\ee^{k}\mp1)}
 }\frac{\mu}{T}
 +\Order\left(\frac{\mu^3}{T^3}\right)
 =g\mu T^2\xi_\mp(x)
 +\Order\left(\frac{\mu^3}{T^3}\right),
\end{equation}
where $\displaystyle
 \xi_\mp(x)\deq \int_{x}^\infty\frac{k\ee^{k}\sqrt{k^2-x^2}\,\dd k}{(\ee^{k}\mp1)^2\pi^2}
$ can be calculated numerically and remain finite.
For fermions,
\begin{align}
 \text{[$\mu\ll T$, $m\ll T$]}\qquad
 (n_\Delta{})_{\mathrm{FD}} = g\mu\left[\frac{T^2}{6} - \frac{m^2}{4\pi^2}
 +\Order\left(\frac{m^4}{T^4}\right)\right]
 +\Order\left(\frac{\mu^2}{T^2}\right).
\end{align}
For Bose--Einstein case, it results in $\xi_-(0)=1/3$ and hence $n_{\Delta;\mathrm{BE}}\questeq g\mu T^2/3$ with no divergence, but we may discuss it is invalid because of the implicit assumption $\mu<m$.


\subsubsection{Thermodynamics}
The entropy density is given by\TODO{write the derivation}
\begin{equation}
    s = \frac{2\pi^2}{45}\left(2T_\gamma^3+\frac78g_\nu T_\nu^3+\cdots\right)\eqd \frac{2\pi^2}{45}g_{\ast s}T_\gamma^3.
\end{equation}
Adiabatic expansion leads to the conservation of entropy per comoving volume,
\begin{equation}
  sa^3=\frac{2\pi^2}{45}g_{\ast s}(aT_\gamma)^3=\text{const.},
  \qquad
  \frac{\dot s}{s} = -3\frac{\dot a}{a},\quad
  \diff{T_\gamma}{t} = -\left(\frac{\dot a}{a} + \frac{\dot g_{\ast s}}{3g_{\ast s}}\right)T_\gamma.\quad
\end{equation}
We define $H\deq \dot a/a$ as the Hubble parameter.


\subsubsection{Thermal average of cross section and decay rate}
A thermal average of a cross section $\sigma(s)$ is schematically given by
\begin{equation}
 \mean{\sigma v}_{AB\to12\cdots n}(T)
=\frac{1}{n_An_B}\int\ddP{k_A}\ddP{k_B}\left(
f_A f_B\right)\Bigl\{\phi_1 \cdots \phi_n\sigma(s)\Bigr\}
v_{\text{M\o l}};\qquad \phi_X = \ee^{(E-\mu)/T}f_X/g,
\end{equation}
Here, the final state statistical factor $\phi_1\cdots\phi_n$ are subject to the phase space integral of the calculation of $\sigma(s)$. They are specifically given by $\phi\w{MB}=1$, $\phi\w{BE}=1+f\w{BE}/g$, and $\phi\w{FD}=1-f\w{FD}/g$.
Similarly, a thermal averaged decay rate is given by
\begin{equation}
 \mean{\Gamma}_{A\to12\cdots n}
=\frac{1}{n_A}\int\ddP{k_A}
f_A \left\{\phi_1 \cdots \phi_n
\frac{m_A}{E_A}\Gamma\right\}.
\end{equation}


With MB approximation,
\begin{align}
 \mean{\sigma v}
&=
\frac{g_Ag_B}{n_An_B}
\ee^{(\mu_A+\mu_B)/T}
\int\ddP{k_A}\ddP{k_B}
\ee^{-(E_A+E_B)/T}
\sigma(s)v_{\text{M\o l}}
\\
&=
\int\frac{\dd s\dd E_+ \dd E_-}{32m_A^2 m_B^2 T^2 K_2(m_A/T)K_2(m_B/T)}
\ee^{-E_+/T}
4E_AE_B\sigma(s)v_{\text{M\o l}}\quad(\text{$\times1/2$ if $A=B$)}
\\
&=
\frac{1}{16m_A^2 m_B^2 T K_2(m_A/T)K_2(m_B/T)}
\int
\frac{K_1(\sqrt{s}/T) \dd s}{\sqrt{s}}
\sqrt{\Kallen(s,m_A^2,m_B^2)}
\cdot2E_A2E_Bv_{\text{M\o l}}\sigma(s)
\quad(\times1/2),
\\
 \mean{\Gamma}
 &=\frac{K_1(m_A/T)}{K_2(m_A/T)}\Gamma.
\end{align}

\subsubsection{Boltzmann Equation}
The Boltzmann equation is given by $\mathop{\hat L}f = \mathop{\hat C}f$ for the phase space density $f(x^\mu, k^\mu)$ of our interested particle.
Assuming $f$ is only dependent on $\|\vc k\|$ and $t$, we find, under FLRW metric,
\begin{equation}
\mathop{\hat L}f(E,t) =
\left[
k^\alpha \partial_\alpha - \Gamma^\alpha_{\beta\gamma} k^\beta k^\gamma \frac{\partial}{\partial k^\alpha}
\right]f(E,t)
= E\pdiff{f}{t} - \frac{\dot a}{a}k^2\pdiff{f}{E}.
\end{equation}
Meanwhile, the collision term is given by, if we consider $A+B\leftrightarrow C+D$,
 (see, e.g., Chap.~7 of Ref.~\cite{Cercignani2002})\footnote{
 Notice the difference in the notation: $\C{f}=f/h^3=f/(2\pi)^3$ and thus $\dd n=\C{f}\dd^3k$.
Total number of collisions (see Sec.~\ref{sec:xs}) is given by
 $n_An_Bv_{\text{M\o l}}\sigma VT\equiv\dd n\dd n_*v_{\text{M\o l}}\C{\sigma\dd\Omega}\dd^3x\Delta t$.
}
\begin{equation}
  \mathop{\hat C}f_A = \text{(elastic)} +
  \int(f'_Cf'_D-f_Af_B)(E_AE_B v_{\text{M\o l}})\dd\sigma\cdot2\dd\Pi_B,
= \text{(elastic)} +
  \int\left(\frac{f_Cf_D}{g_Cg_D}-\frac{f_Af_B}{g_Ag_B}\right)\frac{\dd\Pi_B}{2}\overline{\dd\Pi^{\text{out}}}\sum|\mathcal M|^2,
\end{equation}
where the final state statistical factors are omitted for simplicity. A careful treatment of the internal d.o.f.\ has been carried: as $\sigma$ corresponds to the forward process $AB\to CD$, the standard summation $\overline{\sum}$ is replaced by $\sum/(g_Ag_B)$ and the phase space density for $C$ and $D$ have received a correction from the internal d.o.f.
Now, we have\footnote{We here assume CP-invariance of the matrix element; separate treatment needs to be required otherwise.}
\begin{equation}
  \int\ddP{k}\frac1E\mathop{\hat L}f
  = \dot n + \frac{3\dot a}{a}n,\qquad
  \int\ddP{k}\frac1E\mathop{\hat C}f_A=
  \int\dd \Pi_A\dd\Pi_B\overline{\dd\Pi^{\text{out}}}
  \left(\frac{f_Cf_D}{g_Cg_D}-\frac{f_Af_B}{g_Ag_B}\right)\sum|\mathcal M|^2.
\end{equation}
By including the final state statistical factors
\begin{equation}
 \phi\w{MB}=1,\qquad
 \phi\w{BE}=1+f\w{BE}/g=\frac{1}{1-\ee^{-(E-\mu)/T}},\qquad
 \phi\w{FD}=1-f\w{FD}/g=\frac{1}{1+\ee^{-(E-\mu)/T}},
\end{equation}
or equivalently, $\phi=\ee^{(E-\mu)/T}f/g$, we obtain
\begin{align}
  \dot n_A + 3Hn_A&=
  \int\dd \Pi^{\text{in}}\overline{\dd\Pi^{\text{out}}}\left(
  \frac{\phi_A\phi_B f_C f_D}{g_Cg_D} - \frac{f_Af_B\phi_C\phi_D}{g_Ag_B}
  \right)\sum|\mathcal M|^2
\\&=
  \int\dd \Pi^{\text{in}}\overline{\dd\Pi^{\text{out}}}
  \frac{f_Af_Bf_C f_D}{g_Ag_Bg_Cg_D}\ee^{+E\w{in}/T}
  \left( \ee^{-(\mu_A+\mu_B)/T}-\ee^{-(\mu_C+\mu_D)/T} \right)\sum|\mathcal M|^2
\\&=
  \int\dd \Pi^{\text{in}}\overline{\dd\Pi^{\text{out}}}
  \phi_A\phi_B\phi_C\phi_D\ee^{-E\w{in}/T}\left(
  \ee^{(\mu_C+\mu_D)/T}
  -\ee^{(\mu_A+\mu_B)/T}
  \right)\sum|\mathcal M|^2,
\end{align}
with $H=\dot a/a$. The last equations are usually of use; as a general expression, we have
\begin{equation}
  \dot n + 3Hn=
  \int\dd \Pi^{\text{in}}\overline{\dd\Pi^{\text{out}}}
  \ee^{-E\w{in}/T}
  \left[\prod_{\text{in, out}}\phi\right]
  \left(
  \ee^{\sum_{\text{out}}\mu/T} - \ee^{\sum_{\text{in}}\mu/T}
  \right)\sum|\mathcal M|^2,
\end{equation}
where, in case identical particles are appearing in the initial or final states, additional factors need to be introduced in the phase space integral, statistical factors, and overall multipliers of the collision term.

The next step is typically to normalize the LHS by the entropy density $s$ or the photon number density $n_\gamma$:
\begin{alignat}{2}
  &Y\deq \frac{n}{s},\quad&&
  \dot n+3Hn
   = s\dot Y
  \\
  &\eta\deq\frac{n}{n_\gamma},&&
  \dot n+3Hn
  = n_\gamma\left( \dot\eta - \frac{\dot g_{\ast s}}{g_{\ast s}}\eta \right)
  = zHn_\gamma\cdot
  \frac{\dd \eta/\dd z-(\eta/g_{\ast s})(\dd{g_{\ast s}}/\dd z)}{1-(z/3g_{\ast s})(\dd{g_{\ast s}}/\dd z)},
\end{alignat}
where the latter case introduces $z\deq {\Lambda}/{T_\gamma}$ with an arbitrary scale $\Lambda$.
If one utilizes the Maxwell--Boltzmann approximation, the RHS is evaluated by introducing the equilibrium number density, $n^{\text{eq}} = n|_{\mu=0}$, so that $n_{\text{MB}} = \ee^{\mu/T} n^{\text{eq}}_{\text{MB}}$:
\begin{equation}
  \dot n_A + 3Hn_A
\approx
\left(
\frac{n_{C}}{n^{\text{eq}}_{C}}\frac{n_{D}}{n^{\text{eq}}_{D}}
- \frac{n_{A}}{n^{\text{eq}}_{A}}\frac{n_{B}}{n^{\text{eq}}_{B}}
\right)\w{MB}
  \int\dd \Pi^{\text{in}}\overline{\dd\Pi^{\text{out}}}
  \ee^{-E\w{in}/T}\sum|\mathcal M|^2
\eqd
\left(
\frac{n_{C}}{n^{\text{eq}}_{C}}\frac{n_{D}}{n^{\text{eq}}_{D}}
- \frac{n_{A}}{n^{\text{eq}}_{A}}\frac{n_{B}}{n^{\text{eq}}_{B}}
\right)\w{MB}\cdot\gamma^{\text{EQ}}.
\end{equation}
Otherwise, we can consider the time evolution of the asymmetry $n_\Delta\deq n_+ - n_-$ and evaluate only the first order term in $\mu$, assuming $|\mu|\ll T$ and ignoring the final state factors:
\begin{equation}
  \dot n_A + 3Hn_A \stackrel{\phi=1}{\approx}
  \int\dd \Pi^{\text{in}}\overline{\dd\Pi^{\text{out}}}
  \ee^{-E\w{in}/T}\left(
  \ee^{(\mu_C+\mu_D)/T}
  -\ee^{(\mu_A+\mu_B)/T}
  \right)\sum|\mathcal M|^2
\approx\frac{\mu_C+\mu_D-\mu_A-\mu_B}{T}\gamma^{\text{eq}}
\end{equation}
and one may replace $\mu/T$ by
\begin{equation}
  \frac{\mu}{T} =\frac{n_\Delta}{n^{\text{eq}}}\cdot\frac{3\zeta_3}{\pi^2}\times\text{$1$ ($3/2)$} = 0.365~(0.548)\times\frac{n_\Delta}{n^{\text{eq}}}\quad\text{for a boson (fermion)},
\end{equation}
but this is only 30\% different from the equation for the Maxwell-Boltzmann distribution, $\mu/T=0.5n_\Delta/n^{\text{eq}}$, and thus it seems straightforward, for most cases, to use the Maxwell--Boltzmann approximation from the beginning.



\end{document}




