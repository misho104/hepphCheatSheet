\documentclass[CheatSheet]{subfiles}

\begin{document}
\summarystyle
\section{Neutrino}

(summary page)

\detailstyle
\clearpage
\subsection{Convention and Nomenclature}
We define the neutrino mixing matrix $U$ (or $U_{\alpha i}$) by, following the standard convention,
\begin{equation}
 \ket{\nu_\alpha^{\text{flavor}}} = U_{\alpha i}^*\ket{\nu_i^{\text{mass}}},
\qquad
 \nu^\text{flavor} = U \nu^{\text{mass}}, \qquad[\alpha=e,\mu,\tau,\mathrm{r}_1,\mathrm{r}_2,\dots].\label{eq:U-defining}
\end{equation}
The mass basis are labelled by $i=1,2,\dots$, while $\alpha$ labels the ``flavor'' basis.
The flavor basis for $\alpha=e,\mu,\tau$ is defined by the charged leptons, i.e., so that the lepton doublets are aligned and $\nu_\alpha l^\pm_\beta W^\mp$ interactions are diagonal.
Meanwhile, for other ``extra'' neutrinos, the flavor basis can be left arbitrary (undefined) because it is usually not of our interests.

We, in this note, use the term ``PMNS (matrix)'' only if the neutrino mixing matrix is a $3\times3$ unitary matrix.

\subsection{Models}
\paragraph{Dirac neutrino model}
The simplest model for neutrino masses are given by extending the SM Yukawa \eqref{eq:SMyukawa} with ($n\w{RHN}$ copies of) right-handed neutrino $N$:
\begin{align}
   \mathcal L\w*{Dirac-$\nu$}&=
\left(  \overline{U}\Yu H \PL Q
- \overline{D}\Yd H^\dagger \PL Q
+ \overline{N}\Yn H \PL L
- \overline{E}\Ye H^\dagger \PL L
\right)+ \text{h.c.}
\\&=
\left(- \overline{Q^a}\Yu^\dagger \epsilon^{ab}H^{b*} \PR U
- \overline{Q^a}\Yd^\dagger H^{a} \PR D
- \overline{L^a}\Yn^\dagger \epsilon^{ab}H^{b*} \PR N
- \overline{L^a}\Ye^\dagger H^{a} \PR E
\right) + \text{h.c.}
\end{align}
Neutrinos become Dirac fermions and the discussion goes parallel to the CKM matrix:
\begin{equation}
 \Yn = U_n Y_n\wdiag V_n^\dagger,\qquad
\{\nuL, \eL, \overline N, \overline E\}^{\text{mass basis}} = \{V_n^\dagger L^1, V_e^\dagger L^2, \overline N U_n, \overline E U_e\},
\end{equation}
where  $\Yn\in\mathbb C^{n\w{RHN}\times3}$, $V_n\in\mathbb U_{\mathbb C}^3$, and $U_n\in\mathbb U_{\mathbb C}^{n\w{RHN}}$.
The Lagrangian is now given by
\begin{align}
 \mathcal L&\supset\overline{L}\ii\gamma^\mu(-\ii g_2 W_\mu)\PL L
\supset
\frac{g_2}{\sqrt2}\left[
\overline{L^1} \slashed W^+\PL L^2
+\overline{L^2}\slashed W^-\PL L^1
\right]
\\&=
\frac{g_2}{\sqrt2}\left[
\overline{(L^1)^{\text{mass}}} V_n^\dagger V_e\slashed W^+\PL (L^2)^{\text{mass}}
+\overline{(L^2)^{\text{mass}}}V_e^\dagger V_n\slashed W^-\PL(L^1)^{\text{mass}}
\right]
\\&=
\frac{g_2}{\sqrt2}\left[
\overline{\nuL^{\text{mass}}} U^\dagger\slashed W^+\eL^{\text{mass}}
+\overline{\eL^{\text{mass}}}U\slashed W^-\nuL^{\text{mass}}
\right],
\end{align}
where the definition \eqref{eq:U-defining} provides the equality
\begin{equation}
 U=V_e^\dagger V_n\equiv U\w{PMNS}.
\end{equation}
According to our terminology, this $3\times 3$ unitary matrix $U$ is called ``the PMNS matrix'' for this model.
Note that this is given in \emph{the opposite manner} to the CKM matrix ($V\w{CKM}=V_u^\dagger V_d$).

The mass eigenstate Dirac fields are given by
\begin{equation}
 \nu^{\text{mass}} = \pmat{\nuL^{\text{mass}}\\N^{\text{mass}}} = 
\nuL^{\text{mass}}+N^{\text{mass}}=
V_n^\dagger L^1 + U_n^\dagger N
\qquad
 \Bigl(L^1 = \PL V_n \nu^{\text{mass}},~
 N = \PR U_n \nu^{\text{mass}}\Bigr).
\end{equation}
Note that $n\w{RHN}=\{2,3\}$ because models with $n\w{RHN}<3$ yields $3-n\w{RHN}$ massless left-handed neutrinos, while $n\w{RHN}>3$ results in $n\w{RHN}-3$ massless right-handed neutrinos.

\paragraph{Type-I see-saw}
The type-I see-saw models are given by 
\begin{align}
   \mathcal L_{\text{type-I}}&=
\left(  \overline{U}\Yu H \PL Q
- \overline{D}\Yd H^\dagger \PL Q
+ \overline{N}\Yn H \PL L
- \overline{E}\Ye H^\dagger \PL L
- \frac{1}{2}\overline{N}M_N  N^\cc
\right)+ \text{h.c.}
\\&=
\left(- \overline{Q^a}\Yu^\dagger \epsilon^{ab}H^{b*} \PR U
- \overline{Q^a}\Yd^\dagger H^{a} \PR D
- \overline{L^a}\Yn^\dagger \epsilon^{ab}H^{b*} \PR N
- \overline{L^a}\Ye^\dagger H^{a} \PR E
- \frac{1}{2}\overline{N}M_N  N^\cc\right) + \text{h.c.},
\label{eq:type-1-lagrangian}
\end{align}
where $M_N$ is a $n\w{RHN}\times n\w{RHN}$ complex symmetric matrix.
Models with $n\w{RHN}\ge2$ are viable, because $n\w{RHN}<3$ gives $3-n\w{RHN}$ massless (Weyl) neutrinos and $2n\w{RHN}$ massive (Majorana) neutrinos, while all neutrinos acquire Majorana mass  if $n\w{RHN}\ge 3$.

We introduce left-handed Weyl spinors $\xi$ and $\chi$ to avoid notational confusion and diagonalize the Dirac mass term $M\w{D}$:
\begin{equation}
 L^1 \equiv \pmat{\xi\\0}\equiv \pmat{\nuL\\0},\quad
 N   \equiv \pmat{0\\\bar\chi}\equiv \pmat{0\\\nR},\quad
 M\w{D} := \frac{v}{\sqrt2}Y_n,\quad
 M\w{D}\wdiag := U_n M\w{D} Y_n^\dagger.
\end{equation}
Then the mass term is written in matrix form:
\begin{align}
\label{eq:MajoranaNuLag}
 \mathcal L_{\text{type-I}}
\supset
 -\frac{v}{\sqrt2}\overline{N}Y_n \PL L^1 - \frac{1}{2}\overline{N}M_N  N^\cc + \text{h.c.}
=-\frac12\pmat{\xi & \chi}\pmat{0 & M\w{D}^\TT \\ M\w{D} & M_N } \pmat{\xi \\ \chi} + \text{h.c.} =: -\frac12\tilde\nu^\TT \tilde M\tilde \nu+\text{h.c.},
\end{align}
where tildes denote ``larger'' objects. We can AT-diagonalize $\tilde M$, which is a symmetric $\mathbb{C}^{(3+n\w{RHN})\times(3+n\w{RHN})}$ matrix, as usual:
\begin{equation}
  \tilde M = \tilde R \tilde M\wdiag \tilde R^\TT;
\qquad
-\mathcal L\w*{Dirac-$\nu$}\supset
\frac12\tilde \nu^\TT \tilde M\tilde\nu
=\frac12(\tilde\nu^{\text{mass}})^\TT \tilde M\wdiag \tilde\nu^{\text{mass}};\qquad \tilde\nu^{\text{mass}} =\tilde R^\TT\tilde\nu.
\end{equation}
Mass eigenstates $\nu^{\text{mass}}$ are Majorana fermions given by left-handed Weyl spinors.
The neutrino mixing are given by
\begin{equation}
 \pmat{\xi\\\chi} \equiv \pmat{\nuL\\(\nR)^\dagger} = \tilde R^*\pmat{\nu_{1\TO3} \\ \nu_{4\TO}},\qquad
 \therefore
   U = \pmat{V_e^\dagger & 0 \\ 0 & X}\tilde R^*,
\label{eq:type-I-mixing}
\end{equation}
where $Y_e$ comes from the charged lepton mass diagonalization, while $X$ is a unitary matrix (but unphysical if, as usual, the right-handed neutrinos are indistinguishable).


\paragraph{Connection of the above two models}
If we set $M_N =0$ in the Type-I see-saw ($n\w{RHN}=3$), the AT-diagonalization becomes
\begin{align}
\tilde M = \frac12\pmat{0 & (U_nM\w{D}\wdiag V_n^\dagger)^\TT \\ U_nM\w{D}\wdiag V_n^\dagger & 0},\quad
  \tilde R=\pmat{V_n^* & \ii V_n^* \\ U_n & -\ii U_n},\quad
  \tilde M\wdiag=\pmat{M\w{D}\wdiag & 0 \\ 0 & M\w{D}\wdiag}.
\end{align}
As three pairs of degenerate Weyl fermions form three Dirac neutrinos, the Dirac neutrino model is reproduced as expected.


\subsection{Neutrino mixings}
Experiments have revealed that the upper-left $3\times3$ submatrix is close to a unitary matrix, which we call ``the PMNS matrix'' $U\w{PMNS}$.
Thus\footnote{Sho thanks Josu Hernandez-Garcia for discussion useful in this whole section.}
 we decompose the general neutrino mixing angle $U$ to
\begin{equation}
 U = U'\pmat{U\w{PMNS} & 0 \\ 0 & U_X}
  = \pmat{U'_{11} & U'_{12} \\ U'_{21} & U'_{22}}\pmat{U\w{PMNS} & 0 \\ 0 & U_X}
\end{equation}
with $U_X$ (and hence $U'$) being unitary.
Here $U'_{11}\approx1$ and $U'_{12},U'_{21}\approx0$.
We do not care much about $U'_{22}$ but in type-I see-saw models $U'_{22}U_X$ inherits the redundancy of $X$ and thus we can take, e.g., $U'_{22}\approx 1$ or $U_X=1$.

The matrix $U'$ is theoretically given with some matrix $\Theta$ by~\cite{Blennow:2011vn}\footnote{%
\TODO{}proof? or obvious if considers $\gSU(6)/(\gSU(3)\times\gSU(3))$?}
\begin{equation}
 U'\equiv \exp\pmat{0 & -\Theta^\dagger \\ \Theta & 0}
=\sum_{n=0}^\infty\pmat{
  {(-\Theta^\dagger\Theta)^{n}}/{(2n)!} &
  {-(-\Theta^\dagger\Theta)^{n}\Theta^\dagger}/{(2n+1)!}\\
  {+\Theta(-\Theta^\dagger\Theta)^{n}}/{(2n+1)!}&
  {(-\Theta\Theta^\dagger)^{n}}/{(2n)!}
}
\end{equation}
but practically we derive it in perturbative expansion.
For example, the type-I case is evaluated as\footnote{Cf.~\texttt{calculator/neutrino/diagonalization\_perturbative.wl}}
\begin{align}
 \text{(diag.)} = U^\TT\pmat{0 & M\w{D}^\TT \\ M\w{D} & M_N} U;\qquad
&U'\approx\pmat{
  1-\frac12 \Theta_0^\dagger \Theta_0&
  -\Theta_0^\dagger\\
 \Theta_0&
  1-\frac12\Theta_0\Theta_0^\dagger
};\quad \Theta_0:=-M_N^{-1}M\w{D},
\label{eq:typeIseesaw-blockdiag-1}\\
&\text{(block diag.)}=U'^\TT\pmat{0 & M\w{D}^\TT \\ M\w{D} & M_N} U'\approx\pmat{
  -M\w{D}^\TT M_N^{-1}M\w{D} & 0 \\ 0 & M_N
},
\label{eq:typeIseesaw-blockdiag-2}
\end{align}
where however note that $\Theta_0$ is close but not equal to $\Theta$.

If we focus only on the upper-left $3\times3$ part and are allowed to neglect $\Theta^4$ (or $\Theta_0^4$) terms, the above decompositions of a unitary matrix into two unitary matrices~\cite{Blennow:2016jkn}
can be expressed by
\begin{equation}
 [U]\w{upper left} \simeq (1-\eta)U\w{PMNS};\qquad
  \text{$\eta:=\Theta^\dagger\Theta/2$ is a complex positive-semidefinite Hermitian matrix},
\end{equation}
where the positive-semidefiniteness yields
\begin{equation}
 \eta_{ii}\ge0,\quad |\eta_{ij}|\le\sqrt{\eta_{ii}\eta_{jj}}\le(\eta_{ii}+\eta_{jj})/2.
\end{equation}
The QR decomposition is also useful~\cite{Blennow:2016jkn}; together with the polar decomposition, it is summarized by
\begin{alignat}{2}
 [U]\w{upper left} &= (1-\alpha)U\w{QR};\qquad&
  &\text{QR decomposition; $\alpha$ is a complex upper-triangle matrix},\\
 [U]\w{upper left} &= \eta'U\w{polar};\qquad&
  &\text{polar decomposition; $\eta'$ is a complex positive-semidefinite Hermitian matrix},
\end{alignat}
where  $U\w{QR}\neq U\w{polar}$.
However, as experiments claim that $[U]\w{upper left} \approx U\w{PMNS}$ and $\eta$ and $\alpha$ are still consistent with zero, they are consistent with $U\w{PMNS}$.
Note also that, as $[U]\w{upper left}$ seems to be invertible, these decompositions are given uniquely.


\subsection{PMNS matrix}\label{sec:PMNS}
As experiments have not found any deviation from $[U]\w{upper left}$ being unitary, we express the mixing with the PMNS matrix
 \GRAY{(Pontecorvo--\JAPANESE{牧}--\JAPANESE{中川}--\JAPANESE{坂田})}
parameterized by
\begin{equation}
\begin{split}
   U\w{PMNS}&=
 \pmat{1&&\\&\co{23}&\si{23}\\&-\si{23}&\co{23}}
 \pmat{\co{13}&&\si{13}\ee^{-\ii\delta\w{CP}}\\&1&\\-\si{13}\ee^{\ii\delta\w{CP}}&&\co{13}}
 \pmat{\co{12}&\si{12}&\\-\si{12}&\co{12}&\\&&1}
\pmat{\ee^{\ii\eta_1} \\ & \ee^{\ii\eta_2}\\&&1}
 %
\\&=
 \pmat{
 \co{12} \co{13} & \si{12} \co{13} & \si{13} \ee^{-\ii\delta\w{CP}}\\
 -\si{12} \co{23} - \co{12} \si{23} \si{13} \ee^{\ii\delta\w{CP}}& \co{12} \co{23} - \si{12}\si{23}\si{13} \ee^{\ii\delta\w{CP}}& \si{23}\co{13}\\
  \si{12}\si{23} - \co{12} \co{23} \si{13}\ee^{\ii\delta\w{CP}} & -\co{12}\si{23}-\si{12}\co{23}\si{13}\ee^{\ii\delta\w{CP}} & \co{23} \co{13}
}\pmat{\ee^{\ii\eta_1} \\ & \ee^{\ii\eta_2}\\&&1},
\end{split}
\end{equation}
where $\theta_{ij}\in[0,\pi/2]$ and $\delta\w{CP}\in[0,2\pi]$ as shown in \cref{eq:GeneralUnitary33}.
In the Dirac neutrino model, $\eta_1$ and $\eta_2$ (``Majorana phases'') are unphysical because we can rotate $\nR$ to eliminate them, while the rotation is not allowed in presence of the Majorana mass term.

It should be noted that the discussion in \cref{sec:SM-FFdual} holds.
As far as the baryon number is conserved, we can remove the $\Theta_W W\tilde W$ term by quark rotation.
Hence, the above-discussed models have CP violation only in the CKM matrix and the neutrino mixing matrix.




\paragraph{PDG and NuFIT convention}
Our convention is the same with PDG \cite[\S14]{PDG2020}~\footnote{Sho thinks Eq.~(14.9) of PDG2020 lacks 1/2 in the right-most term.}.

It also agrees with NuFIT~\cite[v5.0]{NUFIT} (their convention is given in Ref.~\cite{Esteban:2018azc}) except for the Majorana phases,
$
 \eta_i = \C{\alpha_i}.
$



\subsection{Casas--Ibarra parameterization}
We start from the approximation given in Eq.~\eqref{eq:typeIseesaw-blockdiag-2}:
\begin{equation}
 m\w L \approx -U\w{PMNS}^\TT M\w{D}^\TT M_N^{-1}M\w{D} U\w{PMNS},\qquad
 m\w H \approx U_X^\TT M_N U_X,
\end{equation}
where $m\w L$ and $m\w H$ are diagonal mass matrices for three light neutrinos and heavier neutrinos.
We then obtain
\begin{equation}
  m\w L
\approx -U\w{PMNS}^\TT M\w{D}^\TT (U_X^* m\w H U_X^\dagger)^{-1}M\w{D} U\w{PMNS}
=       -U\w{PMNS}^\TT M\w{D}^\TT U_Xm\w H^{-1}U_X^\TT M\w{D} U\w{PMNS}
\end{equation}
and decompose it as
\begin{equation}
   [\ii m\w L]^{1/2}[\ii m\w L]^{1/2}
\approx     \left(\frac{v}{\sqrt2}m\w H^{-1/2}U_X^\TT Y_n U\w{PMNS}\right)^{\TT} \left(\frac{v}{\sqrt2} m\w H^{-1/2}U_X^\TT Y_n U\w{PMNS}\right),
\end{equation}
which is the master equation for Casas-Ibarra parameterization~\cite{Casas:2001sr}.
Note that $U_X$ is unphysical, i.e., we can fix the basis of $\nR$ so that $U_X=1$.
This basis is equivalent (under our approximation) to the basis in which $M_N$ becomes diagonal.

\paragraph{Example: three right-handed neutrinos}
Let us assume all the neutrinos are massive thanks to three right-handed neutrinos. Then $M\w L\wdiag$ is invertible and
\begin{equation}
 R:=-\ii m\w H^{-1/2}U_X^\TT M\w{D}  U\w{PMNS}  m\w L^{-1/2}
\quad\Longrightarrow\quad R^\TT R\approx1.
\end{equation}
Conversely, with a matrix $R$ satisfying $R^\TT R=1$, the Yukawa matrix is given by
\begin{equation}
 \frac{v}{\sqrt2}Y_n
  \approx \ii U_X^*\sqrt{m\w H}R\sqrt{m\w L}U\w{PMNS}^\dagger
\end{equation}
or
\begin{equation}
 \frac{v}{\sqrt2}Y_n\Big|\w*{$M_N$ being diagonal}
  \approx \ii \sqrt{m\w H}R\sqrt{m\w L}U\w{PMNS}^\dagger.
\end{equation}

Now we successfully parameterized $Y_n$ by a ``complex orthogonal'' matrix $R$, which can be parameterized by\footnote{For $w\in\mathbb C$, $\sin z_1=w$ and $\cos z_2=w$ always have solutions $z_{1,2}\in\mathbb C$. Meanwhile, $\tan z=w$ has no solution if and only if $w=\pm \ii$. Then, using this fact, one first expresses $R_{i3}$ components by $\zeta\w{A,B,C}=\pm1$ and $\theta\w{A,B}\in\mathbb C$, restricting $0\le\Re\theta\w{A,B}\le\pi/2$ ($\Leftrightarrow\Re{\sin{\theta}}\ge0\land\Re\cos\theta\ge0$), and then gets an expression of $R$ with three angles and six signs. Five signs are absorbed by enlarging $\Re\theta$ and one sign remains, which is $\zeta$.}
\begin{equation}
 R=\pmat{
 \co{12} \co{13} & \si{12} \co{13} & \si{13} \\
 -\zeta \si{12} \co{23}-\co{12} \si{23} \si{13} & \zeta \co{12} \co{23} -\si{12} \si{23} \si{13} & \si{23} \co{13} \\
  \zeta \si{12} \si{23}-\co{12} \co{23} \si{13} & -\zeta\co{12} \si{23} -\si{12} \co{23} \si{13} & \co{23} \co{13}
},
\end{equation}
where $\co{12}\equiv \cos{\theta_{12}}$ etc.\ and
\begin{equation}
 \zeta=\pm1; \qquad
(\theta_{12},\theta_{23},\theta_{13})\in\mathbb C,\quad
|\Re\theta_{12}|\le\pi, \quad |\Re\theta_{23}|\le\pi,  \quad |\Re\theta_{13}|\le\frac{\pi}{2}.
\end{equation}
This $R$ satisfies $RR^\TT=1$, which however is not general (as in the next example).

With this parameterization, the neutrino mixing matrix is given by, in the basis with $M_N$ being diagonal,
\begin{equation}
 U \approx \pmat{
  U\w{PMNS}
  &  M\w{D}^\dagger M_N^{*-1} U_X
  \\ -M_N^{-1} M\w{D} U\w{PMNS}
  &  U_X
}
\approx\pmat{U\w{PMNS}&0\\0&1}\pmat{
  1
  &  -(-\ii{m\w H^{-1/2}}R{m\w L^{1/2}})^\dagger
  \\ -\ii m\w H^{-1/2}R{m\w L^{1/2}}
  &  1
}.
\end{equation}

\paragraph{Example: two right-handed neutrinos}
For models with two right-handed neutrinos, one neutrino is massless and $M\w L\wdiag$ is not invertible. However the parameterization
\begin{equation}
 \frac{v}{\sqrt2}Y_n
  \approx \ii U_X^*\sqrt{m\w H}R\sqrt{m\w L}U\w{PMNS}^\dagger
\end{equation}
works with\footnote{See, e.g., Ref.~\cite{Brdar:2019iem}. Sho also thanks Kai Schmitz for his note.}
\begin{align}
 R_{\text{normal hierarchy}}&=\pmat{0 & \cos z & \zeta\sin z \\ 0 & -\sin z & \zeta \cos z},&
 R_{\text{inverse hierarchy}}=\pmat{\cos z & \zeta\sin z &0 \\ -\sin z & \zeta \cos z & 0},
\end{align}
where $z\in\mathbb C$ and $\zeta=\pm1$.

\end{document}
