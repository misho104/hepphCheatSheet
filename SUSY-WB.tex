%#!latexmk CheatSheet.tex
%%% Time-Stamp: <2010-11-02 23:13:44 misho>

\section{Supersymmetry in the text by Wess \& Bagger}
\subsection{Spinor Convention}
\begin{tabular}{l@{ :\ \ \ }l}
$\epsilon$ tensor
  &$\epsilon^{12}=\epsilon^{\dot1\dot2}=\epsilon_{21}=\epsilon_{\dot2\dot1}=1$
   \qquad{\footnotesize (definition)}\\
Sum Rule & ${}^\alpha{}_{\alpha}$ and ${}_\dalpha{}^{\dalpha}$, except for
  \quad
  $\xi_\alpha=\epsilon_{\alpha\beta}\xi^\beta,\quad
   \xi^\alpha=\epsilon^{\alpha\beta}\xi_\beta,\quad
   \xi_\dalpha=\epsilon_{\dalpha\dbeta}\xi^\dbeta,\quad
   \xi^\dalpha=\epsilon^{\dalpha\dbeta}\xi_\dbeta$.\\[.5zw]
Lorentz�ϊ�&
  $\psi'_\alpha = \Lambda\T_\alpha^\beta\psi_\beta,\quad
   \bar\psi'_\dalpha = \bar\psi_\dbeta\Lambda^\dagger\T^\dbeta_\dalpha,\quad
   \psi'^\alpha = \psi^\beta\Lambda^{-1}\T_\beta^\alpha,\quad
   \bar\psi'^\dalpha = (\Lambda^{-1})^\dagger\T^\dalpha_\dbeta\bar\psi'^\dbeta.
  $\\[.5zw]
$\sigma$ matrices& $
 (\Sm)_{\alpha\dbeta} := ({\RED-1},\vc\sigma)_{\alpha\dbeta},\quad
 (\bSm)^{\dalpha\alpha} :=\epsilon^{\dalpha\dbeta}\epsilon^{\alpha\beta}(\Sm)_{\beta\dbeta}
 =({\RED-1},-\vc\sigma)^{\dalpha\beta}$.
\end{tabular}
\begin{flushright}
{\footnotesize (See App.~\ref{sec:verbose-lorentz-group} for a verbose explanation.)}
\end{flushright}
\vspace{-3zw}

\subsection{Spinor Calculation Cheatsheet}
\begin{preposition}{}\vspace{-.5zw}
\begin{align*}
 \eta = {\RED(-,+,+,+)},\qquad\epsilon^{0123}=-\epsilon_{0123} = 1
\end{align*}\vspace{-2.8zw}
\begin{align*}
  \epsilon^{12}=\epsilon_{21}=\epsilon^{\dot1\dot2}=\epsilon_{\dot2\dot1}&=1,&
\xi^\alpha&:=\epsilon^{\alpha\beta}\xi_\beta,& \xi_\alpha&=\epsilon_{\alpha\beta}\xi^\beta,&
\bar\xi^\dalpha&:=\epsilon^{\dalpha\dbeta}\bar\xi_\dbeta,& \bar\xi_\dalpha&=\epsilon_{\dalpha\dbeta}\bar\xi^\dbeta
\end{align*}\vspace{-2.8zw}
\begin{align*}
  \bSm{}^{\dalpha\alpha} &:= \epsilon^{\alpha\beta}\epsilon^{\dalpha\dbeta}\Sm_{\beta\dbeta}&
 \Sm_{\alpha\dalpha}&= \epsilon_{\alpha\beta}\epsilon_{\dalpha\dbeta}\bSm{}^{\dbeta\beta},&
 \Sm:=({\RED-1},\vc\sigma),\quad\bSm:=({\RED-1},-\vc\sigma)
\end{align*}\vspace{-2.8zw}
\begin{align*}
  (\Smn)\T_\alpha^\beta:=\tfrac14\left(\Sm\bSn-\Sn\bSm\right)\T_{\alpha}^\beta,\quad
 (\bSmn)\T^\dalpha_\dbeta:=\tfrac14\left(\bSm\Sn-\bSn\Sm\right)\T^{\dalpha}_\dbeta =
(\Snm)^\dagger{}\T^\dalpha_\dbeta.
\end{align*}\vspace{-2.7zw}
\end{preposition}\vspace{-1.2zw}

\begin{align*}
&\theta^\alpha\theta^\beta=-\tfrac12\epsilon^{\alpha\beta}\thth&
&\theta_\alpha\theta_\beta=\tfrac12\epsilon_{\alpha\beta}\thth&
(\theta\phi)(\theta\psi)&=-\tfrac12(\psi\phi)(\thth)&
(\theta\Sm\btheta)(\theta\Sn\btheta)&={\RED-\tfrac12}\thth\bthth\eta^{\mu\nu}
\\
&\btheta^\dalpha\btheta^\dbeta=\tfrac12\epsilon^{\dalpha\dbeta}\bthth&
&\btheta_\dalpha\btheta_\dbeta=-\tfrac12\epsilon_{\dalpha\dbeta}\bthth&
(\btheta\bar\phi)(\btheta\bar\psi)&=-\tfrac12(\bar\psi\bar\phi)(\bthth)&
 (\Sm\btheta)_\alpha(\theta\Sn\btheta) &= \tfrac12(\Sm\bSn\theta)_\alpha\bthth
\\
&\theta\Sm\bSn\theta={\RED-\Hmn}\thth&
&\btheta\bSm\Sn\btheta={\RED-\Hmn}\bthth&
&&
 (\theta\Sm)_\dalpha(\theta\Sn\btheta) &= \tfrac12(\btheta\bSn\Sm)_\dalpha\thth
\end{align*}
\vspace{-2zw}
\begin{align*}
 &\Sm\bSn={\RED-\Hmn}+2\Smn&
 &\Sm\bSr\Sn+\Sn\bSr\Sm={\RED-2}\left(\Hmr\Sn+\Hnr\Sm-\Hmn\Sr\right)\\
 &\bSm\Sn={\RED-\Hmn}+2\bSmn&
 &\bSm\Sr\bSn+\bSn\Sr\bSm={\RED-2}\left(\Hmr\bSn+\Hnr\bSm-\Hmn\bSr\right)\\
 &\Smn=-\Snm&
 &\Sm\bSn\Sr-\Sr\bSn\Sm=2\ii\epsilon^{\mu\nu\rho\sigma}\Ssd\\
 &\bSmn=-\bSnm&
 &\bSm\Sn\bSr-\bSr\Sn\bSm=-2\ii\epsilon^{\mu\nu\rho\sigma}\bSsd\\
 &\Tr\bSm\Sn=\Tr\Sm\bSn={\RED-2}\Hmn&
 &\Tr\Smn\Srs=-\tfrac12\left(\Hmr\Hns-\Hms\Hnr\right){\RED-\tfrac\ii2}\epsilon^{\mu\nu\rho\sigma}\\
 &\Tr\Smn=\Tr\bSmn=0&
&\Tr\bSmn\bSrs=-\tfrac12\left(\Hmr\Hns-\Hms\Hnr\right){\RED+\tfrac\ii2}\epsilon^{\mu\nu\rho\sigma}&
\\
&\Sm_{\alpha\dalpha}\bSmd^{\dbeta\beta}={\RED-2}\delta^\beta_\alpha\delta^\dbeta_\dalpha&
&\Sm_{\alpha\dalpha}\Sn_{\beta\dbeta}-\Sn_{\alpha\dalpha}\Sm_{\beta\dbeta}
   = 2\left[(\Smn\epsilon)_{\alpha\beta}\epsilon_{\dalpha\dbeta}
           +(\epsilon\bSmn)_{\dalpha\dbeta}\epsilon_{\alpha\beta}\right]&
\\
&\Sm_{\alpha\dalpha}\Smd{}_{\beta\dbeta}={\RED-2}\epsilon_{\alpha\beta}\epsilon_{\dalpha\dbeta}&
&\Sm_{\alpha\dalpha}\Sn_{\beta\dbeta}+\Sn_{\alpha\dalpha}\Sm_{\beta\dbeta}
   = {\RED-\Hmn}\epsilon_{\alpha\beta}\epsilon_{\dalpha\dbeta}
      {\RED+4\Hrsd}(\Srm\epsilon)_{\alpha\beta}(\epsilon\bSsn)_{\dalpha\dbeta}&
\\
&\bSm{}^{\dalpha\alpha}\bSmd^{\dbeta\beta}={\RED-2}\epsilon^{\alpha\beta}\epsilon^{\dalpha\dbeta}&
&\epsilon_{\dbeta\dalpha}\bSm{}^{\dalpha\alpha}=\epsilon^{\alpha\beta}\Sm_{\beta\dbeta}
\qquad\qquad
\epsilon^{\mu\nu\rho\sigma}\sigma_{\rho\sigma}={\RED-2}\ii\Smn&
\\
&\sigma\T^\mu^\nu_\alpha^\beta\epsilon_{\beta\gamma} =
 \sigma\T^\mu^\nu_\gamma^\beta\epsilon_{\beta\alpha}&
&\epsilon_{\beta\alpha}\bSm{}^{\dalpha\alpha}=\epsilon^{\dalpha\dbeta}\Sm_{\beta\dbeta}
\qquad\qquad
\epsilon^{\mu\nu\rho\sigma}\bar\sigma_{\rho\sigma}={\RED2}\ii\bSmn&
\end{align*}
\vspace{-2.5zw}
\begin{align*}
 \bar\xi\bSm\chi&=-\chi\Sm\bar\xi=(\bar\chi\bSm\xi)^*=-(\xi\Sm\bar\chi)^*&
 (\psi\phi)\chi_\alpha&=-(\phi\chi)\psi_\alpha-(\chi\psi)\phi_\alpha
\\
 \xi\Sm\bSn\chi&=\chi\Sn\bSm\xi=(\bar\chi\bSn\Sm\bar\xi)^*=(\bar\xi\bSm\Sn\bar\chi)^*&
 (\psi\phi)\bar\chi_\dalpha&={\RED-\tfrac12}(\phi\Sm\bar\chi)(\psi\Smd)_\dalpha
\end{align*}
In the following equations, we chose left-differential notation.
\begin{align*}
 \epsilon^{\alpha\beta}\pbib{}{\theta^\beta}&=-\pbib{}{\theta_\alpha}&
 \pbib{}{\theta^\alpha}\thth&=2\theta_\alpha&
 \epsilon^{\alpha\beta}\pbib{}{\theta^\alpha}\pbib{}{\theta^\beta}\thth &= 4\\
 \epsilon_{\alpha\beta}\pbib{}{\theta_\beta}&=-\pbib{}{\theta^\alpha}&
 \pbib{}{\theta_\alpha}\thth&=-2\theta^\alpha&
 \epsilon_{\alpha\beta}\pbib{}{\theta_\alpha}\pbib{}{\theta_\beta}\thth &= -4\\
 \epsilon^{\dalpha\dbeta}\pbib{}{\btheta^\dbeta}&=-\pbib{}{\btheta_\dalpha}&
 \pbib{}{\btheta^\dalpha}\bthth&=-2\btheta_\dalpha&
 \epsilon_{\dalpha\dbeta}\pbib{}{\btheta_\dalpha}\pbib{}{\btheta_\dbeta}\bthth &= 4\\
 \epsilon_{\dalpha\dbeta}\pbib{}{\btheta_\dbeta}&=-\pbib{}{\btheta^\dalpha}&
 \pbib{}{\btheta_\dalpha}\bthth&=2\btheta^\dalpha&
 \epsilon^{\dalpha\dbeta}\pbib{}{\btheta^\dalpha}\pbib{}{\btheta^\dbeta}\bthth &= -4
\end{align*}


\newpage
\subsection{Chiral Superfields : $\bar D_\dalpha \Phi = 0$}
\paragraph{Explicit Expression}
\begin{align}
\Phi
={}& \phi(y) + \sqrt2\theta\psi(y)+\thth F(y)\\
={}& \phi(x){\RED\,+\,} \ii\theta\Sm\btheta\Pm\phi(x)
               {\RED\,+\,} \frac14\thth\bthth\Psq\phi(x)
     + \sqrt2\theta\psi(x){\RED\,+\,}\frac\ii{\sqrt2}\thth\btheta\bSm\Pm\psi(x)
     + \thth F(x)\\
\Phi^\dagger
={}& \phi^*(x){\RED\,-\,} \ii\theta\Sm\btheta\Pm\phi^*(x)
              {\RED\,+\,} \frac14\thth\bthth\Psq\phi^*(x)
   + \sqrt2\btheta\bar\psi(x){\RED\,-\,} \frac\ii{\sqrt2}\bthth\theta\Sm\Pm\bar\psi(x)
   + \bthth F^*(x)
\end{align}

%\paragraph{Changing Bases}
%\begin{align}
% \phi(y)
%={}& \phi(x) + \ii\theta\Sm\btheta\Pm\phi(x)+\frac14\thth\bthth\Psq\phi(x)\\
%={}& \phi(y^+) + 2\ii\theta\Sm\btheta\Pm\phi(y^+)+\thth\bthth\Psq\phi(y^+)\\
% \phi(y^+)
%={}& \phi(x) - \ii\theta\Sm\btheta\Pm\phi(x)+\frac14\thth\bthth\Psq\phi(x)\\
%={}& \phi(y) - 2\ii\theta\Sm\btheta\Pm\phi(y)+\thth\bthth\Psq\phi(y)\\
% \phi(x)
%={}& \phi(y) - \ii\theta\Sm\btheta\Pm\phi(y)+\frac14\thth\bthth\Psq\phi(y)\\
%={}& \phi(y^+) + \ii\theta\Sm\btheta\Pm\phi(y^+)+\frac14\thth\bthth\Psq\phi(y^+)
%\end{align}

\paragraph{Product of Chiral Superfields}
\begin{align}
\Phi^\dagger_i\Phi_j(x,\theta,\btheta)
\leadsto{}
   & \phi^*_i\phi_j
       + \sqrt2\phi^*_i\theta\psi_j + \sqrt2\btheta\bar\psi_i\phi_j
       + \thth \phi^*_iF_j + \bthth F^*_i\phi_j\notag\\
   & {\RED\,+\,} 2\ii(\theta\Sm\btheta)(\phi^*_i\Pm\phi_j)
       {\RED\,-\,} \sqrt2\ii\thth(\Pm\phi^*_i)\btheta\bSm\psi_j
       {\RED\,-\,} \sqrt2\ii\bthth\theta\Sm\bar\psi_i\Pm\phi_j\notag\\
   & + 2\btheta\bar\psi_i\theta\psi_j
       + \sqrt2\thth \btheta\bar\psi_i F_j
       + \sqrt2\bthth F^*_i\theta\psi_j\notag\\
   & + \thth\bthth\left[
        F^*_iF_j {\RED\,-\,} \Pm\phi^*_i\Pmd\phi_j {\RED\,-\,} \ii\bar\psi_i\bSm\Pm\psi_j\right]
\end{align}
\begin{align}
\Phi_i\Phi_j\text{(in $y$-basis)}
={}& \phi_i\phi_j+\sqrt2\theta\left[\psi_i\phi_j+\phi_i\psi_j\right]
     +\thth\left[\phi_iF_j+F_i\phi_j-\psi_i\psi_j\right]\\
\Phi_i\Phi_j\Phi_k\text{(in $y$-basis)}
={}& \phi_i\phi_j\phi_k
     + \sqrt2\theta\left[\psi_i\phi_j\phi_k
                        +\phi_i\psi_j\phi_k
                        +\phi_i\phi_j\psi_k\right]\notag\\
   & +\thth\left[
      F_i\phi_j\phi_k+\phi_iF_j\phi_k+\phi_i\phi_jF_k
      -\psi_i\psi_j\phi_k-\psi_i\phi_j\psi_k
      -\phi_i\psi_j\psi_k
      \right]
\end{align}

Note that products of chiral superfields $\Phi_1\Phi_2\cdots$ are again
chiral superfields.


%%% Local Variables:
%%% TeX-master: "CheatSheet.tex"
%%% End:
