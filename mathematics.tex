\documentclass[CheatSheet]{subfiles}
\begin{document}

\detailstyle
\section{Mathematics}
\subsection{Matrix exponential}
Excerpted from \S2 and \S5 of Hall 2015 \cite{Hall2015}:
\begin{alignat}{2}
 &\ee^X \deq \sum_{m=0}^\infty \frac{X^m}{m!} \text{~~(converges for any $X$)},
\quad&
 &\log X \deq \sum_{m=1}^\infty (-1)^{m+1}\frac{(A-1)^m}{m} \text{~~(conv.~if $\|A-I\|<1$)}.
\end{alignat}
\begin{alignat}{2}
 &\ee^{\log A} = A \text{~~(if $\|A-I\|<1$)},
\quad&
 &\log \ee^X = X \text{~and~} \|\ee^X-1\| < 1 \text{~~(if $\|X\|<\log2$).}
\end{alignat}
\begin{equation}
 \text{Hilbert-Schmidt norm:}~\|X\|^2 \deq \sum_{i,j}|X_{ij}|^2 = {\Tr X^\dagger X}.
\end{equation}
Properties:
\begin{alignat*}{4}
 \ee^{(X^\TT)} &= (\ee^X)^\TT,
&\qquad
 \ee^{(X^*)} &= (\ee^X)^*,
&\qquad
 (\ee^X)^{-1} &= \ee^{-X},
&\qquad
 \ee^{YXY^{-1}} &= Y\ee^X Y^{-1},
\end{alignat*}
\begin{alignat*}{2}
 \det \exp X &= \exp\Tr X,
&\qquad
 \ee^{(\alpha+\beta)X} &= \ee^{\alpha X}\ee^{\beta X} \text{~for $\alpha, \beta\in \mathbb C$};
\end{alignat*}
Baker-Campbell-Hausdorff:
\begin{align}
  \ee^X Y \ee^{-X}
&= Y + [X, Y] + \frac1{2!}[X, [X, Y]] + \frac1{3!}[X, [X, [X, Y]]] + \cdots = \ee^{[X,]}Y;
\\
  \ee^X \ee^Y \ee^{-X}
&= \sum_{n=0}^\infty\frac{1}{n!}(\ee^X Y \ee^{-X})^n
 = \exp\left(\ee^{[X,]}Y\right);
\\
  \log(\ee^X \ee^Y)
&= X + \int_0^1\!\dd t\, g(\ee^{[X,}\ee^{t[Y,})Y
\qquad\Bigl[
  g(z) = \frac{\log z}{1-z^{-1}} = 1-\sum_{n=1}^{\infty}\frac{(1-z)^n}{n(n+1)};
\quad
g(\ee^y)=\sum_{n=0}^{\infty}\frac{B_ny^n}{n!}
\Bigr]\\
&= X + Y + \frac12[X,Y] + \frac1{12}[X, [X,Y]] - \frac1{12}[Y, [X,Y]] + \cdots
\quad\text{(Baker-Campbell-Hausdorff).}
\end{align}
\begin{align}
   \log(\ee^X\ee^Y)
&=\sum_{k=1}^\infty \frac{(-1)^{k-1}}{k}\left(\sum_{m,n=0}^{\infty}\frac{X^mY^n}{m!n!}-1\right)^k
&=\sum_{k=1}^\infty
   \sum_{m_1+n_1>0}\cdots\sum_{m_k+n_k>0}\frac{(-1)^{k-1}}{k}
\frac{X^{m_1}Y^{n_1}\cdots X^{m_k}Y^{n_k}}{m_1!n_1!\cdots m_k!n_k!}
\end{align}
\begin{equation}
   \log(\ee^X\ee^Y)=\sum_{k=1}^{\infty}\sum_{m_1+n_1>0}\cdots\sum_{m_k+n_k>0}
\frac{(-1)^{k-1}}{k\sum_{i=1}^{k}(m_i+n_i)}
\frac{
\Bigl([X,\Bigr)^{m_1}
\Bigl([Y,\Bigr)^{n_1}\cdots
\Bigl([X,\Bigr)^{m_k}
\Bigl([Y,\Bigr)^{n_k}]\cdots]
}{m_1!n_1!\cdots m_k!n_k!}
\end{equation}
\hfill with $[X]$ being $X$.
\\
Derivative:
\begin{alignat}{1}
 &\frac{\dd}{\dd t}\ee^{tX} = X\ee^{t X} = \ee^{tX} X\\
 &\ee^{-X(t)}\left(\frac{\dd}{\dd t}\ee^{X(t)}\right) =
    \frac{I-\ee^{-\ad_X}}{\ad_X}\left(\frac{\dd X}{\dd t}\right)
  = X' + \frac{[-X,X']}{2!} + \frac{[-X,[-X,X']]}{3!} + \cdots
\\
 &\left(\frac{\dd}{\dd t}\ee^{X(t)}\right)\ee^{-X(t)}
= X' + \frac{[X,X']}{2!} + \frac{[X,[X,X']]}{3!} + \cdots
\end{alignat}
\hfill where $X'=\dd X/\dd t$ and $\ad_X(Y)=[X,Y]$ is the adjoint action of a Lie algebra. Thus, explicitly,
\begin{alignat}{1}
\frac{\dd}{\dd t}\ee^{aX(t)}
  = \ee^{aX}\left\{\sum_{n=0}^\infty\frac{a^{n+1}}{(n+1)!}\Bigl([-X,\Bigr)^n X']\right\}
  = \left\{\sum_{n=0}^\infty\frac{a^{n+1}}{(n+1)!}\Bigl([X,\Bigr)^n X']\right\}\ee^{aX}
\end{alignat}

\vspace{1em}

\noindent
Component:\quad If matrices $t^a$ satisfies $[t^a,t^b]=\ii f^{abc} t^c$ with totally-antisymmetric $f^{abc}\in\mathbb R$,
\begin{equation}
  \left[\ee^{\theta^at^a} t_b \ee^{-\theta^ct^c}\right]_{ij}
=  \left[\ee^{\theta^a[t^a,}t_b\right]_{ij}
= \left[\ee^{\ii \theta^af^{a}}\right]^{bc} t^c_{ij}
\end{equation}
holds for $\theta^a\in\mathbb C$, where $[f^a]_{bc}=f^{abc}$. \TODO{needs verification, generalization/restriction, and a nice proof or reference.}


\subsection{General unitary matrix}
\begin{equation}
 U_2 =
\pmat{1&0\\0&\ee^{\ii\alpha}}\pmat{\co\theta&\si\theta\\-\si\theta&\co\theta}\pmat{\ee^{\ii\beta}&0\\0&\ee^{\ii\gamma}}
=
\pmat{\phantom{-}
\co\theta\ee^{\ii\beta} & \si\theta\ee^{\ii\gamma}\\
-\si\theta\ee^{\ii(\alpha+\beta)} & \co\theta\ee^{\ii(\alpha+\gamma)}
},\qquad
0\le\theta\le\frac{\pi}2,\quad \alpha,\beta,\gamma\in\mathbb R;
\end{equation}
\begin{align}
 U_3 &=
\pmat{1&&\\&\ee^{\ii a}&\\&&\ee^{\ii b}}
 \pmat{1&&\\&\co{23}&\si{23}\\&-\si{23}&\co{23}}
 \pmat{\co{13}&&\si{13}\ee^{-\ii\delta}\\&1&\\-\si{13}\ee^{\ii\delta}&&\co{13}}
 \pmat{\co{12}&\si{12}&\\-\si{12}&\co{12}&\\&&1}
\pmat{\ee^{\ii c}&&\\&\ee^{\ii d}&\\&&\ee^{\ii e}}
\\&=
\pmat{1&&\\&\ee^{\ii a}&\\&&\ee^{\ii b}}
 \pmat{
 \co{12} \co{13} & \si{12} \co{13} & \si{13} \ee^{-\ii\delta}\\
 -\si{12} \co{23} - \co{12} \si{23} \si{13} \ee^{\ii\delta}& \co{12} \co{23} - \si{12}\si{23}\si{13} \ee^{\ii\delta}& \si{23}\co{13}\\
  \si{12}\si{23} - \co{12} \co{23} \si{13}\ee^{\ii\delta} & -\co{12}\si{23}-\si{12}\co{23}\si{13}\ee^{\ii\delta} & \co{23} \co{13}}
\pmat{\ee^{\ii c}&&\\&\ee^{\ii d}&\\&&\ee^{\ii e}}
\label{eq:GeneralUnitary33}
\end{align}
with $0\le \theta_{ij}\le\pi/2$ and $a,b,c,d,e,\delta\in\mathbb R$ (see, e.g., Ref.~\cite{Rasin:1997pn}).

\begin{minted}{wolfram}
U3 = Dot[
  DiagonalMatrix[Exp[I {0, a, b}]],
  RotationMatrix[\[Theta]23, {-1, 0, 0}],
  DiagonalMatrix[Exp[I {0,0,+\[Delta]}]],
  RotationMatrix[\[Theta]13, {0, 1, 0}],
  DiagonalMatrix[Exp[I {0,0,-\[Delta]}]],
  RotationMatrix[\[Theta]12, {0, 0, -1}],
  DiagonalMatrix[Exp[I {c, d, e}]]
]
\end{minted}


\subsection{Matrix diagonalization}\label{app:diagonalization}
In this section, $\mathbb K=\mathbb{R}$ or $\mathbb{C}$ and $\mathbb U_{\mathbb K}^{n}\subset \mathbb K^{n\times n}$ is the set of the unitary matrices.

\paragraph{Diagonalization}
A matrix $M\in\mathbb K^{n\times n}$ is called diagonalizable if $\exists P$ and $\exists D$ s.t.
\begin{equation}
 M=PDP^{-1};\qquad
 P\in\mathbb{K}^{n\times n},\quad
 D\colon\text{diagonal matrix}~(D_{ii}\in\mathbb{K}).
\end{equation}
In particular,
\begin{equation}
 \text{$M$ is normal} \stackrel{\text{def}}\iff M^\dagger M = M M^\dagger \iff
 \exists P\in\mathbb{U}_{\mathbb K}^{n} \text{~s.t.~} M=PDP^{-1}.
\end{equation}


\paragraph{Singular value decomposition}
Any $M\in\mathbb{K}^{m\times n}$ can be singular-value decomposed as
\begin{equation}
 M=UDV^\dagger;\qquad
 U\in\mathbb{U}_{\mathbb K}^m,\quad
 V\in\mathbb{U}_{\mathbb K}^n,\quad
 D\colon \text{non-negative real diagonal matrix}~(D_{ii}\ge0).
\end{equation}
Here, the matrix $U$ ($V$) diagonalizes $MM^\dagger$ ($M^\dagger M$) and $(D_{ii})^2$ are the eigenvalues of $MM^\dagger$ (and $M^\dagger M$).

The calculation on Mathematica is straightforward for this convention:
\begin{minted}{wolfram}
{u, d, v} = SingularValueDecomposition[M]
\end{minted}




\paragraph{Autonne-Takagi factorization}
If $M\in\mathbb{C}^{n\times n}$ is symmetric, it can be decomposed as
\begin{equation}
 M=RDR^\TT;\qquad
 R\in\mathbb{U}_{\mathbb C}^n,\quad
 D\colon \text{non-negative real diagonal matrix}~(D_{ii}\ge0).
 \label{eq:ATF}
\end{equation}
Real symmetric matrices are normal and thus do not need this factorization; we can apply the above ``diagonalization'' method.

Sample Mathematica code to calculate $\{D,R\}$ (with ordering, if specified) is:
\begin{minted}{wolfram}
AutonneTakagi[M_, order_ : None] := Module[{v0, v, p, ord, R, D},
  ord = If[order === None, Range[Length[M]], order];
  v0 = Eigenvectors[Conjugate[M].M];
  v = Eigenvectors[v0.M.Transpose[v0]].v0; (*resolve degenerate eigenvalues*)
  p = DiagonalMatrix[If[Abs[#] > 0, (#/Abs[#])^(-1/2), 1] & /@ Diagonal[v.M.Transpose[v]]];
  R = ConjugateTranspose[Reverse[p.v][[ord]] // Orthogonalize];
  D = ConjugateTranspose[R].M.Conjugate[R];
  {D, R}];
\end{minted}


\subsection{Group theory}\label{sec:group-theory}

\paragraph{Lie Groups}
This section is mainly based on Ref.~\cite{Hall2015}. Also, $\KK=\mathbb{R}$ or $\mathbb{C}$ and $M^n_\KK=\KK^{n\times n}$.

With $\Omega=\spmat{0&I_{n/2}\\-I_{n/2}}$ and $H=\spmat{I_p&0\\0&-I_q}$,
\begin{align}
 \mathop{\mathrm  {GL}}(n;\KK) &= \{g\in M^n_\KK\mathbin\mid \det g\neq 0\},&
 \mathop{\mathfrak{gl}}(n;\KK) &= \{a\in M^n_\KK\},\\
%
 \mathop{\mathrm  {SL}}(n;\KK) &= \{g\in M^n_\KK\mathbin\mid \det g=1\},&
 \mathop{\mathfrak{sl}}(n;\KK) &= \{a\in M^n_\KK\mathbin\mid \Tr a=0\},\\
%
 \mathop{\mathrm  {SU}}(n) &= \{g\in M^n_{\mathbb C}\mathbin\mid g^\dagger g=1\land\det g=1\},&
 \mathop{\mathfrak{su}}(n) &= \{a\in M^n_{\mathbb C}\mathbin\mid a+a^\dagger=0\land\Tr a=0\},\\
%
 \mathop{\mathrm  {SO}}(n) &= \{g\in M^n_{\mathbb R}\mathbin\mid g^\TT g=1\land\det g=1\},&
 \mathop{\mathfrak{so}}(n) &= \{a\in M^n_{\mathbb R}\mathbin\mid a+a^\TT=0\},\\
%
 \mathop{\mathrm  {SO}}(n;\mathbb C) &= \{g\in M^n_{\mathbb C}\mathbin\mid g^\TT g=1\land\det g=1\},&
 \mathop{\mathfrak{so}}(n;\mathbb C) &= \{a\in M^n_{\mathbb C}\mathbin\mid a+a^\TT=0\},\\
%
 \mathop{\mathrm  {SO}}(p,q) &= \{g\in M^{p+q}_{\mathbb R}\mathbin\mid H g^\TT H=g^{-1}\land\det g=1\},&
 \mathop{\mathfrak{so}}(p,q) &= \{a\in M^{p+q}_{\mathbb R}\mathbin\mid H a^\TT H=-a\},\\
%
 \mathop{\mathrm  {Sp}}(n;\KK) &= \{g\in M^n_\KK\mathbin\mid \Omega g^\TT \Omega=-g^{-1}\},&
 \mathop{\mathfrak{sp}}(n;\KK) &= \{g\in M^n_\KK\mathbin\mid \Omega a^\TT \Omega=a\},\\
%
 \mathop{\mathrm  {Sp}}(n) &= \{g\in M^n_{\mathbb C}\mathbin\mid \epsilon g^\TT \epsilon=-g^{-1}\land g^\dagger g=1\},&
 \mathop{\mathfrak{sp}}(n) &= \{g\in M^n_{\mathbb C}\mathbin\mid \epsilon a^\TT \epsilon=a\land a^\dagger+a=0\}.
\end{align}
Of course, $\mathop{\mathrm{U}}(n)$ and $\mathop{\mathrm{O}}(n;\KK)$ (and corresponding Lie algebra) are obtained by removing the condition $\det g=1$ (and $\Tr a=0$).


Important identities are:
\begin{equation}
 \gSp(2;\KK)=\gSL(2;\KK),\quad \gSp(2)=\gSU(2),
\end{equation}



The complexification of a finite-dimensional \emph{real} vector space $V$ is defined by
\begin{equation}
 V_{\mathbb C}=V\otimes\mathbb C=\{v_1+\ii v_2\mathbin\mid v_1,v_2\in V\}\cong V+\ii V.
\end{equation}
Then, complexification of a Lie algebra $\mathfrak a$ of a matrix Lie group is defined as follows.
Since $\mathfrak a$ is a \emph{real} Lie algebra, it can be complexified to $\mathfrak a_{\mathbb C}$, where its bracket operation has a unique extension so that $\mathfrak a_{\mathbb C}$ is a \emph{complex} lie algebra.
The complex Lie algebra $\mathfrak a_{\mathbb C}$ is called the complexification of $\mathfrak a$.
In particular,
\begin{align}\notag
 &
 \mathop{\mathfrak{gl}}(n;\mathbb R)_{\mathbb C}
 \cong \mathop{\mathfrak{u}}(n)_{\mathbb C}
 \cong \mathop{\mathfrak{gl}}(n;\mathbb C),&
 &
 \mathop{\mathfrak{so}}(n)_{\mathbb C}
 \cong \mathop{\mathfrak{so}}(n;\mathbb C),\\
 &
 \mathop{\mathfrak{sl}}(n;\mathbb R)_{\mathbb C}
 \cong \mathop{\mathfrak{su}}(n)_{\mathbb C}
 \cong \mathop{\mathfrak{sl}}(n;\mathbb C),&
 &
 \mathop{\mathfrak{sp}}(n;\mathbb R)_{\mathbb C}
 \cong \mathop{\mathfrak{sp}}(n)_{\mathbb C}
 \cong \mathop{\mathfrak{sp}}(n;\mathbb C).
\end{align}
On the other hand, a complex simple Lie algebra $\mathfrak a$ can be, on the other hand, viewed as a real Lie algebra, whose (real) dimension is twice as the original (complex) dimension.
We describe such real Lie algebra by $\mathfrak a_{\mathbb R}$.

Important identities are:
\begin{align}
 &\aSO(3,1)\cong \aSL(2;\mathbb C)_{\mathbb R},&
 &\aSO(3,1)_{\mathbb C}\cong \aSL(2;\mathbb C)\oplus\aSL(2;\mathbb C).
\end{align}


\subsection{Mathematical Foundations for Spinors}
\subsubsection{Clifford Algebra}
\paragraph{Quadratic form}
Let~\cite[\S6.3]{Jacobson1} $V$ be an $n$-dimensional vector space over a field $K$. A quadratic form $Q$ on $V$ is defined by
\begin{equation}
\begin{split}
   Q\colon V\to K,\qquad
   &\forall k\in K, v\in V, Q(k v)=k^2Q(v),\\
   &B\colon (v,w)\mapsto Q(v+w)-Q(v)-Q(w)\text{~is linear in $v$ and $w$.}
\end{split}
\end{equation}
The bilinear form $B$ satisfies $B(x,x)=2Q(x)$. As long as the characteristic of $F$ is not two, $Q$ is determined by $B$.

Equivalently, $Q$ is characterized by a symmetric ``matrix'' $\hat Q$ such that $Q(v)=v^\TT \hat Q v$.
Focusing on $K=\RR$, $\hat Q$ is a real symmetric matrix and thus diagonalizable. Sylvester's law of inertia states that there exists a basis such that $\hat Q=\diag(1,1,\cdots,1,-1,-1,\cdots,-1,0,\cdots,0)$, where the number of $+1$ and $-1$ are respectively $p$ and $q$.
The pair $(p,q)$ is unique and is called the signature of $Q$, where $\rank \hat Q=p+q~~(\le n)$.
Meanwhile, if $K$ is algebraically closed such as $\CC$, we can take $\hat Q=\diag(1,1,\cdots,1,0,\cdots,0)$ and the algebra is characterized by $\rank\hat Q$.


Quadratic form induces orthogonality: a basis $\{e_i\}$ of $V$ is called
\begin{equation}
  \text{orthogonal under $Q$}\stackrel{\text{def}}\iff B(e_i,e_j)\propto \delta_{ij}\qquad
  \Bigl[\text{orthonormal under $Q$}\stackrel{\text{def}}\iff B(e_i,e_j)=c_i\delta_{ij}\text{~with~}c_i\in\{2,0,-2\}\Bigr]
\end{equation}
and $(V,Q)$ has an orthogonal basis as long as the characteristic of $K$ is not two.
If $K$ is a ``spin field'' such as $\RR$ or $\CC$ (or, precisely, $\forall \alpha\in K,\exists \beta\in K$ s.t. $\alpha=\beta^2$ or $-\beta^2$), $(V,Q)$ has an orthonormal basis.

\paragraph{Geometric algebra}
Let~\cite{Lundholm:2009xd} $V$ be an $n$-dimensional vector space over a field $K$. Let $Q$ be a quadratic form on $V$.
Tensor algebra on $V$ is defined by
\begin{equation}
  T(V) = \bigoplus_{k} V^{\otimes k}
       = K\oplus V\oplus (V\otimes V)\oplus(V\otimes V\otimes V)\oplus\cdots.
\end{equation}
It has an ideal $K_Q$ generated by $\{v\otimes v - Q(v)\mid v\in V\}$:
\begin{equation}
  K_Q = \left\{\sum_{i} a_i\otimes(v_i\otimes v_i-Q(v_i))\otimes b_i\relmiddle| v_i\in V, a_i,b_i \in T(V)\right\},
\end{equation}
and the geometric algebra $\GeoA(V,Q)\deq T(V)/K_Q$.
The product in $\GeoA(V,Q)$ is written by $vw \deq [v\otimes w]=v\otimes w+K_q$. Particularly, $v^2 = Q(v)\in K$ and $vw+wv=B(v,w)\in K$.

The pseudoscalar $I$ is defined by $I\deq e_1e_2\cdots e_n$ with the basis $\{e_i\}$. For $\GeoA(\RR^{s,t,u})$, $I^2=(-1)^{n(n-1)/2+t}\delta_{0u}$ and $I$ commutes (anti-commutes) with all $e_i$ if $n$ is even (odd).


\paragraph{Clifford Algebra}
For a finite set $X$ and a commutative ring $R$ with unit,
the Clifford algebra $\Cl(X,R,r)$ is defined as the free $R$-module generated by the set $2^X$ of all subset of $X$.
Here, $r\colon X\to R$ is an arbitrary function thought of as a signature.
The Clifford algebra $\mathrm{Cl}$ has
addition $\mathrm{Cl}\times\mathrm{Cl}\to\mathrm{Cl}$,
scalar multiplication $R\times\mathrm{Cl}\to\mathrm{Cl}$, and
product
\begin{equation}
  \mathrm{Cl}\times\mathrm{Cl}\to\mathrm{Cl}\ni AB
  \deq \tau(A,B)A\setdiff B
\end{equation}
defined based on the following map $\tau\colon2^X\times2^X\to R$:
\begin{equation}
  \begin{split}
  &\tau(\{x\},\{x\}) = r(x)\quad\forall x\in X,&
  &\tau(A,B)\tau(A\setdiff B,C)=\tau(A,B\setdiff C)\tau(B,C)\quad\forall A,B,C\in 2^X,\\
  &\tau(\{y\},\{x\}) = -\tau(\{x\},\{y\})\quad\forall x,y\in X,x\neq y,&
  &\tau(\emptyset,A)=\tau(A,\emptyset)=1\quad\forall A\in2^X,\\
  &\tau(A,B)\in\{-1,1\}\quad \text{if}~A\cap B=\emptyset,
  \end{split}
\end{equation}
where $A\setdiff B=(A\cup B)\setminus(B\cap A)$. Namely,
\begin{equation}
  \{x\}\{x\} = r(x)\emptyset,\quad
  \{x\}\{y\}=-\{y\}\{x\},\quad
  \emptyset A=A\emptyset=A,\quad
  (AB)C=A(BC),\quad
  A\cap B=\emptyset\Rightarrow AB=\pm1(A\cup B).
\end{equation}


Several standard operations are defined: for $A\in2^X$ and $k=\binom{|A|}{2}$,
\begin{equation}
\text{Grade involution}\quad A^\star\deq(-1)^{|A|}A,\qquad
\text{Reversion}\quad A^\dagger\deq(-1)^{k}A,\qquad
\text{Clifford conjugate}\quad A^\mdlgwhtsquare\deq A^{\star\dagger}.
\end{equation}
The sign of grade involution (reversion) is negative iff $|A|$ mod 4 is 1 or 3 (2 or 3). Also, $(xy)^\star=x^\star y^\star$ and $(xy)^\dagger=y^\dagger x^\dagger$.


Let $(V,Q)$ be an $n$-dimensional vector space $V$ over a field $K$ equipped with a quadratic form $Q$. Let $E=\{e_i\}$ be an orthogonal basis of $(V,Q)$.
Then
\begin{equation}
  \Cl(E, K, Q|_E) \cong \GeoA(V,Q).
\end{equation}

The Clifford algebra $\Cl(X,R,r)$ has subspaces of $k$-vectors $\mathrm{Cl}^k$ and of even and odd vectors $\mathrm{Cl}^\pm$:
\[
\Cl(X,R,r)=\mathrm{Cl}^+\oplus\mathrm{Cl}^-=
\mathrm{Cl}^0\oplus\mathrm{Cl}^1\oplus\mathrm{Cl}^2\oplus\cdots\oplus\mathrm{Cl}^{|X|}.
\]
In particular, $\mathrm{Cl}^+$ is a subalgebra.

\paragraph{Isomorphisms}
There are several algebra-isomorphisms~\cite{Lundholm:2009xd}:
\begin{align*}
&\GeoA(\RR^{s,t})\cong \GeoA(\RR^{t+1,s-1}),\quad
\GeoA^+(\RR^{s,t})\cong \GeoA(\RR^{s,t-1})\cong \GeoA(\RR^{t,s-1})\cong\GeoA^+(\RR^{t,s}),\\
&\GeoA(\RR^{n+2,0})\cong\GeoA(\RR^{0,n})\otimes\GeoA(\RR^{2,0}),\quad
\GeoA(\RR^{0,n+2})\cong\GeoA(\RR^{n,0})\otimes\GeoA(\RR^{0,2}),\quad
\GeoA(\RR^{s+1,t+1})\cong\GeoA(\RR^{s,t})\otimes\GeoA(\RR^{1,1}),\\
&\GeoA(\RR^{1,0})\cong\RR\oplus\RR,\quad
\GeoA(\RR^{0,1})\cong\CC,\quad
\GeoA(\RR^{2,0})\cong\RR^{2\times2},\quad
\GeoA(\RR^{0,2})\cong\mathbb{H},\\
&\GeoA(\CC^{s+t})\cong\GeoA(\RR^{s,t})\otimes_\RR\CC,\quad
\GeoA(\CC^{0})\cong\CC,\quad
\GeoA(\CC^{1})\cong\CC\oplus\CC,\quad
\GeoA(\CC^{n+2})\cong\GeoA(\CC^2)\otimes_\CC\GeoA(\CC^n)\cong\GeoA(\CC^n)\otimes_\CC \CC^{2\times2},
\end{align*}
\[
  \text{$s+t$ is odd and $I^2=-1$}\then \GeoA(\RR^{s,t})\cong\GeoA^+(\RR^{t,s})\otimes \CC\cong\GeoA(\CC^{s+t-1}).
\]

\paragraph{Orthogonal Group}
The orthogonal (and related) group for $(V,Q)$ is defined by
\begin{align*}
&\gO(V,Q)\deq \{f\colon V\to V\colon\text{linear bijection s.t. $Q\circ f=Q$, i.e., $\forall v\in V, Q(f(v))=f(v)$}\},\\
&\gSO(V,Q)\deq \{f\in\gO(V,Q)\mid \det f=1\},\qquad\gSO(V,Q)\deq\text{the part of $\gSO(V,Q)$ connected to 1},
\end{align*}
where the determinant is defined with identifying $V\to V$ as a matrix.

\paragraph{Groups in Geometric Algebra}
Here we assume $K=\RR$ and $\GeoA$ is non-degenerate, having $\GeoA(\RR^{s,t})$ in mind.
The geometric algebra $\GeoA(V,Q)$ contains a group
\begin{equation}
\GeoA^\times \deq \{g\in\GeoA\mid \exists g^{-1}\in\GeoA, gg^{-1}=g^{-1}g=1\}
\end{equation}
with its addition and scalar multiplication being forgotten.
Furthermore, with $V^\times=\{v\in V\mid Q(v)\neq 0\}$, it contains
\begin{align}
  &
  \Gamma\deq\{v_1v_2\cdots v_n\mid v_i\in V^\times\},
  \qquad \tilde \Gamma\deq \{g\in\GeoA^\times\mid gVg^{-1}\in V\},
\notag\\
  &\Pin\deq\{x\in \Gamma\mid xx^\dagger = \pm1\},
 \qquad\Spin\deq \Pin\cap\GeoA^+,
 \qquad\Spin^+\deq \{x\in\Spin\mid xx^\dagger=1\}.
\end{align}
The Lipschitz group $\tilde\Gamma$ is actually equal to the versor group $\Gamma$ and is the smallest group that contains $V^\times$. To summarize,
\[\GeoA\supset\GeoA^\times\supset \Gamma=\tilde\Gamma\supset V^\times;\qquad\Gamma\supset\Pin\supset\Spin\supset\Spin^+.\]

Consider a function $R\colon \RR\to\Spin^+(V,Q)$. It is known that $R'(0)\in\GeoA^2(V,Q)$, i.e., with considering the rotor group $\Spin^+$ as a Lie group, we see the corresponding Lie algebra $\mathfrak{spin}$ is equal to $\GeoA^2$ (bivector Lie algebra), where the product of $\mathfrak{spin}$ is the commutator in $\GeoA$.
Namely,
\begin{equation}
\aSO(V,Q)\cong\mathfrak{spin}(V,Q)=\GeoA^2(V,Q),\qquad
\pm\exp\left(\GeoA^2(V,Q)\right)\subset\Spin^+(V,Q).
\end{equation}


There is an epimorphism with kernel $\RR^\times$
\begin{equation}
\tAd\colon \Gamma(V,Q)\to\gO(V,Q),\quad g\to (v\mapsto g^\star v g^{-1}),\qquad(\therefore\quad \Gamma(V,Q)/\RR^\times\cong\gO(V,Q)).
\end{equation}
Here, $v\in V$, $g\in \Gamma$, and it is easily shown that $g^\star v g^{-1}\in V$, which concludes $v\mapsto g^\star v g^{-1}$ is a linear bijection $V\to V$.

\medskip

Focusing on $\GeoA(\RR^{s,t})$, we have epimorphisms with kernel $\pm1$:
\begin{equation}
  \Pin(s,t)\to\gO(s,t),\qquad
  \Spin(s,t)\to\gSO(s,t),\qquad
  \Spin^+(s,t)\to\gSO^+(s,t),
\end{equation}
Furthermore, if $s\ge1$ and $t\ge 1$, $\Spin^+$ is of importance because, with any $e^\pm\in V$ s.t. $e^\pm e^\pm=\pm1$ and $e^+e^-+e^-e^+=0$,
\begin{equation}
\Pin(s,t)=\Spin^+(s,t)\cdot\{1,e^+,e^-,e^+e^-\},\qquad
\Spin(s,t)=\Spin^+(s,t)\cdot\{1,e^+e^-\}.
\end{equation}
It is also known that 
\begin{align}
&\text{$\Spin^+(s,t)$ is simply connected for $(s,t)=(\text{1~or~0},n)$ and $(n,\text{1~or~0})$ with $n\ge3$},\\
&
\text{if $s\le1$ or $t\le1$, }\Spin^+({s,t})=\pm\exp\left(\GeoA^2(\RR^{s,t})\right)
\end{align}
with the minus part requires only for $(0,0)$, $(1,0\text{--}3)$, and $(0\text{--}3,1)$.


\subsubsection{Representations}
\paragraph{Primer}
As given in \eqref{eq:lorentz4parts},
a Lorentz transformation is an element of $\gO(1,3)$ and the proper orthochronous part $L_0$ is isomorphic to $\gSO^+(1,3)$, which is doubly covered by $\Spin^+(1,3)=\pm\exp(\GeoA^2(\RR^{1,3}))$.
The corresponding geometric algebra $\GeoA(\RR^{1,3})$ can be embedded in $\GeoA(\CC^4)\cong\CC^{4\times 4}$ and thus has a representation by complex $4\times4$ matrices:
\begin{align*}
&L_0\cong \gSO^+(1,3)\cong\Spin^+(1,3)/\ZZ_2\subset L\cong \gO(1,3)\cong\Pin(1,3)/\ZZ_2 \subset \Gamma(\RR^{1,3})\subset \GeoA^\times(\RR^{1,3})\qquad\text{(as groups)},\\
&\GeoA(\RR^{1,3})\cong\GeoA(\RR^4)\cong\GeoA(\RR^{0,2})\otimes\GeoA(\RR^{2,0})\cong\RR^{2\times2}\otimes\mathbb{H},\qquad
\GeoA(\RR^{1,3})\otimes_\RR \CC\cong\GeoA(\CC^4)\cong\CC^{4\times4},\\
&\GeoA(\CC^4)\cong\GeoA(\RR^{1,4})\quad\text{if $(e_0e_1e_2e_3e_4)^2=-1$ }\qquad\text{(as algebras)}.
\end{align*}
The above equations are slightly modified for $(\mppp)$-metric. In particular,
$\GeoA(\RR^{3,1})\cong\GeoA(\RR^{2,0})\otimes\GeoA(\RR^{1,1})\cong\RR^{4\times 4}$ and
$\GeoA(\RR^{3,1})\otimes_\RR \CC\cong\CC^{4\times4}$.

The isomorphism $\GeoA(\RR^{1,3})\cong\RR^{2\times2}\otimes\mathbb{H}$ motivates the representation
\begin{equation}
  g^\mu\req\left\{\spmat{1&0\\0&-1},\spmat{0&\ii\\\ii&0},\spmat{0&\mathrm{j}\\\mathrm{j}&0},\spmat{0&\mathrm{k}\\\mathrm{k}&0}\right\}.
\end{equation}
Here and hereafter, $\req$ denotes that the algebraic object in the left-hand side is represented by a matrix in the right-hand side.
The basis of $\GeoA(\RR^{1,3})$ is represented by, with representing the elementary quaternions $\mathrm{i,j,k}$ by $q$,
\[
  1\req\spmat{1&0\\0&1},~~
  g^\mu\req\spmat{1&0\\0&-1},\spmat{0&q\\q&0},~~
  \tfrac12[g^0,g^i]\req\spmat{0&q\\-q&0},~~
  \tfrac12[g^{i+1},g^{i+2}]\req\spmat{q&0\\0&q},~~
  g^\mu I\req\spmat{0&-1\\-1&0},\spmat{q&0\\0&-q},~~
  I\req\spmat{0&-1\\1&0},
\]
which explicitly shows that they span $\RR^{2\times2}\otimes\mathbb{H}$.
However, this representation has little connection to the popular ones.
Although we get $4\times4$ matrices
$
  g^\mu=\left\{\spmat{1&0\\0&-1},-\ii\spmat{0&\sigma_i\\-\sigma_i&0}\right\}
$
by replacing quaternions by Pauli matrices $\mathrm{\{i,j,k\}}\mapsto-\ii\{\sigma_1,\sigma_2,\sigma_3\}$, we need to perform a basis change, $\hat\gamma^\mu=\hat R^{-1}g^\mu \hat R$ with $\hat R=\diag(1,1,\ii,\ii)$, to reach the Dirac representation, but this is not allowed within $V=\RR$.

\paragraph{Dirac and Chiral representations}
We instead use the isomorphism $\GeoA(\RR^{2,3})\cong\GeoA(\CC^4)$. By defining $g^5\deq\ii g^0g^1g^2g^3$,
\[
\{g^0,g^1,g^2,g^3,g^5\}\req \left\{\spmat{1&0\\0&-1},-\ii\spmat{0&\sigma_i\\-\sigma_i&0},\spmat{0&-\ii\\\ii&0}\right\}
\]
forms a basis of $\GeoA(\RR^{2,3})$ and thus the 32 combinations among them form a basis of $\CC^{4\times4}$.
Then, we are allowed to use the above $\hat R=\diag(1,1,\ii,\ii)$ to get the Dirac representation and the Chiral representation:
\begin{align}
&\hat\gamma^0=\hat R^{-1}g^0 \hat R=\spmat{1&0\\0&-1},\quad
\hat\gamma^i=\hat R^{-1}g^i \hat R=\spmat{0&\sigma^i\\-\sigma^i&0},\quad
\hat\gamma^5=\hat R^{-1}g^5 \hat R=\ii\hat\gamma^0\hat\gamma^1\hat\gamma^2\hat\gamma^3=\spmat{0&1\\1&0},\\
&\gamma^0=R^{-1}g^0 R=\spmat{0&1\\1&0},\quad
\gamma^i=R^{-1}g^i R=\spmat{0&\sigma^i\\-\sigma^i&0},\quad
\gamma^5=R^{-1}g^5 R=\ii\gamma^0\gamma^1\gamma^2\gamma^3=\spmat{-1&0\\0&1};\quad
R\deq\spmat{1&1\\-\ii&\ii}.
\end{align}
The pseudoscalar is $I=-\ii$ in both representations.
(Notice that, for $(\mppp)$-metric, we should use $\GeoA(\RR^{4,1})\cong\GeoA(\CC^4)$ by defining $g^5\deq g^0g^1g^2g^3$, where $I=-\ii$; this choice is not included in the following discussion.)

In Chiral or Dirac representation, the conjugate operations are given by, with $A^{\TT*}$ being  the conjugate-transpose of $A$,
\begin{equation}
  \text{Grade involution:}\quad -\gamma^2 A^* \gamma^2,\qquad
  \text{Reversion}\quad \gamma^1\gamma^3 A^\TT \gamma^3\gamma^1,\qquad
  \text{Clifford conjugate}\quad \gamma^5\gamma^0A^{\TT*}\gamma^0\gamma^5.
\end{equation}
Note that $\gamma^a$ and $\hat\gamma^a$ are all unitary (and traceless) and the 32 basis matrices are all unitary.

Let us give the explicit expressions for the spaces. The vector space $V$ is given by
\begin{equation}
V=\{v_0\gamma^0+\cdots+v_4\gamma^5\mid v_i\in\RR\},
\end{equation}
where all the elements are traceless.
The inner product and norm are given by
\begin{equation}
B(v,w)=(vw+wv)/2=(v_0w_0-v_1w_1-v_2w_2-v_3w_3+v_4w_4)E,\quad
Q(v)=vv=(v_0^2-v_1^2-v_2^2-v_3^2+v_4^2)E,
\end{equation}
where $E$ is the $4\times 4$ identity matrix. Since $v^\dagger=v$, $v^\star=-v$, and $vv^\dagger =Q(v)$, and $\det v=Q(v)^2$ for $v\in V$,
\begin{equation}
V^\times=\{v\in V \mid  \det v>0\},\quad
v^{-1}=v/Q(v).
\end{equation}
The versor group $\Gamma$ is characterized by the invertible matrices $v_1,\dots, v_n$, and thus $5n$ real numbers $v_{10},\dots, v_{n4}$:
\begin{equation}
  \Gamma(2,3)=\{v_1v_2\cdots v_n\mid v_i\in V, \det v_i>0\}.
\end{equation}
For $x=v_1\cdots v_n\in \Gamma$, $xx^\dagger = v_1\cdots v_n v_n^\dagger\cdots v_1^\dagger=\prod Q(v_i)$. Therefore,
\begin{equation}\begin{split}
  &\Pin(2,3)=\{v_1v_2\cdots v_n\mid v_i\in V, \det v_i=1\},\\
  &\Spin(2,3)=\{v_1v_2\cdots v_{2n}\mid v_i\in V, \det v_i=1\},\qquad
  \Spin^+(2,3)=\{v_1v_2\cdots v_{2n}\mid v_i\in V, Q(v_i)=1\}.
\end{split}\end{equation}
The epimorphism $\tAd$, which induces the isomorphism $\Gamma(2,3)/\RR^\times\cong\Pin(2,3)/\ZZ_2\cong \gO(2,3)$, is described as
\begin{align*}
&x\in\Gamma,\quad x\req x_1\cdots x_n,\quad x_i\in V, \\
&v\in V, \quad v \req v_0\gamma^0+\cdots+v_4\gamma^5,\quad v_a\in\RR,\qquad
\tAd_x(v)=x^\star v x^{-1}\in V
\end{align*}
and it preserves the norm: $Q(\tAd_x(v))=\tAd_x(v)\tAd_x(v)=Q(v)=vv$.
If $s\in\Pin(2,3)$,
\[
\tAd_s(v)=s^\star v s^{-1}=(-1)^ns_1\cdots s_n\,v\,s_n\cdots s_1.
\]

\paragraph{Lorentz Group} Consider $\tAd_w(v)$ with $w=w_a\gamma^a$ limited to $V$. If $w_5=0$, $\tAd_w$ does not modify the fifth component. So, we can restrict $\Gamma(2,3)$ to $\Gamma(1,3)$ and so on:
\begin{align}
  &V_4=\{v_0\gamma^0+\cdots+v_3\gamma^3\mid v_i\in\RR\};\\
  &\text{for~} x\in \{x_1x_2\cdots x_n\mid x_i\in V_4,\det x_i=1\},\qquad \tAd_x\text{~corresponds to an element in~}\gO(1,3) \quad\text{(double cover)}.
\end{align}
Note that, for $v\in V_4$,
\[
  v^{-1}=v/Q(v),\quad
  v^*=-\gamma^2 v^\star \gamma^2=\gamma^2 A\gamma^2,\quad
  v^\TT=\gamma^1\gamma^3 v\gamma^3\gamma^1,\quad
  v^{\TT*}=\gamma^0 v\gamma^0.
\]

The group $\Spin^+(1,3)$ is simply connected and isomorphic to $\pm\exp(\GeoA^2(\RR^{1,3}))$, where
\begin{equation}
  \GeoA^2(\RR^{1,3})=\mathop{\mathrm{span}}\left[
    \gamma^0\gamma^i\req\spmat{-\sigma^i&0\\0&\sigma^i},\quad
    \gamma^i\gamma^j\req\epsilon^{ijk}\spmat{-\ii\sigma^k&0\\0&-\ii\sigma^k}.
   \right],
\end{equation}
and, in this representation, the isomorphism \TODO{???} is explicit.

\paragraph{Spinor} In Chiral representation, a matrix $v=v_0\gamma^0+\cdots+v_3\gamma^3$ in $V_4$ can be expressed by
\[
 v = \psi\psi^{\TT*}\gamma^0;\quad \psi\deq
 \frac{1}{\sqrt{2(v_0-m)}}
 \left[(v_0 - m)\gamma^0 + v_1\gamma^1 + v_2\gamma^2 + v_3\gamma^3\right],\quad m\deq\sqrt{v_0^2-v_1^2-v_2^2-v_3^2}.
\]
Thus, the operation $\tAd_s(v)$ is written by
\[
\tAd_s(v)=s^\star v s^{-1}
=(-1)^ns_1\cdots s_n \psi\psi^{\TT*}\gamma^0 s_n\cdots s_1
=(-1)^n(s_1\cdots s_n \psi)(s_1\cdots s_n\psi)^{\TT*}\gamma^0.
\]
and each column of $\psi$ is understood as a Dirac spinor.
With identifying, as usual, $v_\mu$ as the momentum, we have
\[
\psi= \frac{1}{\sqrt{2\tilde E}}\pmat{
  0 & 0 & \tilde E+p_z & \rho^* \\
  0 & 0 & \rho & \tilde E-p_z \\
  \tilde E-p_z & -\rho^* & 0 & 0 \\
  -\rho & \tilde E+p_z & 0 & 0 \\
};\qquad \rho\deq p_x+\ii p_y\in\CC,\quad \tilde E\deq E-m\in\RR.
\]




\subsubsection{Fragments}
\paragraph{For Minkowski spacetime}
Also, for our interest, there are isomorphisms~\cite{RauschdeTraubenberg:2005aa,Yamaguchi:spinor}
\begin{equation}
  \GeoA(\RR^{4,1})\cong \GeoA(\RR^{1,3})\otimes \CC \cong \GeoA(\CC^4)\cong\CC^{4\times4},\qquad
  \Spin^+(1,3)\cong\gSL(2,\CC)\cong\gSp(2,\CC),\qquad
  L_0\cong\mathop{\mathrm{PSL}}(2;\CC)=\gSL(2;\CC)/\ZZ_2.
\end{equation}
Meanwhile, the Lorentz algebra $\aSO(1,3)$ is isomorphic to $\aSL(2;\mathbb C)$ viewed as a real Lie albegra~\cite[\S7.8]{Hall2015},
and its complexification
$\aSO(1,3)_{\mathbb C}$ is isomorphic to $
\aSU(2)_{\mathbb C}\oplus\aSU(2)_{\mathbb C}
=\aSL(2,\mathbb C)\oplus\aSL(2,\mathbb C)$.

\TODO{SO algebra and Grade-2 Clifford Algebra}


\end{document}
