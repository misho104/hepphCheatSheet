%%% Time-Stamp: <2019-07-28 23:49:22 misho>
\documentclass[CheatSheet]{subfiles}
\begin{document}

\summarystyle
\section[Supersymmetry with $\eta=\diag(+,-,-,-)$]{Supersymmetry with $\bm{\eta=\mathop{\mathsf{diag}}(+,-,-,-)}$}

\input{calculator/susy/formulae.txt}

\paragraph{Superfields}


\clearpage
\detailstyle

\subsection{Lorentz symmetry as SU(2)$\times$SU(2)}


\subsection{Supersymmetry algebra}
We define the generators as
\begin{alignat}{3}
 P_\mu
 &:= \ii\partial_\mu,
\quad&
 \{Q_\alpha, \bar Q_\dalpha\}
 &= -2\ii\sigma^{\mu}_{\alpha\dalpha}\partial_\mu = -2\sigma^{\mu}_{\alpha\dalpha}P_\mu,
\quad&
  \{Q_\alpha, Q_\beta\} =   \{\bar Q_\dalpha, \bar Q_\dbeta\} = 0,
\end{alignat}
which is realized by
\begin{align*}
Q_\alpha &= \frac{\partial}{\partial \theta^{\alpha}}+\ii (\sigma^{\mu}\btheta){}_{\alpha}\partial_{\mu},
&
\bar Q_\dalpha&=-\frac{\partial}{\partial \btheta^{\dalpha}}-\ii (\theta \sigma^{\mu}){}_{\dalpha}\partial_{\mu},
&
Q^\alpha&=-\frac{\partial}{\partial \theta_{\alpha}}-\ii (\btheta\bsigma^{\mu})^{\alpha}\partial_{\mu},
&
\bar Q^{\dalpha}&=\frac{\partial}{\partial \btheta_{\dalpha}}+\ii (\bsigma^{\mu}\theta)^{\dalpha}\partial_{\mu},
\\
D_\alpha&=\frac{\partial}{\partial \theta^{\alpha}}-\ii (\sigma^{\mu}\btheta){}_{\alpha}\partial_{\mu},
&
\bar D_\dalpha&=-\frac{\partial}{\partial \btheta^{\dalpha}}+\ii (\theta \sigma^{\mu}){}_{\dalpha}\partial_{\mu},
&
D^\alpha&=-\frac{\partial}{\partial \theta_{\alpha}}+\ii (\btheta\bsigma^{\mu})^{\alpha}\partial_{\mu},
&
\bar D^\dalpha&=\frac{\partial}{\partial \btheta_{\dalpha}}-\ii (\bsigma^{\mu}\theta)^{\dalpha}\partial_{\mu};
\end{align*}
$D_\alpha$ etc.~works as covariant derivatives because of the commutation relations
\begin{align*}
\{D_\alpha, \bar D_\dalpha\}&=+2\ii\sigma^{\mu}_{\alpha\dalpha}\partial_\mu,&
\{Q_\alpha, D_\beta\}&=
\{Q_\alpha, \bar D_\dbeta\}=
\{\bar Q_\dalpha, D_\beta\}=
\{\bar Q_\dalpha, \bar D_\dbeta\}=
\{D_\alpha, D_\beta\}=
\{\bar D_\dalpha, \bar D_\dbeta\}=0.
\end{align*}

\paragraph{Derivative formulae}

\begin{align*}
&
\epsilon^{\alpha\beta}\frac{\partial}{\partial\theta^\beta}=-\frac{\partial}{\partial\theta_\alpha}
&&
\frac{\partial}{\partial \theta^{\alpha}}\theta \theta =2 \theta_{\alpha}
&&
\frac{\partial}{\partial \theta^{\alpha}}\frac{\partial}{\partial \theta_{\beta}}\theta \theta =-2 \delta^{\beta}_{\alpha}
&&
\frac{\partial}{\partial \btheta^{\dalpha}}\frac{\partial}{\partial \btheta_{\dbeta}}\btheta\btheta=2 \delta^{\dbeta}_{\dalpha}
\\&
\epsilon_{\alpha\beta}\frac{\partial}{\partial\theta_\beta}=-\frac{\partial}{\partial\theta^\alpha}
&&
\frac{\partial}{\partial \theta_{\alpha}}\theta \theta =-2 \theta^{\alpha}
&&
\frac{\partial}{\partial \theta_{\alpha}}\frac{\partial}{\partial \theta_{\beta}}\theta \theta =2 \epsilon^{\alpha \beta}
&&
\frac{\partial}{\partial \btheta_{\dalpha}}\frac{\partial}{\partial \btheta_{\dbeta}}\btheta\btheta=-2 \epsilon^{\dalpha\dbeta}
\\&
\epsilon^{\dalpha\dbeta}\frac{\partial}{\partial\btheta^\dbeta}=-\frac{\partial}{\partial\btheta_\dalpha}
&&
\frac{\partial}{\partial \btheta_{\dalpha}}\btheta\btheta=2 \btheta^{\dalpha}
&&
\frac{\partial}{\partial \theta_{\alpha}}\frac{\partial}{\partial \theta^{\beta}}\theta \theta =2 \delta^{\alpha}_{\beta}
&&
\frac{\partial}{\partial \btheta_{\dalpha}}\frac{\partial}{\partial \btheta^{\dbeta}}\btheta\btheta=-2 \delta^{\dalpha}_{\dbeta}
\\&
\epsilon_{\dalpha\dbeta}\frac{\partial}{\partial\btheta_\dbeta}=-\frac{\partial}{\partial\btheta^\dalpha}
&&
\frac{\partial}{\partial \btheta^{\dalpha}}\btheta\btheta=-2 \btheta_{\dalpha}
&&
\frac{\partial}{\partial \theta^{\alpha}}\frac{\partial}{\partial \theta^{\beta}}\theta \theta =-2 \epsilon_{\alpha \beta}
&&
\frac{\partial}{\partial \btheta^{\dalpha}}\frac{\partial}{\partial \btheta^{\dbeta}}\btheta\btheta=2 \epsilon_{\dalpha\dbeta}
\end{align*}

\subsection{Superfields}
\paragraph{SUSY-invariant Lagrangian}
SUSY transformation is induced by $\xi Q + \bar\xi\bar Q = \xi^\alpha\partial_\alpha + \bar\xi_\dalpha\partial^{\dalpha} + \ii(\xi\sigma^\mu\btheta+\bar\xi\bar\sigma^\mu\theta)\partial_\mu$.
Therefore, for an object $\Psi$ in the superspace,
\begin{equation}
 \bigl[\Psi\bigr]_{\theta^4}\xrightarrow{\text{SUSY}}
 \left[\Psi + \xi^\alpha\partial_\alpha\Psi + \bar\xi_\dalpha\partial^{\dalpha}\Psi + \ii(\xi\sigma^\mu\btheta+\bar\xi\bar\sigma^\mu\theta)\partial_\mu\Psi\right]_{\theta^4}
=
 \left[\Psi +
\ii(\xi\sigma^\mu\btheta+\bar\xi\bar\sigma^\mu\theta)\partial_\mu\Psi\right]_{\theta^4},
\end{equation}
which means $\bigl[\Psi\bigr]_{\theta^4}$ is SUSY-invariant up to total derivative, i.e., $\int\dd^4x \bigl[\Psi\bigr]_{\theta^4}$ is SUSY-invariant action. Also,
\begin{equation}
  \bigl[\Psi\bigr]_{\theta^2}\xrightarrow{\text{SUSY}}
 \left[\Psi + \bar\xi_\dalpha\left(\partial^{\dalpha} + \ii(\bar\sigma^{\mu}\theta)^\dalpha\partial_\mu\right)\Psi\right]_{\theta^2}
=\left[\Psi + \bar\xi_\dalpha\bar D^\dalpha\Psi+2\ii(\bar\sigma^{\mu}\theta)^\dalpha\partial_\mu\Psi\right]_{\theta^2}
\end{equation}
will be SUSY-invariant if $\bar D_\dalpha\Psi=0$, i.e., $\Psi$ is a chiral superfield. Therefore, SUSY-invariant Lagrangian is given by
\begin{equation}
 \mathcal L = \bigl[\text{(any real superfield)}\Bigr]_{\theta^4} + \bigl[\text{(any chiral superfield)}\Bigr]_{\theta^2} + \bigl[\text{(any chiral superfield)}^*\Bigr]_{\btheta^2}.
\end{equation}
\paragraph{Chiral superfield} A chiral superfield is a superfield that satisfies $\bar D_\dalpha \Phi = 0$.
Because of $\bar D_\dalpha y^\mu=\bar D_\dalpha \theta=0$ holds for a variable $y^\mu := x^\mu - \ii(\theta\sigma^{\mu}\btheta)$, this condition is equivalent to $\partial_\dalpha \Phi(y,\theta,\btheta)=0$.
Using the Taylor expansion formula
\begin{equation}
f(y,\theta,\btheta)=f(x,\theta,\btheta)-\ii (\theta \sigma^{\mu}\btheta) \partial_{\mu}f(x,\theta,\btheta)-\frac{1}{4} \theta^4 \partial^2f(x,\theta,\btheta),
\end{equation}
we find
\begin{align}
 \Phi
 &= \phi(y) + \sqrt{2} \theta \psi(y) + \theta^2 F(y)\\
 &= \phi(x)+\sqrt{2} \theta \psi(x)-\ii \partial_{\mu}\phi(x) (\theta \sigma^{\mu}\btheta)+F(x) \theta^2+\frac{\ii}{\sqrt{2}} (\partial_{\mu}\psi(x)\sigma^{\mu}\btheta) \theta^2-\frac{1}{4} \partial^2\phi(x) \theta^4
\\
 \Phi^*
 &= \phi^*(x)+\sqrt{2} \bar{\psi}(x)\btheta+F^*(x) \btheta^2+\ii \partial_{\mu}\phi^*(x) (\theta \sigma^{\mu}\btheta)-\frac{\ii}{\sqrt{2}}[\theta \sigma^{\mu}\partial_{\mu}\bar{\psi}(x)] \btheta^2-\frac{1}{4} \partial^2\phi^*(x) \theta^4;
\end{align}
their product is expanded as
\begin{align}
\begin{split}
 \Phi_i^* \Phi_j
 &=
 \phi_i^*\phi_j + \sqrt{2} \phi_i^*(\theta \psi_j)+\sqrt{2} (\bar{\psi_i}\btheta)\phi_j
 + \phi_i^* F_j \theta^2+2 (\bar{\psi_i}\btheta)(\theta \psi_j)
 -\ii \left(\phi_i^*\partial_{\mu}\phi_j-\partial_{\mu}\phi_i^*\phi_j\right) (\theta \sigma^{\mu}\btheta)
 +F_i^*\phi_j \btheta^2
 \\&\quad
+\left[
\sqrt{2} \bar{\psi_i}\btheta F_j-\frac{\ii\left( \partial_{\mu}\phi_i^*\cdot\psi_j\sigma^{\mu}\btheta-\phi_i^*\partial_{\mu}\psi_j\sigma^{\mu}\btheta\right)}{\sqrt2}\right]\theta^2
+\left[\sqrt2 F_i^*\theta \psi_j+\frac{\ii\left(\theta \sigma^{\mu}\bar{\psi_i}\partial_{\mu}\phi_j-\theta \sigma^{\mu}\partial_{\mu}\bar{\psi_i}\phi_j\right)}{\sqrt2}\right] \btheta^2
\\&\quad
+\frac{1}{4} \left(4 F_i^*F_j-\phi_i^*\partial^2\phi_j-(\partial^2\phi_i^*)\phi_j+2 (\partial_{\mu}\phi_i^*)(\partial^{\mu}\phi_j)+2 \ii (\psi_j\sigma^{\mu}\partial_{\mu}\bar{\psi_i})-2 \ii (\partial_{\mu}\psi_j\sigma^{\mu}\bar{\psi_i})\right) \theta^4
\end{split}
\\
\begin{split}
 &\equiv
 \phi_i^*\phi_j + \sqrt{2} \phi_i^*(\theta \psi_j)+\sqrt{2} (\bar{\psi_i}\btheta)\phi_j
 + \phi_i^* F_j \theta^2+2 (\bar{\psi_i}\btheta)(\theta \psi_j)
 -2\ii \left(\phi_i^*\partial_{\mu}\phi_j\right) (\theta \sigma^{\mu}\btheta)
 +F_i^*\phi_j \btheta^2
 \\&\quad
+\sqrt2\left(
\bar{\psi_i}\btheta F_j+\ii\phi_i^*\partial_{\mu}\psi_j\sigma^{\mu}\btheta\right)\theta^2
+\sqrt2\left(F_i^*\theta \psi_j-\ii\theta \sigma^{\mu}\partial_{\mu}\bar{\psi_i}\phi_j\right) \btheta^2
\\&\quad
+\left(F_i^*F_j+(\partial_{\mu}\phi_i^*)(\partial^{\mu}\phi_j)+\ii\bar{\psi_i}\sigma^\mu\partial_{\mu}\psi_j\right) \theta^4
\end{split}
\end{align}
\begin{align}
 \Phi_i\Phi_j\Big|_{\theta^2}&=-\psi_i\psi_j+F_i\phi_j+\phi_iF_j
\\
\Phi_i\Phi_j\Phi_k\Big|_{\theta^2}&=
-(\psi_i\psi_j)\phi_k-(\psi_k\psi_i)\phi_j-(\psi_j\psi_k)\phi_i+\phi_i\phi_jF_k+\phi_k\phi_iF_j+\phi_j\phi_kF_i
\end{align}
\begin{equation}
 \begin{split}
\ee^{k\Phi} &=
\ee^{k\phi}\left[1+\sqrt{2} k \theta \psi+\Bigl(kF-\frac{k^2}{2}\psi\psi\Bigr) \theta^2-\ii k \partial_{\mu}\phi (\theta \sigma^{\mu}\btheta)
%\right.\\&\qquad\qquad\left.
+\frac{\ii k \left(\partial_{\mu}\psi+k\psi\partial_{\mu}\phi\right)\sigma^{\mu}\btheta \theta^2}{\sqrt{2}}-\frac{k}{4} \left(\partial^2\phi+k \partial_{\mu}\phi\partial^{\mu}\phi\right) \theta^4\right];
\end{split}
\end{equation}
note that $\Phi_i\Phi_j$, $\Phi_i\Phi_j\Phi_k$, and $\ee^{k\Phi}$ are all chiral superfields.
\end{document}
