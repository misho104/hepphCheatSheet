%#!latexmk CheatSheet.tex
%%% Time-Stamp: <2010-03-23 00:38:46 misho>

\section{Verbose Notes}
\subsection{Polarization Sum}\label{Sec:Verbose:PolarizationSum}
Firstly we focus on the single photon case $M=\epsilon_\mu^*(k)\epsilon_\nu'^*(k')M^{\mu\nu}$.
Here we set $k=(E,0,0,E)$, and $\epsilon=(0,1,0,0)\oplus(0,0,1,0)$. Then
\begin{equation}
  \sum\s{pol.}\left|M\right|^2
= \sum\s{pol.}\epsilon_\mu^*(k)\epsilon_\nu(k)M^{\mu}M^{\nu*}
= |M^1|^2+|M^2|^2,
\end{equation}
while
\begin{equation}
 \Hmnd M^{\mu}M^{\nu*} = |M^1|^2+|M^2|^2
\end{equation}
for Ward identity $k_\mu M^\mu=0$. Therefore the replacement
\begin{equation}
 \sum\s{pol.}\epsilon_\mu\epsilon'_\nu \to \Hmnd
\end{equation}
is valid.

Secondly we think about the double photons case
\footnote{This part is derived from �_���K��'s notebook.}
$M=\epsilon_\mu^*(k)\epsilon_\nu'^*(k')M^{\mu\nu}$.
Here we set
\begin{align}
 k &=(E,0,0, E) & \epsilon &=(0,1,0,0)\oplus(0,0,1,0)\\
 k'&=(E,0,0,-E) & \epsilon'&=(0,\cos\theta,\sin\theta,0)\oplus(0,-\sin\theta,\cos\theta,0).
\end{align}
Then doing some simple calculations, we can get
\begin{equation}
  \sum\s{pol.}\left|M\right|^2 = |M^{11}|^2+|M^{12}|^2+|M^{21}|^2+|M^{22}|^2.
\end{equation}
Nevertheless, na\"ive replacement does not work, because our Ward identities
\begin{equation}
 k_\mu\epsilon_\nu'^*(k')M^{\mu\nu} =
 \epsilon_\mu^*(k)k'_\nu M^{\mu\nu} = 0 \label{eq:WardIdentities}
\end{equation}
obviously does not help us. If we can omit $\epsilon$s from these
identities, that is if
\begin{equation}
 k_\mu M^{\mu\nu} = k'_\nu M^{\mu\nu} = 0,\label{eq:TrueIdentities}
\end{equation}
we can recover validity of the replacement:
\begin{align}
 \Hmrd\Hnsd M^{\mu\nu}M^{\rho\sigma*}
&= -\Hnsd\left(
M^{1\nu}M^{1\sigma*}+M^{2\nu}M^{2\sigma*}
\right)\\
&= |M^{11}|^2+|M^{12}|^2+|M^{21}|^2+|M^{22}|^2.
\end{align}

Then what's happening? Why this replacement is not valid?
Actually our new conditions \eqref{eq:TrueIdentities} seem to guarantee
that we are summing not only ``physical'' but also ``unphysical'' polarizations.
Meanwhile if we use some physical condition such as $\epsilon\cdot k=0$,
\eqref{eq:TrueIdentities} break down while Ward identities
\eqref{eq:WardIdentities} are still valid.

Now let's check what is happening from another viewpoint. First we suppose
$M$ satisfies our new conditions \eqref{eq:TrueIdentities}, and define
$\tilde M^{\mu\nu}$ and $\tilde M$ as
\begin{align}
 \tilde M^{\mu\nu}& := M^{\mu\nu} + k^\mu p^\nu + p'^\mu k'^\nu,\\
 \tilde M         & := \epsilon_\mu^*(k)\epsilon_\nu'^*(k')\tilde M^{\mu\nu}.
\end{align}
This alternative amplitude satisfies Ward identities (since photon is
massless and $\epsilon\cdot k=0$), and furthermore $\tilde M = M$.
Therefore $\tilde M$ is physically identical to $M$.
However technically these are very different, just because we cannot
perform our ``na\"ive replacement'' for this $\tilde M$:
\begin{align}
 \Hmrd\Hnsd \tilde M^{\mu\nu}\tilde M^{\rho\sigma*}
 &=  \Hmrd\Hnsd
\left(M^{\mu\nu} + k^\mu p^\nu + p'^\mu k'^\nu \right)
\left(M^{\rho\sigma*} + k^{\rho} p^{\sigma*} + p'^{\rho*}k'^{\sigma} \right)\\
 &= \sum\s{pol.}|M|^2 + \left[(k\cdot p'^*)(k'\cdot p) + \Hc\right].
\end{align}
After all, we have obtained following expression:
\begin{align}
 \sum\s{pol.}|\tilde M|^2 &=
 \sum\s{pol.}|M|^2\qquad \text{(Furthermore $\tilde M=M$)}\notag\\
&= \sum\s{pol.}|\epsilon_\mu^*(k)\epsilon_\nu'^*(k')M^{\mu\nu}|^2
 = \sum\s{pol.}|\epsilon_\mu^*(k)\epsilon_\nu'^*(k')\tilde M^{\mu\nu}|^2\\
&= \Hmrd\Hnsd M^{\mu\nu}M^{\rho\sigma*}\notag\\
&\ne \Hmrd\Hnsd \tilde M^{\mu\nu}\tilde M^{\rho\sigma*}
 = \sum\s{pol.}|\tilde M|^2 + \left[(k\cdot p'^*)(k'\cdot p) + \Hc\right].\notag
\end{align}

\subsection{Phantom Terms in the Gauge Theory}
\label{sec:no-other-term}
You may think we forget to introduce
$\ol\psi\G5\psi$�C
$\ol\psi\G5\slashed D\psi$�C
$\epsilon^{\mu\nu\rho\sigma}F^a_{\mu\nu}F^a_{\rho\sigma}$�C
$\epsilon^{\mu\nu\rho\sigma}D_\mu D_\nu F^a_{\rho\sigma}$
terms, but being a bit careful,
\begin{itemize}
 \item the first two terms are nonsense, for now we use $\PL$ and $\PR$,
 \item the last term is equivalent to the third term as\\[.5zw]\qquad
$\epsilon^{\mu\nu\rho\sigma}D_\mu D_\nu F^a_{\rho\sigma}
=\epsilon^{\mu\nu\rho\sigma}\dfrac12[D_\mu,D_\nu]F^a_{\rho\sigma}
=\dfrac12\epsilon^{\mu\nu\rho\sigma}F^a_{\mu\nu}F^a_{\rho\sigma}$.
\end{itemize}\vspace{.5zw}
Therefore, we have to discuss only the $\epsilon FF$ terms.
If the gauge group is simple, we can take the structure constant as totally antisymmetric, which leads these terms to fall into surface terms as:
\begin{align}
 \epsilon^{\mu\nu\rho\sigma}f^{abc}f^{ade}A^b_\mu A^c_\nu A^d_\rho A^e_\sigma
 &= \epsilon^{\mu\nu\rho\sigma}\left(-f^{acd}f^{abe}-f^{adb}f^{ace}\right)
     A^b_\mu A^c_\nu A^d_\rho A^e_\sigma\notag\\
 &= -2\epsilon^{\mu\nu\rho\sigma}f^{abc}f^{ade}A^b_\mu A^c_\nu A^d_\rho A^e_\sigma\\
 &=  0,\notag
\end{align}
\begin{align}
 \therefore\ \epsilon^{\nu\nu\rho\sigma}F_{\mu\nu}F_{\rho\sigma}
&= 4\epsilon^{\mu\nu\rho\sigma}\Pm A^a_\nu\Pr A^a_\sigma
 + 4g\epsilon^{\mu\nu\rho\sigma}f^{abc}A^a_\mu A^b_\nu\Pr A^c_\sigma\notag\\
&= 2\Pm G^\mu,&
\end{align}
where $G^\mu$ is the Chern--Simons term which is defined as
\begin{equation}
  G^\mu
:= 2\epsilon^{\mu\nu\rho\sigma}
\left( A^a_\nu\Pr A^a_\sigma+\frac13gf^{abc}A^a_\nu A^b_\rho A^c_\sigma \right)
= \epsilon^{\mu\nu\rho\sigma}
\left( A^a_\nu F^a_{\rho\sigma}-\frac13gf^{abc}A^a_\nu A^b_\rho A^c_\sigma \right).
\end{equation}
See Appendix~\ref{sec:cs-instanton} for the instanton effect.

\subsection{Instanton}
\label{sec:cs-instanton}
\TODO{Postpone...}
%%% Local Variables:
%%% TeX-master: "CheatSheet.tex"
%%% End:
