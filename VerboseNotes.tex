%#!platexmake CheatSheet
%%% Time-Stamp: <2014-01-10 22:28:55 misho>
%%% 一部で日本語が使用されています。

\section{Verbose Notes}
\subsection{Spinor Fields}
\def\redblue#1#2{({\RED{#1}},{\BLUE{#2}})}

\subsubsection{Lorentz group and Lorentz algebra}
\label{sec:verbose-lorentz-group}
\begin{tabular}{l@{ :\ \ \ }l}
Metric &  ${\RED\eta} = \diag(+1, -1, -1, -1),\qquad {\BLUE \eta} = \diag(-1,+1,+1,+1).$\\
Lorentz transf. in $\mathbb R^{1,3}$ & Linear transf. $x^\mu \mapsto \Lambda\T^\mu_\nu x^\nu$ which conserve $x^2$.\\
 & $\Longrightarrow \Hrsd = \Hmnd \Lambda\T^\mu_\rho \Lambda\T^\nu_\sigma.$
 and form a group $L$.\\
 & \qquad
   ($\Longrightarrow\quad
     (\Lambda^{-1})\T^\mu_\nu=\eta_{\nu\alpha}\eta^{\mu\beta}\Lambda\T^\alpha_\beta=:\Lambda\T_\nu^\mu\quad\Longrightarrow\quad\Lambda\T^\mu_\nu\Lambda\T_\mu^\rho=\delta^\rho_\nu$)\\[.8zw]

Disconnected parts of $L$ &
  $L_0 := \left\{ \det\Lambda = +1 \land \Lambda\T_0^0 >0\right\} \quad
   L_P := \left\{ \det\Lambda = -1 \land \Lambda\T_0^0 >0\right\}$\\
& $L_T := \left\{ \det\Lambda = +1 \land \Lambda\T_0^0 <0\right\} \quad
   L_{PT} := \left\{ \det\Lambda = -1 \land \Lambda\T_0^0 <0\right\}$\\
& \qquad ($L_0$ is identical with $\redblue{\gSO(1,3)}{\gSO(3,1)}$. )\\[1zw]
Infinitesimal one in $L_0$& $\Lambda\T^\mu_\nu = \delta^\mu_\nu + \epsilon\T^\mu_\nu$ where $\epsilon_{\mu\nu}=-\epsilon_{\nu\mu}$ (for $\eta=\eta\Lambda\Lambda$)\\[1zw]
\end{tabular}

\vspace{1zw}

微小変換は$\epsilon\T^\mu_\nu=\spmat{
0&\beta_x&\beta_y&\beta_z\\
\beta_x & 0 & -\theta_z & \theta_y\\
\beta_y & \theta_z & 0 & -\theta_x\\
\beta_y & -\theta_y & \theta_x & 0}$の形となっているので,回転生成子$\vc J$と加速生成子$\vc K$は
\begin{equation}
 \Lambda = \exp\left[\kappa\ii(\vipro\theta J +\vipro\beta K)\right]
\quad\Longrightarrow\quad
J_x = -\kappa\ii\spmat{0&0&0&0\\0&0&0&0\\0&0&0&-1\\0&0&1&0},\quad
K_x = -\kappa\ii\spmat{0&1&0&0\\1&0&0&0\\0&0&0&0\\0&0&0&0}\label{eq:explicit_JK}
\end{equation}
の形である。ここで$\kappa=\pm1$はnotationである。

一方,微小変換から生成子を
 $\epsilon\T^\mu_\nu  =: \mp\frac\ii2\epsilon^{\rho\sigma}(J_{\rho\sigma})\T^\mu_\nu$
と定義すると,計量によらずに$\epsilon_{\mu\nu}$は反対称となり,
\begin{align*}
  \vc\theta &= \redblue+-(\epsilon^{23},\epsilon^{31},\epsilon^{12}),&
  \vc\beta  &=
\redblue+-(\epsilon^{10},\epsilon^{20},\epsilon^{30}).\\
&=\redblue+-(\epsilon_{23},\epsilon_{31},\epsilon_{12})&
&=\redblue-+(\epsilon_{10},\epsilon_{20},\epsilon_{30})
\end{align*}
で,従って$J^{\rho\sigma}$も反対称。
$ (J_{\rho\sigma})\T^\mu_\nu =
\pm\ii\left(\delta^\mu_\rho\Hsnd-\delta^\mu_\sigma\Hrnd\right)$
となり,交換関係が得られ,これが閉じているのでLie代数であることもわかる:
\begin{equation}
 [J_{\mu\nu},J_{\rho\sigma}] = \mp\ii(
    \Hmrd J_{\nu\sigma} + \Hnsd J_{\mu\rho} - \Hmsd J_{\nu\rho} - \Hnrd J_{\mu\sigma}).
\end{equation}

生成子の具体形は計量に依存し,
$
 \displaystyle J\T_1_0^\mu_\nu=\redblue{\pm\ii}{\mp\ii}
 \spmat{0&1&0&0\\1&0&0&0\\0&0&0&0\\0&0&0&0}_{\mu\nu},
 \displaystyle J\T_2_3^\mu_\nu=\redblue{\pm\ii}{\mp\ii}
 \spmat{0&0&0&0\\0&0&0&0\\0&0&0&-1\\0&0&1&0}_{\mu\nu}
$となる。
よって,ここでの複号の取り方と$\kappa$および計量の定義によって,$\vc J$・$\vc K$と$J_{\rho\sigma}$の対応が定まることになる。
\starline

$\kappa$と複号について$({\RED -,上})$,$({\RED +,下})$,$({\BLUE -,下})$,$({\BLUE +,上})$を取れば
\begin{equation}
  \vc J     = (J_{23}, J_{31}, J_{12}),\quad  \vc K = (J_{10},J_{20},J_{30});\qquad
\end{equation}
となる:
\begin{equation}
 \Lambda = \exp\epsilon = \exp\left[\kappa\ii(\vipro\theta J +\vipro\beta K)\right]
         = \exp\left[\mp\ii(\epsilon^{\rho\sigma}J_{\rho\sigma})/2\right].
\end{equation}

\newpage
\subsubsection{Lorentz group and $\gSL(2,\Complex)$}
次に,連結Lie群$L_0$が,連結Lie群$\gSL(2,\Complex)/\gZ_2$と同型であることを見る:
\begin{align}
 \aSL(2,\Complex) :=& \{a\in \aGL(2,\Complex)\ |\ \Tr(a)=0\},&
 \gSL(2,\Complex) :=& \{g\in \gGL(2,\Complex) |\ \det(g)=1\}.
\end{align}
まず,$\Sm$を(極めて一般的に) $\Sm:=(\alpha1,\beta\vc\sigma)$と定義する($\alpha=\beta=\pm1$)。
$x^2 = ({\RED+}{\BLUE-})\det(x_\mu\Sm)$なので
\begin{equation}
 f^g: (x_\mu\Sm)\mapsto g (x_\mu\Sm) g^\dagger;\quad g\in\gSL(2,\Complex)
\end{equation}
は$x^2$を保存する。よってLorentz変換であり,生成子を比べることで局所同型だとわかる:
\begin{rightnote}
$\gSL(2,\Complex)\ni g = \exp(-\ii a)$として
$x_\mu (g \Sm g^\dagger) = \Lambda\T_\mu^\nu x_\nu \Sm$を微小展開すると
\begin{equation}
\Lambda\T^\mu_\nu\Sn=g^{-1}\Sm (g^{-1})^\dagger\quad\Longrightarrow\quad
\epsilon\T^\mu_\nu\Sn=\ii(a\Sm-\Sm a^\dagger)
\end{equation}
であり,ここから$g$がわかる:
\begin{equation}
 g=\exp\left(-\frac{\ii}2\vipro\theta\sigma-\frac{\alpha\beta}2\vipro\beta\sigma\right).
\end{equation}
\end{rightnote}

このことを別の観点から見る。Lorentz群の生成子の交換関係を見ると,(正しく複号を取った場合)
\begin{equation}
  [J_i, J_j]= \ii\epsilon_{ijk}J_k,\qquad
 [J_i, K_j]= \ii\epsilon_{ijk}K_k,\qquad
 [K_i, K_j]= -\ii\epsilon_{ijk}J_k
\end{equation}
となるので,
\begin{equation}
  \vc A:=\frac{1}2(\vc J+\ii\vc K), \qquad \vc B := \frac{1}2(\vc J-\ii\vc K).
\end{equation}
と定義すると
\begin{equation}
  [A_i, A_j] = \ii\epsilon_{ijk}A_k,\qquad
 [B_i, B_j] = \ii\epsilon_{ijk}B_k,\qquad
 [A_i, B_j] = 0,
\end{equation}
となり,Lorentz群が$\gSU(2)\times\gSU(2)$に分解できる。

\subsection{Weyl Spinor}\label{sec:weyl-spinor}
$\gSU(2)_A\times\gSU(2)_B$に対して$(1/2,0)$表現を為すものを左巻きspinor $\xi$,$(0,1/2)$表現を為すものを右巻きspinor $\bar\xi$と定義する。
\begin{align}
 \xi&\mapsto\left(1-\frac\ii2\vipro\theta\sigma-\frac12\vipro\beta\sigma\right)\xi&
 \bar\xi&\mapsto\left(1-\frac\ii2\vipro\theta\sigma+\frac12\vipro\beta\sigma\right)\bar\xi.
\end{align}

$\alpha\beta=1$とすると$g$は左巻きspinorの変換子となる。記号を
$\xi_\alpha\mapsto g\T_\alpha^\beta\xi_\beta$,
$\bar\xi^\dalpha\mapsto (g^\dagger)^{-1}\T^\dalpha_\dbeta\bar\xi^\dbeta$
と定義する。

次に,$\xi^\alpha\chi_\alpha$および$\bar\xi_\dalpha\bar\chi^\dalpha$がscalarとなるようにしたい。
$E=\spmat{0&1\\-1&0}$としておくと,$-E\trans{g}E=g^{-1}$より
\begin{equation}
 (\xi')^\alpha=\xi^\beta(g^{-1})\T_\beta^\alpha
              =-\xi^\beta(E_{\beta A}g\T_B^AE_{B\alpha})\qquad\therefore (-E_{\gamma\alpha})(\xi')^\alpha = -g\T_\gamma^\beta(E_{\beta\delta}\xi^\delta).
\end{equation}
よって,$\epsilon^{12}=\epsilon_{21}=1$として
$\xi^\alpha:=\epsilon^{\alpha\beta}\xi_\beta$,$\xi_\alpha=\epsilon_{\alpha\beta}\xi^\beta$
とすれば良い。


同様に,$\bar\xi'_\dalpha=-E(g^\dagger)^{-1}E\bar\xi_\dbeta$から
$(E^{\dalpha\dbeta}\bar\xi_\dbeta)'=(g^\dagger)^{-1}{}\T^\dalpha_\dbeta (E^{\dbeta\dgamma}\bar\xi_\dgamma)$となる。$\epsilon_{12}=\epsilon_{\dot1\dot2}$として$\bar\xi_\dalpha:=\epsilon_{\dalpha\dbeta}\bar\xi^\dalpha$とするのが一般的である。
また,このことから$(\xi_a)^*=\bar\xi_\dalpha$(或いは$\xi^\dagger=\bar\xi$)が分かる。

$x_\mu\Sm\mapsto x_\mu(g\Sm g^\dagger)$であるので,
$x_\mu(\xi^\alpha \Sm \bar\chi^\dalpha)$はscalarである。よって,$\Sm_{\alpha\dalpha}$のように書ける。
更にここで$x_\mu(\bar\xi_\dalpha\bSm{}^{\dalpha\alpha}\chi_\alpha)$もscalarとなるように$\bar\sigma$を定めよう。
\begin{equation}
 \bar\xi_\dalpha\bSm{}^{\dalpha\alpha}\chi_\alpha=-\epsilon_{\alpha\beta}\epsilon_{\dalpha\dbeta}\chi^\beta\bSm{}^{\dalpha\alpha}\bar\xi^\dbeta
\qquad\therefore \Sm_{\beta\dbeta}\propto \epsilon_{\alpha\beta}\epsilon_{\dalpha\dbeta}\bSm{}^{\dalpha\alpha}
\end{equation}
であり,あとはconventionである。

\begin{conclusion}{}
\begin{align*}
  \epsilon^{12}=\epsilon_{21}=\epsilon^{\dot1\dot2}=\epsilon_{\dot2\dot1}&=1,&
\xi^\alpha&:=\epsilon^{\alpha\beta}\xi_\beta,& \xi_\alpha&=\epsilon_{\alpha\beta}\xi^\beta,&
\bar\xi^\dalpha&:=\epsilon^{\dalpha\dbeta}\bar\xi_\dbeta,& \bar\xi_\dalpha&=\epsilon_{\dalpha\dbeta}\bar\xi^\dbeta
\end{align*}\vspace{-2.8zw}
\begin{align*}
  \xi_\alpha      &\mapsto g\T_\alpha^\beta\xi_\beta,
& \xi^\alpha      &\mapsto \xi^\beta (g^{-1})\T_\beta^\alpha,
& \bar\xi_\dalpha &\mapsto \bar\xi_\dbeta(g^\dagger)\T^\dbeta_\dalpha
& \bar\xi^\dalpha &\mapsto (g^\dagger)^{-1}\T^\dalpha_\dbeta\bar\xi^\dbeta
\end{align*}\vspace{-2.8zw}
\begin{align*}
  \bSm{}^{\dalpha\alpha} &:= \epsilon^{\alpha\beta}\epsilon^{\dalpha\dbeta}\Sm_{\beta\dbeta}&
 \Sm_{\alpha\dalpha}&= \epsilon_{\alpha\beta}\epsilon_{\dalpha\dbeta}\bSm{}^{\dbeta\beta}&
 \therefore \Sm:=(\alpha1,\beta\vc\sigma),\quad\bSm:=(\alpha1,-\beta\vc\sigma)
\end{align*}
\end{conclusion}


\subsection{Polarization Sum}\label{Sec:Verbose:PolarizationSum}
Firstly we focus on the single photon case $M=\epsilon_\mu^*(k)M^{\mu}$.
In this case, the replacement
\begin{equation}
 \sum\s{pol.}\epsilon_\mu\epsilon'_\nu \to \Hmnd
\end{equation}
is valid. Let us prove this validity. First we set $k=(E,0,0,E)$, and $\epsilon=(0,1,0,0)\oplus(0,0,1,0)$. Then
\begin{equation}
  \sum\s{pol.}\left|M\right|^2
= \sum\s{pol.}\epsilon_\mu^*(k)\epsilon_\nu(k)M^{\mu}M^{\nu*}
= |M^1|^2+|M^2|^2,
\end{equation}
while
\begin{equation}
 \Hmnd M^{\mu}M^{\nu*} = |M^1|^2+|M^2|^2
\end{equation}
for Ward identity $k_\mu M^\mu=0$. Now we can see the validity easily.

Next we think about the double photons case
\footnote{This part is derived from 濱口幸一's notebook.}
$M=\epsilon_\mu^*(k)\epsilon_\nu'^*(k')M^{\mu\nu}$.
Here we set
\begin{align}
 k &=(E,0,0, E) & \epsilon &=(0,1,0,0)\oplus(0,0,1,0)\\
 k'&=(E,0,0,-E) & \epsilon'&=(0,\cos\theta,\sin\theta,0)\oplus(0,-\sin\theta,\cos\theta,0).
\end{align}
Then doing some simple calculations, we can get
\begin{align}
  \sum\s{pol.}\left|M\right|^2
&= \sum\s{pol.}\epsilon_\mu^*(k)\epsilon_\nu(k)\epsilon_\rho^*(k')\epsilon_\sigma(k')
M^{\mu\rho}M^{\nu\sigma*}\\
&= |M^{11}|^2+|M^{12}|^2+|M^{21}|^2+|M^{22}|^2\\
&\stackrel{?}{=} \Hmn\Hrs M^{\mu\rho}M^{\nu\sigma*}.
\end{align}
Badly, our Ward identities
\begin{equation}
 k_\mu\epsilon_\nu'^*(k')M^{\mu\nu} =
 \epsilon_\mu^*(k)k'_\nu M^{\mu\nu} = 0 \label{eq:WardIdentities}
\end{equation}
do not help us; now we want to be given
\begin{equation}
 k_\mu M^{\mu\nu} = k'_\nu M^{\mu\nu} = 0,\label{eq:TrueIdentities}
\end{equation}
to recover validity of the replacement.

The difference between \eqref{eq:WardIdentities} and \eqref{eq:TrueIdentities} is that the former considers (and tries to sum up) all polarizations but the latter does only physical ones.
Actually, as long as we are summing up all polarizations, the replacement is still valid; these two conditions are equivalent because of cancellation of unphysical polarizations.
However, once we restrict the polarizations (for example with using a relation $\epsilon\cdot k=0$),
we can no more use \eqref{eq:TrueIdentities} and thus the replacement becomes invalid.

Now let's check what is happening from another viewpoint. First we suppose
$M$ satisfies our latter conditions \eqref{eq:TrueIdentities}, and define
$\tilde M^{\mu\nu}$ and $\tilde M$ as
\begin{align}
 \tilde M^{\mu\nu}& := M^{\mu\nu} + k^\mu p^\nu + p'^\mu k'^\nu,\\
 \tilde M         & := \epsilon_\mu^*(k)\epsilon_\nu'^*(k')\tilde M^{\mu\nu}.
\end{align}
Here $\tilde M^{\mu\nu}\neq M^{\mu\nu}$ but $\tilde M=M$; thus $\tilde M$ satisfies Ward identities
(since photon is massless and $\epsilon\cdot k=0$).
However, we cannot utilize the replacement for $\tilde M$, while it is valid for $M$.
If you did the replacement, a wrong result would come out, like
\begin{align}
 \Hmrd\Hnsd \tilde M^{\mu\nu}\tilde M^{\rho\sigma*}
 &=  \Hmrd\Hnsd
\left(M^{\mu\nu} + k^\mu p^\nu + p'^\mu k'^\nu \right)
\left(M^{\rho\sigma*} + k^{\rho} p^{\sigma*} + p'^{\rho*}k'^{\sigma} \right)\\
 &= \sum\s{pol.}|M|^2 + \left[(k\cdot p'^*)(k'\cdot p) + \Hc\right].
\end{align}
After all, we have obtained following expression:
\begin{align}
 \sum\s{pol.}|M|^2
&= \sum\s{pol.}|\epsilon_\mu^*(k)\epsilon_\nu'^*(k')M^{\mu\nu}|^2
 = \Hmrd\Hnsd M^{\mu\nu}M^{\rho\sigma*}
\notag\\
 = \sum\s{pol.}|\tilde M|^2
& = \sum\s{pol.}|\epsilon_\mu^*(k)\epsilon_\nu'^*(k')\tilde M^{\mu\nu}|^2
\ne \Hmrd\Hnsd \tilde M^{\mu\nu}\tilde M^{\rho\sigma*}
 = \sum\s{pol.}|\tilde M|^2 + \left[(k\cdot p'^*)(k'\cdot p) + \Hc\right].
\end{align}

To check the Ward identity always helps us!

\subsection{Phantom Terms in the Gauge Theory}
\label{sec:no-other-term}
You may think we forget to introduce
$\overline\psi\G5\psi$,
$\overline\psi\G5\slashed D\psi$,
$\epsilon^{\mu\nu\rho\sigma}F^a_{\mu\nu}F^a_{\rho\sigma}$,
$\epsilon^{\mu\nu\rho\sigma}D_\mu D_\nu F^a_{\rho\sigma}$
terms, but being a bit careful,
\begin{itemize}
 \item the first two terms are nonsense, for now we use $\PL$ and $\PR$,
 \item the last term is equivalent to the third term as\\[.5zw]\qquad
$\epsilon^{\mu\nu\rho\sigma}D_\mu D_\nu F^a_{\rho\sigma}
=\epsilon^{\mu\nu\rho\sigma}\dfrac12[D_\mu,D_\nu]F^a_{\rho\sigma}
=\dfrac12\epsilon^{\mu\nu\rho\sigma}F^a_{\mu\nu}F^a_{\rho\sigma}$.
\end{itemize}\vspace{.5zw}
Therefore, we have to discuss only the $\epsilon FF$ terms.
If the gauge group is simple, we can take the structure constant as totally antisymmetric, which leads these terms to fall into surface terms as:
\begin{align}
 \epsilon^{\mu\nu\rho\sigma}f^{abc}f^{ade}A^b_\mu A^c_\nu A^d_\rho A^e_\sigma
 &= \epsilon^{\mu\nu\rho\sigma}\left(-f^{acd}f^{abe}-f^{adb}f^{ace}\right)
     A^b_\mu A^c_\nu A^d_\rho A^e_\sigma\notag\\
 &= -2\epsilon^{\mu\nu\rho\sigma}f^{abc}f^{ade}A^b_\mu A^c_\nu A^d_\rho A^e_\sigma\\
 &=  0,\notag
\end{align}
\begin{align}
 \therefore\ \epsilon^{\nu\nu\rho\sigma}F_{\mu\nu}F_{\rho\sigma}
&= 4\epsilon^{\mu\nu\rho\sigma}\Pm A^a_\nu\Pr A^a_\sigma
 + 4g\epsilon^{\mu\nu\rho\sigma}f^{abc}A^a_\mu A^b_\nu\Pr A^c_\sigma\notag\\
&= 2\Pm G^\mu,&
\end{align}
where $G^\mu$ is the Chern--Simons term which is defined as
\begin{equation}
  G^\mu
:= 2\epsilon^{\mu\nu\rho\sigma}
\left( A^a_\nu\Pr A^a_\sigma+\frac13gf^{abc}A^a_\nu A^b_\rho A^c_\sigma \right)
= \epsilon^{\mu\nu\rho\sigma}
\left( A^a_\nu F^a_{\rho\sigma}-\frac13gf^{abc}A^a_\nu A^b_\rho A^c_\sigma \right).
\end{equation}
See Appendix~\ref{sec:cs-instanton} for the instanton effect.

\subsection{楊-Mills Theory}\label{sec:yang-mills-theory}

\subsubsection{General Gauge Theory}\vskip-4pt
For any Lie group $G$, we can consider ``gauge transformation'' $\phi(x) \mapsto V(x)\phi(x)$, where $V:\Real^{1,3}\to G$. Also we can define a ``connection field'' $A_\mu(x)$ as:
\begin{align}
 \phi_\parallel (x+\dd x):= \phi(x)+\ii g A_\mu(x)\phi(x)\dd x^\mu
\qquad\text{s.t.}\quad \phi_\parallel(x+\dd x)\mapsto V(x+\dd x)\phi_\parallel(x+\dd x).
\end{align}
Then the covariant derivative $\Dm$ can be defined as
\begin{equation}
 \Dm\phi(x)\dd x^\mu := \Delta_{\dd x}\phi(x):= \phi(x+\dd x)-\phi_\parallel(x+\dd x)\qquad
\therefore  \Dm := \Pm - \ii g A_\mu.
\end{equation}
Note that $\Delta_{\dd x}\phi(x)\mapsto V(x+\dd x)\Dm\phi(x)\dd x^\mu$, which means
$\Dm\phi(x)\mapsto V(x)\Dm\phi(x)$.
Now we can see
\begin{align}
 \phi&\mapsto V\phi,&
  A_\mu&\mapsto V\left(A_\mu + \frac\ii g \partial_\mu\right)V^{-1},&
 \Dm&\mapsto V\Dm V^{-1}.\label{eq:gaugetransf}
\end{align}
We can do another discussion: we can define $\Dm$ as a kind of derivative which satisfies \eqref{eq:gaugetransf}{}.

Next we introduce the curvature tensor, or ``field strength'' as
\begin{align}
& \Delta\phi(x):=\phi_{\parallel}^{xy}(x+\dd x+\dd y)-\phi_{\parallel}^{yx}(x+\dd x+\dd y)
= \left[\Dm,\Dn\right]\phi(x)\dd x^\mu\dd y^\nu
=: -\ii g F_{\mu\nu}\phi(x)\dd x^\mu\dd y^\nu;\\
&F_{\mu\nu}(x):=\frac{\ii}g\left[\Dm,\Dn\right]
 = \Pm A_\nu(x)-\Pn A_\mu(x)-\ii g\left[A_\mu(x),A_\nu(x)\right].
\end{align}
$\Delta\phi(x)$ is transformed in terms of $V(x+\dd x+\dd y)\simeq V(x)$, thus $F_{\mu\nu}(x)\mapsto V(x)F_{\mu\nu}(x)V^{-1}(x)$.



\subsubsection{Compact Gauge Theory}
\subparagraph{Generators}
If the gauge group $G$ is {\bf compact}, it has a finite-dimensional
unitary representation. The generators $T_a$ can be taken to be Hermitian, and $V(x) = \exp\left[\ii g\theta^a(x) T^a\right]$ for $\theta^a(x)\in\Real$;
\begin{align}
 [T^a,T^b] &= \ii f\T^a^b_c T^c\quad(f\in\Real)&
 0 &=f\T^D_a_b f\T^E_D_c + f\T^D_c_a f\T^E_D_b  + f\T^D_b_c f\T^E_D_a
\end{align}

For the sake of the compactness Killing form is positive-definite, where we can normalize the generators as $\Tr T^aT^b=\frac12\delta^{ab}$, and the structure constant $f^{abc}$ would be totally antisymmetric.

\subparagraph{Adjoint Representations}
\begin{equation}
 [\tilde T^a]\T_i^j:=-\ii f^{aij};\qquad
[\tilde\Dm]\T_i^j:=\delta_i^j\Pm+gf^{iaj}A^a_\mu.
\end{equation}

\subparagraph{Field Expansion}
In this {\em normalized Hermitian} basis, the relations would be\footnote{We can expand $A_\mu$ in $T^a$-basis, because it is induced by the gauge transformation.}
\begin{align*}
\phi'&=\ee^{\ii gT^a\theta^a}\phi;&
F_{\mu\nu}^a  &= \Pm A_\nu^a-\Pn A_\mu^a + gf^{abc}A_\mu^bA_\nu^c\\
A_\mu^a{}'&\simeq A_\mu^a+\Pm\theta^a+gf^{abc}A_\mu^b\theta^c&
F^a_{\mu\nu}{}'&=[\ee^{\ii g\theta^c\tilde T^c}]^{ab}F^b_{\mu\nu}\\
         &=A_\mu^a+(\tilde\Dm\theta)^a,&
&\simeq F_{\mu\nu}^a+gf^{abc}F^b_{\mu\nu}\theta^c.
\end{align*}

\subparagraph{Covariant Derivative}
For a field $\lambda^a$ under the adjoint representation,
\begin{equation}
 (\Dm\lambda)^a   = \Pm\lambda^a    + g f^{abc} A_\mu^b \lambda^c\qquad\text{or}\quad
  \Dm\lambda^aT^a = \Pm\lambda^aT^a - \ii g [A_\mu^bT^b,\lambda^aT^a]\ \footnotemark
\end{equation}
\footnotetext{Note that we can use any representation $T^a$ but must the same ones for $A^a_\mu T^a$ and $\lambda^a T^a$.}

\subparagraph{Bianchi Equation}
\begin{equation}
 \epsilon^{\mu\nu\rho\sigma}\left[\Dn,\left[\Dr,\Ds\right]\right] = \epsilon^{\mu\nu\rho\sigma}\Dn F_{\rho\sigma} = 0.
\end{equation}

\subparagraph{Conjugate representation}
Consider a representation $r$ and its generators $\{T_r^a\}$.
Then $\{-(T_r^a)^*\}$ also satisfies the group structure.
The corresponding representation is called the {\em conjugate representation} $\overline r$ of $r$:
\begin{equation}
 T_{\overline r}^a:=-(T_r^a)^*.
\end{equation}
If $\overline r$ is equivalent to $r$, i.e. there exists an unitary matrix $U$ such that
$T_{\overline r}^a=U^\dagger(T_r^a)U$, the representation $r$ is called a {\em real representation}.
This $U$ is unique up to a trivial scale factor, and either symmetric or anti-symmetric.
If $U$ is symmetric, the representation is called {\em real-positive}; otherwise {\em real-negative} or {\em pseudo-real}.

The $\boldsymbol2$ representation of $\gSU(2)$ is pseudo-real because
\begin{equation}
 T_{\overline {{\boldsymbol 2}}}^a \equiv -(\sigma^a/2)^* = \sigma^2(\sigma^a/2)\sigma^2.
\end{equation}

If a field $X$ transforms under a representation $r$, its complex conjugate does under $\overline r$:
\begin{equation}
  \delta X=\ii g\theta^a(T_r^a X)
\quad\Longrightarrow\quad
 \delta X^*
=-\ii g\theta^a(T_r^A X)^*
=\ii g\theta^a(T_{\overline r}^AX^*)
\end{equation}

\newpage







\subsection{Spinor}
\paragraph{$\eta^{\mu\nu}=(-,+,+,+)$ case}
\begin{tabular}[t]{l@{\ :\ }l}
 Grassmann Number
& $(ab)^\dagger=b^\dagger a^\dagger$ for $a,b\in\Grassmann$\\
& $\Longrightarrow$ for $a,b\in\RealGrassmann$, $ab\in\ii\RealGrassmann$\\
 $\gamma$ matrix
& $\displaystyle\{\Gm,\Gn\}=2\minkow\mu\nu\cdot\vc1$\\
& $\displaystyle\G{\mu\nu} = \frac12\left(\Gm\Gn-\Gn\Gm\right)$
  \quad etc...\\
& $\displaystyle(\ii\G0)^\dagger:=\ii\G0,\quad\G i^\dagger:=\G i$\\
 Dirac Conjugate
& $\bar\psi=\ii\psi^\dagger\G0$
\end{tabular}

\TODO{SPINOR}
\subsection{Instanton}
\label{sec:cs-instanton}
\TODO{INSTANTON}

\subsection{Note on the symmetry factor in the Feynman Rules}
\label{sec:note-symmetry-factor}
If a vertex consists of different particles, the Feynman rule for the vertex is defined as $\ii\lambda$, where $\lambda$ is the coupling constant for the vertex in the Lagrangian.
However, if a particle appears $n$-times in the coupling, we define the vertex as $(n!)\times\ii\lambda$.

This factor $n!$ corresponds to the freedom of contraction, which we encounter when contracting the fields. For example, $\bra{\phi_{k_1}\phi_{k_2}\phi_{k_3}}(\lambda\phi^3\psi)\ket{\psi_p}$ has $3!$ choice of the contraction; thus we define the vertex rule as $(3!)\ii\lambda$, and the amplitude is evaluated as $6\ii\lambda$.

For this multiplication, we have to consider a symmetry factor.
$\begC1{\bra{\phi}}\conC{(\lambda}\endC1{\phi}\begC1{{}^4}\conC{)(\lambda'}\endC1{\phi^3}\begC1{\chi}\conC{)}\endC1{\ket{\chi}}$ has $4\times3!$ freedom, but the rule is already multiplied by $4!\times3!$; thus we have to divide the amplitude by the symmetry factor ($=6$) to obtain $24\ii\lambda\lambda'$.
Similarly, in
$\begC1{\bra{\phi}}\conC{\frac12(\lambda}\endC1{\phi}\begC1{{}^4}\conC{)(\lambda}\endC1{\phi}\begC1{{}^4}\conC{)}\endC1{\ket{\phi}}+\begC2{\bra{\phi}}\conC{\frac12(\lambda}\begC3{\phi}\begC1{{}^4}\conC{)(\lambda}\endC1{\phi}\endC2{{}^4}\conC{)}\endC3{\ket{\phi}}
=
\begC1{\bra{\phi}}\conC{(\lambda}\endC1{\phi}\begC1{{}^4}\conC{)(\lambda}\endC1{\phi}\begC1{{}^4}\conC{)}\endC1{\ket{\phi}}$,
 the symmetry factor $6=(4!)^2/(4\cdot4\cdot3!)$ appears.
(Note that the $1/2$ from the Fourier expansion cancels with the freedom of operators' order.)






%%% Local Variables:
%%% TeX-master: "CheatSheet.tex"
%%% End:
