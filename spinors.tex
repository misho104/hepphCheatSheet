\documentclass[CheatSheet]{subfiles}
\begin{document}

\changefontsizes{10pt}
\section{Spinors}

\changefontsizes{9pt}
\begin{equation}
 (\overline{\psi_1}\psi_2)^* = (\psi_2)^\dagger(\overline\psi_1)^\dagger = \overline{\psi_2}\psi_1.
\end{equation}


\subsection{Convention \TODO{WIP!}}
First we prepare a vector $x^\mu$ and a symmetric matrix $\eta^{\mu\nu}$, which we call ``contravariant vector'' $x^\mu$ and the metric $\eta^{\mu\nu}$.
Then we perform a Lorentz transformation on $x^\mu$ to obtain $(x')^\mu$, with which we can define a matrix $\Lambda(\vc va,\vc \theta)^{\mu}_{\nu}$ through 
$\Lambda{}^\mu{}_\nu x^{\nu}=(x')^{\mu}$.

We then consider $\Lambda$s for infinitesimal transformations and define $\vc S$, $\vc J$, and $\vc K$ by
\begin{align}
 \Lambda^\mu{}_\nu
  \simeq \delta^\mu_\nu-\ii(\vc\theta\cdot\vc J^\mu{}_\nu+\vc\beta\cdot\vc K^\mu{}_\nu)
  \simeq \delta^\mu_\nu-\frac{\ii}{2}  \left[\Lambda^{\alpha\beta}S_{\alpha\beta}\right]^\mu{}_\nu
\end{align}

Imposing ``Lorentz condition'' (\TODO{what?}), we get the expression for $S=\ii(\delta\cdots)$ and $[J^i,J^j]=\cdots$; further, we get $\Lambda^{\mu}_{\nu}=\exp(-\ii\vc\theta\cdot\vc J -\ii \vc\xi\cdot\vc K)$, $\vc\theta=(\theta_{23},\theta_{31}, \theta_{12})$, $\vc\xi = \hat{\vc v}\tanh^{-1}\|\vc v\|=(\theta^{10},\theta^{20},\theta^{30})$; $J=(S_{23},S_{31},S_{12})$, $K=(S^{01}, S^{02}, S^{03})$....?




Lorentz transformation with a rotation $\theta$ around an axis $\hat{\vc\theta}$ and a boost $\vc v$ are given by
\begin{equation}
 \Lambda = \exp\left[-\ii(\vc\theta\cdot\vc J+\vc \beta\cdot \vc K)\right];
\qquad
\vc\theta:=\theta\hat{\vc\theta}, \quad \vc\beta := \hat{\vc v}\tanh^{-1}\|\vc v\|,
\end{equation}
\TODO{check!}


Lorentz transformation (infinitesimal):
$\Lambda=\pmat{
  0 &   & \vc{\beta}^\TT & \\
    & 0 &-\theta_z & \theta_y\\
  \vc{\beta} & \theta_z & 0 & -\theta_x \\
   & -\theta_y & \theta_x & 0
}$

$[J_{\mu\nu}]^{\alpha}{}_{\beta}=
 \ii(\delta^\alpha_\mu\eta_{\nu\beta}-\delta^\alpha_\nu\eta_{\mu\beta})
$

\vspace{2em}

Lorentz tensor $
M^{\mu_1\mu_2\cdots\mu_n}\propto
\bsigma^{\mu_1\dbeta_1\alpha_1}\cdots M_{\alpha_1\cdots\dbeta_1\cdots}
$

Especially
$V^\mu =: \frac12 \bsigma^{\mu\dbeta\alpha}V_{\alpha\dbeta}$,
$V_{\alpha\dbeta}=V^\mu\sigma_{\mu\alpha\dbeta}$; hermite $V_{\alpha\dbeta}$ $\Leftrightarrow$ $real V^\mu$.

\begin{align}
 &(V^\TT)_{\alpha\dbeta} = V_{\beta\dalpha},\text{\TODO{(correct? possibly wrong dot-positions?)}}\\
 &(V^*)_{\dalpha\beta}:= (V_{\alpha\dbeta})^*,\\
 &(V^\dagger)_{\alpha\dbeta} := (V_{\beta\dalpha})^* = (V^*)_{\dbeta\alpha}
\end{align}
\TODO{anyway not very sure about the reasoning; though my old note says like this...}


In general, metric is symmetric.

\begin{equation}
 (\Lambda^{-1})^\mu{}_\nu = \eta_{\nu\rho}\Lambda^{\rho}{}_\sigma(\eta^{-1})^{\sigma\mu}=:\Lambda_\nu{}^\mu
\end{equation}


\end{document}
