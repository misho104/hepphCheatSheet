\documentclass[CheatSheet]{subfiles}

\newcommand{\LorTr}[2]{\Lambda^{#1}{}_{#2}}
\newcommand\defeq{\mathrel{\stackrel{\text{def}}=}}
\begin{document}

\changefontsizes{10pt}
\section{Spinors}

\changefontsizes{9pt}
\begin{equation}
 (\overline{\psi_1}\psi_2)^* = (\psi_2)^\dagger(\overline\psi_1)^\dagger = \overline{\psi_2}\psi_1.
\end{equation}


\subsection{Verbose derivation}
We here derive the fermion convention under the most generic with signs $h_i=\pm1$, following
Refs.~\cite{Kugo1}.

\paragraph{Lorentz group and Lorentz tensors}
The Lorentz transformation $\LorTr\mu\nu$ is defined as a linear transformation $x^\mu\mapsto\LorTr\mu\nu x^\nu$ that conserves $x^2=\eta_{\mu\nu}x^\mu x^\nu$, where $x^\mu$ is a spacetime point and $\eta$ is the Minkowski metric:
\begin{equation}
 \eta^{\mu\nu}=\eta_{\mu\nu}\defeq h_\eta\times\diag(+1,-1,-1,-1),\qquad
 \eta^{\mu\alpha}\eta_{\alpha\nu}=\delta^{\mu}_\nu;\qquad
 \eta_{\rho\sigma}\defeq\eta_{\mu\nu}\LorTr\mu\rho\LorTr\nu\sigma ~\text{(defining equation)}.
\label{eq:LorTr}
\end{equation}
Its nice to denote its inverse, which satisfies $x_\nu\mapsto x_\mu(\Lambda^{-1})^\mu{}_\nu$, by $\Lambda_\nu{}^\mu$:
\begin{equation}
 (\Lambda^{-1})^\mu{}_\nu = \eta^{\mu\alpha}\LorTr\beta\alpha\eta_{\beta\nu}
\equiv \Lambda_\nu{}^\mu;\qquad
x_\mu\mapsto \Lambda_\mu{}^\nu x_\nu,\qquad
\delta^\alpha_\beta
%=(\Lambda^{-1})^\alpha{}_\mu\LorTr\mu\beta
%=\LorTr\alpha\mu (\Lambda^{-1})^\mu{}_\beta
=\Lambda_\mu{}^\alpha\LorTr\mu\beta
=\LorTr\alpha\mu \Lambda_\beta{}^\mu.
\end{equation}
They form a group $L\cong\gO(1,3)$ (Lorentz group), which has four disconnected parts:
\begin{align}
 L\w 0  &=\{\Lambda\mathbin|\det\Lambda=+1, \LorTr00\ge1\}\cong \mathop{\mathrm{SO^+}}(1,3),&
 L\w P  &=\{\Lambda\mathbin|\det\Lambda=-1, \LorTr00\ge1\},\notag\\
 L\w T  &=\{\Lambda\mathbin|\det\Lambda=+1, \LorTr00\le-1\},&
 L\w{PT}&=\{\Lambda\mathbin|\det\Lambda=-1, \LorTr00\le-1\}.
\end{align}

Tensors $T^{\mu_1\mu_2\cdots}_{\nu_1\nu_2\cdots}$ and pseudo-tensors $\tilde T^{\mu_1\mu_2\cdots}_{\nu_1\nu_2\cdots}$ are objects that satisfy
\begin{equation}
 T^{\mu_1\mu_2\cdots}_{\nu_1\nu_2\cdots}\mapsto
\LorTr{\mu_1}{\alpha_1}\cdots
\Lambda_{\nu_1}{}^{\beta_1}\cdots
 T^{\alpha_1\alpha_2\cdots}_{\beta_1\beta_2\cdots},
\qquad
 \tilde T^{\mu_1\mu_2\cdots}_{\nu_1\nu_2\cdots}\mapsto(\det\Lambda)
\LorTr{\mu_1}{\alpha_1}\cdots
\Lambda_{\nu_1}{}^{\beta_1}\cdots
 \tilde T^{\alpha_1\alpha_2\cdots}_{\beta_1\beta_2\cdots}.
\end{equation}
There are two constants that qualify to be (pseudo-)tensors: $\eta^{\mu\nu}$ (and $\eta_{\mu\nu}$, $\delta^{\mu}_\nu$) and the anti-symmetric tensor $\vep^{\mu\nu\rho\sigma}$:
\begin{equation}
\eta^{\mu\nu}\mapsto \LorTr\mu\alpha\LorTr\mu\beta\eta^{\alpha\beta}=\eta^{\mu\nu},\qquad
 \vep^{\mu\nu\rho\sigma}\mapsto \LorTr\mu\alpha\LorTr\nu\beta\LorTr\rho\gamma\LorTr\sigma\delta\vep^{\alpha\beta\gamma\delta}=(\det\Lambda)\vep^{\mu\nu\rho\sigma};\qquad
 \vep^{0123}\defeq1,\quad \vep_{0123}=-1.
\end{equation}

\paragraph{Infinitesimal transformation and Lorentz algebra}
Equation~\eqref{eq:LorTr} gives the representation for a proper orthochronous Lorentz transformation $\Lambda\in L_0$:
\begin{equation}
 \Lambda= 1 + \lambda + \Order(\lambda^2);\quad
\lambda=\spmat{
 0 & -\omega_x & -\omega_y & -\omega_z \\
 -\omega_x & 0        &+\theta_z &-\theta_y\\
 -\omega_y &-\theta_z & 0        &+\theta_x\\
 -\omega_z &+\theta_y &-\theta_x & 0
}
~\defeq~
\ii\vc{\omega}\cdot\vc K+\ii\vc\theta\cdot\vc J
~\defeq~
\frac{-\ii}{2}d^{\mu\nu}M_{\mu\nu}
\label{eq:LorTrInf}
\end{equation}
where $\theta_i$ is the angle of a \emph{passive} rotation around $i$-axis and $\omega_i$ describes the \emph{passive} boost along $i$-axis with velocity $\beta=\tanh\omega$.

The last definition of Eq.~\eqref{eq:LorTrInf} is used to consider Lorentz algebra $\{M\}$.
As $[M_{\rho\sigma}]{}^\mu{}_\nu$ should be a tensor, the parameter $d$ should also be a tensor and thus $d^{\mu\nu}=k\cdot\eta^{\mu\rho}\lambda{}^\nu{}_\rho$ ($k$ is a constant), i.e.,
\begin{equation}
 M_{\mu\nu}=-M_{\nu\mu}, \qquad
 d^{\mu\nu}=-d^{\nu\mu}, \qquad
 \{d^{01},d^{02},d^{03}\}=kh_\eta\times\vc\omega,\qquad
 \{d^{32},d^{13},d^{21}\}=kh_\eta\times\vc\theta.
\end{equation}
Then the element of the Lorentz algebra is given by
$ (M_{\rho\sigma}){}^\mu{}_\nu=
(\ii/k)(\delta^\mu_\rho\eta_{\nu\sigma}-\delta^\mu_\sigma \eta_{\nu\rho})
$, which gives the Lorentz algebra
\begin{equation}
[M_{\mu\nu},M_{\rho\sigma}] = 
(-\ii/k)\left(\eta_{\mu\rho}M_{\nu\sigma}-\eta_{\mu\sigma}M_{\nu\rho}
+\eta_{\nu\sigma}M_{\mu\rho}-\eta_{\nu\rho}M_{\mu\sigma}
\right).
\label{eq:LorentzAlgebra}
\end{equation}
We will take $k=+1$ to match the notation of Ref.~\cite{Kugo1}.

\paragraph{Rotation and boost}
The boost and rotation operators, $\vc K$ and $\vc J$, are now given in an abstract form by
\begin{equation}
 \vc J = h_\eta (M_{23}, M_{31}, M_{12}), \qquad
 \vc K = -h_\eta(M_{01}, M_{02}, M_{03}),
\end{equation}
so their commutation relation is read from \cref{eq:LorentzAlgebra}:
\begin{align}
 [J_i, J_j] &= \ii \epsilon_{ijk}J_k,&
 [J_i, K_j] &= \ii \epsilon_{ijk}K_k,&
 [K_i, K_j] &= -\ii \epsilon_{ijk}J_k,
\end{align}
which leads to
\begin{align}
 \vc A&\defeq\frac{\vc J+\ii \vc K}{2},&
 \vc B&\defeq\frac{\vc J-\ii \vc K}{2};&
 [A_i, A_j] &= \ii \epsilon_{ijk}A_k,&
 [B_i, B_j] &= \ii \epsilon_{ijk}B_k,&
 [A_i, B_j] &= 0.
\end{align}
This means $\aSO(1,3)$ is somewhat similar to $\aSU(2)\oplus \aSU(2)$, or in fact,
as discussed in \cref{sec:group-theory},
$\aSO(1,3)_{\mathbb C}\cong \aSU(2)_{\mathbb C}\oplus\aSU(2)_{\mathbb C}$.

\paragraph{More isomorphism}
Let us see a bit more of  mathematical structure, following the discussion in \cref{sec:group-theory}.
The isomorphic groups
$\mathop{\mathrm{Spin}}(1,3)^+\cong\gSL(2,\mathbb C)\cong\gSp(2,\mathbb C)$
is a double cover of $L_0$;
in particular, $L_0\cong \mathop{\mathrm{PSL}}(2;\mathbb C)=\gSL(2;\mathbb C)/\mathbb Z_2$.

Meanwhile, the Lorentz algebra $\aSO(1,3)$ is isomorphic to $\aSL(2;\mathbb C)$ viewed as a real Lie albegra~\cite[\S7.8]{Hall2015},
and its complexification
$\aSO(1,3)_{\mathbb C}$ is isomorphic to $
\aSU(2)_{\mathbb C}\oplus\aSU(2)_{\mathbb C}
=\aSL(2,\mathbb C)\oplus\aSL(2,\mathbb C)$.

\paragraph{Spinors from Clifford algebra}
First we construct spinors from Clifford algebra $\mathfrak C_{1,3}$ given in \cref{sec:group-theory}. A matrix representation $\gamma^\mu$ of $\mathfrak C_{1,3}$ and spinors $\psi$ are introduced:
\begin{equation}
 \overline\psi = \psi^\dagger\beta, \quad\beta\beta=1,\qquad
\{\gamma^\mu,\gamma^\nu\}=2\eta^{\mu\nu}, \quad
\gamma_5 = h_5\times\ii\gamma^0\gamma^1\gamma^2\gamma^3; \qquad
\psi\mapsto T\psi,\quad
\overline\psi\mapsto \overline\psi\beta T^\dagger \beta;\quad T\in \Gamma_{1,3}.
\end{equation}
Thus, in order that $\overline\psi\psi$ and $\overline\psi\gamma^\mu\psi$ be respectively scalar and vector, $T$ should satisfy
\begin{equation}
 T^{-1}\gamma^\mu T = \LorTr\mu\nu\gamma^\nu,\quad
 \beta T^\dagger \beta T=1,
\end{equation}
or in infinitesimal form $T=1+(-\ii/2)d^{\mu\nu}S_{\mu\nu}$,
\begin{equation}
 (S_{\mu\nu})^\dagger = \beta S_{\mu\nu} \beta,\quad
[S_{\mu\nu},\gamma^\alpha] = -(M_{\mu\nu})^{\alpha}{}_\beta\gamma^\beta;
\qquad
\therefore S_{\mu\nu}=\frac{\ii}{4}[\gamma_\mu,\gamma_\nu];
\quad
[\gamma_\mu^\dagger, \gamma_\nu^\dagger]=\beta[\gamma_\mu,\gamma_\nu]\beta;
\end{equation}
finally, a representation of the Lorentz group is constructed as
\begin{equation}
\Lambda\stackrel{\text{rep}}=
 \exp\left(\frac{-\ii}{2}d^{\mu\nu}S_{\mu\nu}\right);\qquad
 [S_{\mu\nu},S_{\rho\sigma}] = -\ii\left(
\eta_{\mu\rho} S_{\nu\sigma}
-\eta_{\nu\rho} S_{\mu\sigma}
+\eta_{\mu\sigma} S_{\nu\rho}
-\eta_{\nu\sigma} S_{\mu\rho}\right),
\end{equation}
which is exactly the same as \cref{eq:LorentzAlgebra}.






\subsection[Convention]{Convention \TODO{WIP!}}
First we prepare a vector $x^\mu$ and a symmetric matrix $\eta^{\mu\nu}$, which we call ``contravariant vector'' $x^\mu$ and the metric $\eta^{\mu\nu}$.
Then we perform a Lorentz transformation on $x^\mu$ to obtain $(x')^\mu$, with which we can define a matrix $\Lambda(\vc va,\vc \theta)^{\mu}_{\nu}$ through 
$\Lambda{}^\mu{}_\nu x^{\nu}=(x')^{\mu}$.

We then consider $\Lambda$s for infinitesimal transformations and define $\vc S$, $\vc J$, and $\vc K$ by
\begin{align}
 \Lambda^\mu{}_\nu
  \simeq \delta^\mu_\nu-\ii(\vc\theta\cdot\vc J^\mu{}_\nu+\vc\beta\cdot\vc K^\mu{}_\nu)
  \simeq \delta^\mu_\nu-\frac{\ii}{2}  \left[\Lambda^{\alpha\beta}S_{\alpha\beta}\right]^\mu{}_\nu
\end{align}

Imposing ``Lorentz condition'' (\TODO{what?}), we get the expression for $S=\ii(\delta\cdots)$ and $[J^i,J^j]=\cdots$; further, we get $\Lambda^{\mu}_{\nu}=\exp(-\ii\vc\theta\cdot\vc J -\ii \vc\xi\cdot\vc K)$, $\vc\theta=(\theta_{23},\theta_{31}, \theta_{12})$, $\vc\xi = \hat{\vc v}\tanh^{-1}\|\vc v\|=(\theta^{10},\theta^{20},\theta^{30})$; $J=(S_{23},S_{31},S_{12})$, $K=(S^{01}, S^{02}, S^{03})$....?




Lorentz transformation with a rotation $\theta$ around an axis $\hat{\vc\theta}$ and a boost $\vc v$ are given by
\begin{equation}
 \Lambda = \exp\left[-\ii(\vc\theta\cdot\vc J+\vc \beta\cdot \vc K)\right];
\qquad
\vc\theta:=\theta\hat{\vc\theta}, \quad \vc\beta := \hat{\vc v}\tanh^{-1}\|\vc v\|,
\end{equation}
\TODO{check!}


Lorentz transformation (infinitesimal):
$\Lambda=\pmat{
  0 &   & \vc{\beta}^\TT & \\
    & 0 &-\theta_z & \theta_y\\
  \vc{\beta} & \theta_z & 0 & -\theta_x \\
   & -\theta_y & \theta_x & 0
}$

$[J_{\mu\nu}]^{\alpha}{}_{\beta}=
 \ii(\delta^\alpha_\mu\eta_{\nu\beta}-\delta^\alpha_\nu\eta_{\mu\beta})
$

\vspace{2em}

Lorentz tensor $
M^{\mu_1\mu_2\cdots\mu_n}\propto
\bsigma^{\mu_1\dbeta_1\alpha_1}\cdots M_{\alpha_1\cdots\dbeta_1\cdots}
$

Especially
$V^\mu =: \frac12 \bsigma^{\mu\dbeta\alpha}V_{\alpha\dbeta}$,
$V_{\alpha\dbeta}=V^\mu\sigma_{\mu\alpha\dbeta}$; hermite $V_{\alpha\dbeta}$ $\Leftrightarrow$ $real V^\mu$.

\begin{align}
 &(V^\TT)_{\alpha\dbeta} = V_{\beta\dalpha},\text{\TODO{(correct? possibly wrong dot-positions?)}}\\
 &(V^*)_{\dalpha\beta}:= (V_{\alpha\dbeta})^*,\\
 &(V^\dagger)_{\alpha\dbeta} := (V_{\beta\dalpha})^* = (V^*)_{\dbeta\alpha}
\end{align}
\TODO{anyway not very sure about the reasoning; though my old note says like this...}


In general, metric is symmetric.

\begin{equation}
 (\Lambda^{-1})^\mu{}_\nu = \eta_{\nu\rho}\Lambda^{\rho}{}_\sigma(\eta^{-1})^{\sigma\mu}=:\Lambda_\nu{}^\mu
\end{equation}


\end{document}
