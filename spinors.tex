%!TEX program = lualatex
\documentclass[CheatSheet]{subfiles}

\begin{document}

\summarystyle
\section{Spinors}
\begin{enumerate}[label={}]
  \item Gamma matrices: \quad
$\{\gamma^\mu,\gamma^\nu\}=2\eta^{\mu\nu}$,\quad
$\gamma_5=\ii\gamma^0\gamma^1\gamma^2\gamma^3$;\quad
$\{\gamma^\mu,\gamma_5\}=0$,~
$\gamma^5\gamma^5=1$.
 \item Conjugates: $\overline{\psi} \deq \psi^\dagger \beta,$\qquad $\psi^\cc\deq C(\overline\psi)^\TT$
\end{enumerate}

\paragraph{Chiral notation}
\begin{align}\label{eq:chiralnotation}
& \overline{\psi}=\psi^\dagger\gamma^0;~
 \gamma^\mu=\pmat{0&\sigma^\mu\\\bar\sigma^\mu&0},~
 \gamma_5=\pmat{-1&0\\0&1};~
 \PL = \frac{1-\gamma_5}{2},~
 \PR = \frac{1+\gamma_5}{2};\qquad C=-\ii\gamma^2\gamma^0.
\\&
(\gamma^\mu)^\dagger = \gamma^0\gamma^\mu\gamma^0,~
(\gamma^\mu)^*       = \gamma^2\gamma^\mu\gamma^2,~
(\gamma^\mu)^\TT     = \gamma^0\gamma^2\gamma^\mu\gamma^2\gamma^0;
\\&
C\gamma^\mu C^\dagger = -(\gamma^\mu)^\dagger,~
C=C^*=-C^{-1}=-C^\dagger=-C^\TT
\end{align}
 \begin{align}
  \psi&=\pmat{\psi\w L\\[.5em]\psi\w R}=\pmat{\xi_\alpha\\[.5em]\bar\chi^\dalpha}
       =\spmat{\xi_1\\\xi_2\\\bar\chi^{\dot1}\\\bar\chi^{\dot2}}&
  &\buildrel{C}\over{\longrightarrow}&
  \psi^\cc &=
  C(\overline\psi)^\TT=
  -\ii\gamma^2\psi^*=\spmat{-\!\!&(\bar\chi^{\dot2})^*\\&(\bar\chi^{\dot1})^*\\&(\xi_2)^*\\-\!\!&(\xi_1)^*}
               =\spmat{-\!\!&\chi^2 \\ &\chi^1\\&\bar\xi_{\dot2}\\ -\!\!&\bar\xi_{\dot1}}
               =\pmat{\chi_\alpha \\[.5em] \bar\xi^\dalpha}\\
  \overline\psi&=\pmat{\chi^\alpha&\bar\xi_\dalpha}
          =\spmat{\chi^1&\chi^2&\bar\xi_{\dot 1}&\bar\xi_{\dot 2}}&
  &\buildrel{C}\over{\longrightarrow}&
  \overline{\psi^\cc} &=\psi^\TT C=\ii\psi^\TT\gamma^0\gamma^2
                =\spmat{\xi_2&-\xi_1&-\bar\chi^{\dot2}&\bar\chi^{\dot1}}
                =\pmat{\xi^\alpha&\bar\chi_{\dalpha}}
  \end{align}
  It is instructive to write, e.g., for $e=e^-$ and $e^\cc=e^+$,
  \begin{align*}
  &\psi_e=\pmat{e\w L\\e\w R}=\pmat{e_\alpha\\(\bar{e}^\cc)^\dalpha},
  &&\psi_{e^\cc}=\pmat{e^\cc_\alpha\\(\bar{e})^\dalpha}=\pmat{e^\cc_\alpha\\(\bar{e}^{\alpha})^*},
  &&\overline\psi_e=\pmat{e^{\cc\alpha}&\bar e_\dalpha},
  &&\overline\psi_{e^\cc}=\pmat{e^{\alpha}&\bar e^\cc_\dalpha},
  \end{align*}
  where $\bar e^\cc$ should be read as ``bar of $e^\cc$'', namely $(\bar{e}^\cc)^\dalpha=((e^\cc)^{\alpha})^*$ and $(\bar e)^\dalpha=(e^\alpha)^*$. Then
     \begin{align}
   &A^\alpha B_\alpha = \overline\psi_{A^\cc}\PL\psi_{B} = \overline\psi_{B^\cc}\PL\psi_{A},\qquad
   \bar A_\dalpha\bar B^\dalpha = \overline\psi_A\PR\psi_{B^\cc} = \overline\psi_B\PR\psi_{A^\cc},\\
    &\bar A\bsigma B=\overline\psi_{A}\gamma^\mu\PL\psi_{B}
    =-B\sigma^{\mu}\bar A=-\overline\psi_{B^\cc}\gamma^\mu\PR\psi_{A^\cc}
    =(\bar B\bsigma A)^*=(\overline\psi_{B}\gamma^\mu\PL\psi_{A})^*
    =-(\overline\psi_{A^\cc}\gamma^\mu\PR\psi_{B^\cc})^*,\\
 &(\overline{\psi_A}\psi_B)^* = (\psi_B)^\dagger(\overline\psi_A)^\dagger = \overline{\psi_B}\psi_A
 =A^\alpha B_\alpha+\bar{A}^\cc_\dalpha\bar B^{\cc\dalpha},\\
    &\overline \psi_A\gamma^\mu \psi_B
    = \overline\psi_A\gamma^\mu\PL \psi_B + \overline \psi_A\gamma^\mu\PR \psi_B
    = \overline\psi_A\gamma^\mu\PL \psi_B - \overline \psi_{B^\cc}\gamma^\mu\PL \psi_{A^\cc}.
   \end{align}
 \begin{RemarkBox}Dirac notation is given by
\begin{align}\label{eq:diracnotation}
& %\overline{\psi}=\psi^\dagger\hat\gamma^0;~
 \hat\gamma^0=\pmat{1&0\\0&-1},~
 \hat\gamma^i=\pmat{0&\sigma^i\\\bar\sigma^i&0},~
 \hat\gamma_5=\pmat{0&1\\1&0};~
 \hat\PL = \frac{1-\hat\gamma_5}{2},~
 \hat\PR = \frac{1+\hat\gamma_5}{2}.
\\&
(\hat\gamma^\mu)^\dagger = \hat\gamma^0\hat\gamma^\mu\hat\gamma^0,~
(\hat\gamma^\mu)^*       = \hat\gamma^2\hat\gamma^\mu\hat\gamma^2,~
(\hat\gamma^\mu)^\TT     = \hat\gamma^0\hat\gamma^2\hat\gamma^\mu\hat\gamma^2\hat\gamma^0,
\end{align}
\end{RemarkBox}


Defining
\begin{align}
 &S^i = \{1\},~~
&&V^i = \{\gamma^\mu\},~~
&&T^i = \{\sigma^{\mu\nu}\},~~
&&A^i = \{\ii\gamma^\mu\gamma_5\},~~
&&P^i = \{\gamma_5\},\\
%
 &S_i = \{1\},~~
&&V_i = \{\gamma_\mu\},~~
&&T_i = \{\sigma_{\mu\nu}\},~~
&&A_i = \{\ii\gamma_\mu\gamma_5\},~~
&&P_i = \{\gamma_5\},\\
%
\end{align}



\paragraph{Fierz Transformation}
\begin{align}
  &(AD)(BC)=-\frac12(AB)(CD)-\frac18(A\sigma^{\mu\nu}B)(C\sigma_{\mu\nu}D)
  \\
  &(\bar A\bar D)(\bar B\bar C)=-\frac12(\bar A\bar B)(\bar C\bar D)-\frac18(\bar A\bar \sigma^{\mu\nu}\bar B)(\bar C\bar \sigma_{\mu\nu}\bar D)
  \\
  &(AD)(\bar B\bar C)=-\frac12(A\sigma^\mu\bar B)(\bar C\bar\sigma_\mu D)
\end{align}

\newpage
\detailstyle

\subsection{Verbose derivation}
We provide a discussion on the fermion convention, introducing various signs $h_i=\pm1$ in order to keep generality as much as possible. We follow Ref.~\cite{Kugo1}.

\paragraph{Lorentz group and Lorentz tensors}
The Lorentz transformation $\LorTr\mu\nu$ is defined as a linear transformation $x^\mu\mapsto\LorTr\mu\nu x^\nu$ conserving $x^2=\eta_{\mu\nu}x^\mu x^\nu$, where $x^\mu$ is a spacetime point and $\eta$ is the Minkowski metric:
\begin{equation}
 \eta^{\mu\nu}=\eta_{\mu\nu}\eqdef h_\eta\times\diag(+1,-1,-1,-1),\qquad
 \eta^{\mu\alpha}\eta_{\alpha\nu}=\delta^{\mu}_\nu;\qquad
 \eta_{\rho\sigma}\eqdef\eta_{\mu\nu}\LorTr\mu\rho\LorTr\nu\sigma ~\text{(defining equation)}.
\label{eq:LorTr}
\end{equation}
Consequently, $
  \delta^\beta_\sigma
  = (\eta_{\alpha\mu}\eta^{\beta\nu}\LorTr\mu\nu)\LorTr\alpha\sigma
  \eqcolon (\Lambda^{-1})^\beta{}_\alpha\LorTr\alpha\sigma$
should define the inverse of $\Lambda$:
\begin{equation}
  \Lambda_\beta{}^\alpha
  \deq (\Lambda^{-1})^\beta{}_\alpha
  = \eta_{\alpha\mu}\eta^{\beta\nu}\LorTr\mu\nu;
  \qquad
  x_\nu
  \mapsto x_\mu(\Lambda^{-1})^\mu{}_\nu
  =\Lambda_\mu{}^\nu x_\nu,\qquad
  \delta^\alpha_\beta
  %=(\Lambda^{-1})^\alpha{}_\mu\LorTr\mu\beta
  %=\LorTr\alpha\mu (\Lambda^{-1})^\mu{}_\beta
  =\Lambda_\mu{}^\alpha\LorTr\mu\beta
  =\LorTr\alpha\mu \Lambda_\beta{}^\mu,
\end{equation}
They form a group $L\cong\gO(1,3)$ (Lorentz group), which has four disconnected parts:
\begin{align}
 L\w 0  &=\{\Lambda\mathbin|\det\Lambda=+1, \LorTr00\ge1\}\cong \mathop{\mathrm{SO^+}}(1,3),&
 L\w P  &=\{\Lambda\mathbin|\det\Lambda=-1, \LorTr00\ge1\},\notag\\
 L\w T  &=\{\Lambda\mathbin|\det\Lambda=+1, \LorTr00\le-1\},&
 L\w{PT}&=\{\Lambda\mathbin|\det\Lambda=-1, \LorTr00\le-1\}.
\label{eq:lorentz4parts}
\end{align}

Tensors $T^{\mu_1\mu_2\cdots}_{\nu_1\nu_2\cdots}$ and pseudo-tensors $\tilde T^{\mu_1\mu_2\cdots}_{\nu_1\nu_2\cdots}$ are objects that satisfy
\begin{equation}
 T^{\mu_1\mu_2\cdots}_{\nu_1\nu_2\cdots}\mapsto
\LorTr{\mu_1}{\alpha_1}\cdots
\Lambda_{\nu_1}{}^{\beta_1}\cdots
 T^{\alpha_1\alpha_2\cdots}_{\beta_1\beta_2\cdots},
\qquad
 \tilde T^{\mu_1\mu_2\cdots}_{\nu_1\nu_2\cdots}\mapsto(\det\Lambda)
\LorTr{\mu_1}{\alpha_1}\cdots
\Lambda_{\nu_1}{}^{\beta_1}\cdots
 \tilde T^{\alpha_1\alpha_2\cdots}_{\beta_1\beta_2\cdots}.
\end{equation}
In particular, metrics and the antisymmetric tensors are constant (pseudo-)tensors.
\begin{align}
&\eta^{\mu\nu}\mapsto \LorTr\mu\alpha\LorTr\mu\beta\eta^{\alpha\beta}=\eta^{\mu\nu},\qquad
 \vep^{\mu\nu\rho\sigma}\mapsto \LorTr\mu\alpha\LorTr\nu\beta\LorTr\rho\gamma\LorTr\sigma\delta\vep^{\alpha\beta\gamma\delta}=(\det\Lambda)\vep^{\mu\nu\rho\sigma};\\
&\vep^{0123}\deq1,\quad \vep_{0123}=-1.
\end{align}

\paragraph{Infinitesimal transformation} In order to consider infinitesimal transformation of a proper orthochronous Lorentz transformation $\Lambda\in L_0$, we identify $\Lambda$ as a matrix:
\[ \LorTr\mu\nu \circeq \tilde\Lambda\stackrel{\text{def}}\Longleftrightarrow\LorTr\mu\nu=\tilde\Lambda_{\mu+1,\nu+1};\qquad
\tilde\eta\deq h_\eta\diag(+1,-1,-1,-1).\]
Then, Eq.~\eqref{eq:LorTr} is read by $\tilde\eta=\tilde\Lambda^\TT\tilde\eta\tilde\Lambda$ and the infinitesimal transformation given by $\tilde\Lambda=1+\tilde\lambda+\Order(\lambda^2)$ must satisfy
\begin{equation}
\lambda^\mu{}_\nu
= -\eta_{\nu\alpha}\eta^{\mu\beta} \lambda^\alpha{}_\beta\Longleftrightarrow
\tilde\lambda^\TT = -\tilde\eta\tilde\lambda\tilde\eta.
\qquad\therefore\quad
  \tilde\lambda=\spmat{
    0 & -\omega_x & -\omega_y & -\omega_z \\
    -\omega_x & 0        &+\theta_z &-\theta_y\\
    -\omega_y &-\theta_z & 0        &+\theta_x\\
    -\omega_z &+\theta_y &-\theta_x & 0
  }\eqcolon\ii\vc{\omega}\cdot\vc K+\ii\vc\theta\cdot\vc J,
\end{equation}
where we can interpret $\theta_i$ as the \Emph{passive} rotation angle about $i$-axis and $\omega_i$ as the \Emph{passive} boost along $i$-axis with velocity $\beta=\tanh\omega$.
The operators $J_i$ and $K_i$ are defined to be pure-imaginary; $J_i$ is Hermitian, $K_i$ is pseudo-Hermitian, and
\begin{equation}
  [J_i, J_j] = \ii \epsilon_{ijk}J_k,\qquad
  [J_i, K_j] = \ii \epsilon_{ijk}K_k,\qquad
  [K_i, K_j] = -\ii \epsilon_{ijk}J_k.
\end{equation}
Noticing that $\tilde\eta\tilde\lambda$ is anti-symmetric, we can express the infinitesimal parameters by anti-symmetric coefficients
\[
d^{\mu\nu} \deq h_dh_\eta\lambda^\mu{}_\alpha\eta^{\alpha\nu};\qquad d^{\mu\nu}=-d^{\nu\mu},\qquad
\{d^{01},d^{02},d^{03}\}=h_d\vc\omega,\qquad
\{d^{32},d^{13},d^{21}\}=h_d\vc\theta.
\]
The corresponding generators are then given by anti-symmetric matrices
\[
 \{\tilde M_{01},\tilde M_{02},\tilde M_{03}\}=-h_M \vc K,\qquad \{\tilde M_{32},\tilde M_{13},\tilde M_{21}\}=-h_M \vc J;\qquad
 \tilde M\circeq (M_{\rho\sigma})^\mu{}_\nu\deq h_M h_\eta\cdot\ii(\delta^\mu_\rho\eta_{\nu\sigma}-\delta^\mu_\sigma\eta_{\nu\rho}).
\]
The infinitesimal transformation is given by
\begin{equation}
\tilde\lambda = -h_Mh_d\frac{\ii}{2} d^{\mu\nu}\tilde M_{\mu\nu}.
\label{eq:infinitesimal}
\end{equation}
The Lorentz algebra is given by
\begin{equation}
h_Mh_\eta[M_{\mu\nu},M_{\rho\sigma}] = 
-\ii\left(\eta_{\mu\rho}M_{\nu\sigma}-\eta_{\mu\sigma}M_{\nu\rho}
+\eta_{\nu\sigma}M_{\mu\rho}-\eta_{\nu\rho}M_{\mu\sigma}
\right).
\label{eq:LorentzAlgebra}
\end{equation}

In literature, the sign $h_\eta$ is first determined; $h_M$ is fixed by the Lorentz algebra \eqref{eq:LorentzAlgebra} and then $h_d$ is set by \eqref{eq:infinitesimal}.

%The signs in Ref.~\cite{Kugo1} are $h_d=h_\eta=1$ and $h_M=1$.

\paragraph{Isomorphism of Lorentz Algebra}
The commutation relations of $\vc J$ and $\vc K$ lead to
\begin{align}
 \vc A&\eqdef\frac{\vc J+\ii \vc K}{2},&
 \vc B&\eqdef\frac{\vc J-\ii \vc K}{2};&
 [A_i, A_j] &= \ii \epsilon_{ijk}A_k,&
 [B_i, B_j] &= \ii \epsilon_{ijk}B_k,&
 [A_i, B_j] &= 0.
\end{align}
This means $\aSO(1,3)$ is somewhat similar to $\aSU(2)\oplus \aSU(2)$, or in fact,
as discussed in \cref{sec:group-theory},
$\aSO(1,3)_{\mathbb C}\cong \aSU(2)_{\mathbb C}\oplus\aSU(2)_{\mathbb C}$.
Explicitly,
\[
\tilde M^{\pm}_{\alpha\beta} =
\frac{\ii}{2}\left(\pm M_{\alpha\beta} + \frac{-1}{2}\eta_{\alpha\mu}\eta_{\beta\nu}\ii\epsilon^{\mu\nu\rho\sigma}M_{\rho\sigma}\right),\qquad
\{\tilde M^+_{01},\tilde M^+_{02},\tilde M^+_{03}\}=-h_M\vc A,\qquad
\{\tilde M^-_{01},\tilde M^-_{02},\tilde M^-_{03}\}=-h_M\vc B.
\]

\TODO{reviewed upto here}


\paragraph{Representation of Clifford algebra}
To construct an irreducible representation of $\mathfrak C_{1,3}$, we utilize the fact that we can form two sets of creation-annihilation operators
\begin{align}
 a^\pm = \sqrt{h_\eta}\frac{e^0\pm e^3}{2},\quad
 b^\pm = \sqrt{h_\eta}\frac{\pm e^2-\ii e^1}{2};\qquad
 \{a^+,a^-\}=1,~
 \{b^+,b^-\}=1,~
 \{\text{(others)}\} = 0.
\end{align}
These ladder operator allows us to construct four states starting from $\ket{00}$, which is a non-zero state with $a^-\ket{00}=b^-\ket{00}=0$, and to construct an irreducible representation of $\mathfrak C_{1,3}$ (and, in fact, it is unique for even dimension):
\begin{equation}
 \ket{10} = a^+\ket{00},\quad
 \ket{01} = b^+\ket{00},\quad
 \ket{11} = a^+b^+\ket{00}\quad\leadsto\quad
 a^+=\spmat{0&0&1&0\\0&0&0&1\\0&0&0&0\\0&0&0&0},\quad
 b^+=\spmat{0&-1&0&0\\0&0&0&0\\0&0&0&1\\0&0&0&0},
\end{equation}
and $a^-=(a^+)^\dagger$, $b^-=(b^+)^\dagger$.
We then obtain a representation $\gamma$, which is called ``standard representation.''\footnote{The chiral notation \eqref{eq:chiralnotation} and Dirac notation \eqref{eq:diracnotation} are equivalent to this standard representation, i.e., related by unitary matrices.}
They are not Hermitian, but as we will see, this non-Hermiticity is solved by amending the inner product by a matrix $\beta$: $(\psi,\gamma^\mu\psi):=\psi^\dagger\beta \gamma^\mu\psi$.

Although $\psi$ forms an irreducible representation $\gamma^\mu$ of $\mathfrak C_{1,3}$, the resulting representation $S_{\mu\nu}$ (see the next paragraph) is a reducible representation of $\mathop{\mathrm {Spin}}(1,3)^+$.
This is confirmed by
\begin{equation}
 \gamma_5\gamma_5=1,\quad
\{\gamma_5,\gamma^\mu\}=0,\quad
 [\gamma_5,S_{\mu\nu}]=0;\qquad \gamma_5:=\ii\gamma^0\gamma^1\gamma^2\gamma^3,
\end{equation}
and $P{}^{\text L}\w{R} =(1\mp\gamma_5)/2$ works as the projection operators.
In addition, four state are eigenstates of $J_3=h_\eta S_{12}$ because 
\begin{equation}
 [J_3,b^+b^-] = 0,~
 J_3=b^+b^--\frac12,
\end{equation}
which also guarantees that spinors have spin $1/2$.
In summary,
\begin{align}
 &\ket{00} =\ket{-\w L},\quad
 \ket{10} =:\ket{-\w R},\quad
 \ket{01} =:\ket{+\w R},\quad
 \ket{11} =:\ket{+\w L};\\
&J_3\ket{\pm\w H}=\pm\tfrac12\ket{\pm\w H},\quad
\PL\ket{\pm\w L}=\ket{\pm\w L},~
\PR\ket{\pm\w R}=\ket{\pm\w R};\quad
\PL\ket{\pm\w R}=
\PR\ket{\pm\w L}=0.
\end{align}
For example, in chiral notation with $(+,-,-,-)$, the Lorentz generators $S_{\mu\nu}$ are block diagonal and $\ket{\pm\w L}$ ($\ket{\pm\w R}$) has non-zero component only in the upper (lower) two component:
\begin{equation}
 \ket{-\w L} = \spmat{0\\1\\0\\0},~
 \ket{-\w R} = \spmat{0\\0\\0\\1},~
 \ket{+\w R} = \spmat{0\\0\\\ii\\0},~
 \ket{+\w L} = \spmat{\ii\\0\\0\\0}.
\end{equation}

\paragraph{Four-spinors and Lorentz transformation}
The above ``theoretical'' discussion can be seen more explicitly, starting from spinors and a matrix representation $\gamma^\mu$ given by
\begin{equation}
 \overline\psi = \psi^\dagger\beta, \quad\beta\beta=1,\qquad
\{\gamma^\mu,\gamma^\nu\}=2\eta^{\mu\nu};\qquad
\psi\mapsto T\psi,\quad
\overline\psi\mapsto \overline\psi\beta T^\dagger \beta;\quad T\in \mathop{\mathrm{Spin}}(1,3)^+.
\end{equation}
For $\overline\psi\psi$ and $\overline\psi\gamma^\mu\psi$ to be respectively scalar and vector, $T$ should satisfy
\begin{equation}
 T^{-1}\gamma^\mu T = \LorTr\mu\nu\gamma^\nu,\quad
 \beta T^\dagger \beta T=1,
\end{equation}
or in infinitesimal form $T=1+(-\ii/2)d^{\mu\nu}S_{\mu\nu}$,
\begin{equation}
 (S_{\mu\nu})^\dagger = \beta S_{\mu\nu} \beta,\quad
[S_{\mu\nu},\gamma^\alpha] = -(M_{\mu\nu})^{\alpha}{}_\beta\gamma^\beta;
\qquad
\therefore S_{\mu\nu}=\frac{\ii}{4}[\gamma_\mu,\gamma_\nu];
\quad
[\gamma_\mu^\dagger, \gamma_\nu^\dagger]=\beta[\gamma_\mu,\gamma_\nu]\beta;
\label{eq:betacondition}
\end{equation}
the first condition leads to a representation of the Lorentz group
\begin{equation}
\Lambda\stackrel{\text{rep}}=
 \exp\left(\frac{-\ii}{2}d^{\mu\nu}S_{\mu\nu}\right);\qquad
 [S_{\mu\nu},S_{\rho\sigma}] = -\ii\left(
\eta_{\mu\rho} S_{\nu\sigma}
-\eta_{\nu\rho} S_{\mu\sigma}
+\eta_{\mu\sigma} S_{\nu\rho}
-\eta_{\nu\sigma} S_{\mu\rho}\right)
\end{equation}
as seen in \cref{eq:LorentzAlgebra}, while the second condition determines what $\beta$ should be, as given in, e.g., \cref{eq:chiralnotation} and \cref{eq:diracnotation}.

\paragraph{Charge conjugation and Majorana spinor}
The charge conjugation of $\psi$ is something like $\psi^*$ but should obey the same representation as $\psi$ does, i.e., $C'\psi^*$ with $C'$ being a unitary matrix such that $B\psi^*\to TB\psi^*$. Or it can be seen that the, the previous procedure with $\psi^*$ may generate another irreducible representation and it should be related to $\gamma^\mu$ by an unitary matrix. Anyway, we define the charge conjugation by
\begin{equation}
 \psi^\cc = C(\overline\psi)^\TT = C\beta^\TT\psi^*;\quad CC^\dagger = 1.
\quad
\therefore
C^*\beta[\gamma_\mu^\dagger,\gamma_\nu^\dagger]\beta C^\TT =
[\gamma_\mu^\TT,\gamma_\nu^\TT].
\end{equation}
Combining with \cref{eq:betacondition},
\begin{equation}
 \beta\beta=1,\quad C C^\dagger = 1,\quad
 \beta\gamma^\mu\beta=h_\beta(\gamma^\mu)^\dagger,\quad
 C\gamma_\mu^* C^\dagger = h_C\gamma_\mu^\dagger.
\end{equation}
In even dimensions, the expressions have many choices as seen in the signs $h_C$ and $h_\beta$; moreover, the sign $h\w{cc}$ defined in $(\psi^\cc)^\cc=h\w{cc}\psi$ depends on the definition.
It is thus useful to use a specific notation for further discussion.

For example, the construction of a Majorana spinor $\psi\w M$, which satisfies $(\psi\w M)^\cc=\psi\w M$, is simply done as $\psi\w{M}\propto\psi+\psi^\cc$ if $\eta\w{cc}=1$, but needs some phases if $\eta\w{cc}=-1$.

\paragraph{fragments}
\vspace{2em}

Lorentz tensor $
M^{\mu_1\mu_2\cdots\mu_n}\propto
\bsigma^{\mu_1\dbeta_1\alpha_1}\cdots M_{\alpha_1\cdots\dbeta_1\cdots}
$

Especially
$V^\mu =: \frac12 \bsigma^{\mu\dbeta\alpha}V_{\alpha\dbeta}$,
$V_{\alpha\dbeta}=V^\mu\sigma_{\mu\alpha\dbeta}$; hermite $V_{\alpha\dbeta}$ $\Leftrightarrow$ $real V^\mu$.

\begin{align}
 &(V^\TT)_{\alpha\dbeta} = V_{\beta\dalpha},\text{\TODO{(correct? possibly wrong dot-positions?)}}\\
 &(V^*)_{\dalpha\beta}:= (V_{\alpha\dbeta})^*,\\
 &(V^\dagger)_{\alpha\dbeta} := (V_{\beta\dalpha})^* = (V^*)_{\dbeta\alpha}
\end{align}
\TODO{anyway not very sure about the reasoning; though my old note says like this...}

\subsection{Dirac spinors for massive fermions}
%\RefBox{\S2 of \cite{kugo1}; \cite{Kryuchkov:2015acw}}
Dirac equation $(\ii\slashed{\partial}-m)\psi(x)=0$ has plain-wave solutions $\tilde u(p)\ee^{-\ii p x}$, i.e., $(\slashed{p}-m)\tilde u(p)=0$ with $m>0$.
Non-zero solution is available if and only if $\det(\slashed p-m)=0\Leftrightarrow p^2=m^2$.
We hereafter fix $p^0>0$ and consider
\begin{equation}
 u(\vc p)\text{~satisfying~}(\slashed p-m)u(\vc p)=0,
\qquad
 v(\vc p)\text{~satisfying~}(\slashed p+m)v(\vc p)=0,
\end{equation}
where the general solution of the Dirac equation is given by linear combination of $\{u(\vc p)\ee^{-\ii px}, v(\vc p)\ee^{\ii px}\}_{p^0>0}$.

As $\rank(\slashed p\pm m)=2$, each equation has $4-2 = \text{two}$ linear-independent solutions\footnote{%
  Let $X^\pm=\slashed{p}\pm m$. Denote $\det A$ by $|A|$, and $\rank A$ by $\RANK{A}$.
  As Lorentz invariance guarantees $|X^\pm|$ is a function of $p^2$, $|A^+|=|A^-|$.
  Then $|A^+||A^-| = (p^2-m^2)^4$ gives $|A^\pm|=(p^2-m^2)^2$.
  One can show that $\RANK{A^\pm}=2$ if $p^2=m^2$ as follows:
\\%
Frobenius inequality:\qquad $\RANK{AB}+\RANK{BC}\le \RANK B + \RANK{ABC}$\qquad
$\therefore 2\RANK{A^\pm}=\RANK{A^\pm\gamma_5\gamma_5} + \RANK{\gamma_5A^\mp\gamma_5} \le \RANK{\gamma_5}+\RANK{A^\pm\gamma_5\gamma_5A^\mp\gamma_5}=4$,
\\%
rank subadditivity:\qquad$\RANK{A+B}\le\RANK{A}+\RANK{B}$\qquad
$\therefore4\le\RANK{A^+}+\RANK{-A^-}.$

Similar discussion applies for Pauli--\Lubanski operator: as $[\det(\gamma_5\slashed{e}-x)]^2=-e^2-x^2$, it has eigenvalues $\pm1/2$ for $u(\vc p)$ and $v(\vc p)$, and 
$B^\pm=(\gamma_5\slashed e\slashed p\pm m/2)$ both have rank-2.
Furthermore, \TOCHECK as $B^+$ should have solutions in both subspaces spanned by $u(\vc p)$ and $v(\vc p)$, each subspace is spanned by eigenstates with different eigenvalues for the Pauli--\Lubanski operator and thus the label is used to distinguish two $u$s (and $v$s).
}, for which we introduce another label $s=1,2$.
We use the following convention, noting the inner product of this vector space is $(\psi,\psi)=\overline\psi\psi$:
\begin{equation}
 v^s(\vc p)\coloneq C[\overline{u}^s(\vc p)]^\TT,\qquad
 \overline u^s(\vc p)u^t(\vc p)\coloneq 2m\delta^{st}.
\end{equation}
Then $\gamma^\mu v = (h_\beta/h_C)C\beta^\TT(\gamma^\mu u)^*$, for which $v(\vc p)$ satisfies $(\slashed{p}+m)v(\vc p)=0$ as requested.
Orthogonality for $s\neq t$ is guaranteed by its definition given later.
Also, as $u$ and $v$ have different eigenvalues of the matrix $\slashed p$,
\begin{equation}
  \overline v^s(\vc p)v^t(\vc p)=-2m\delta^{st},\qquad
 \overline u^s(\vc p)v^t(\vc p)=\overline v^s(\vc p)u^t(\vc p)=0.
\end{equation}

Several conventions for the label $s$ are available:\vspace{-1em}
\begin{DownPara}
\paragraph{Pauli--\Lubanski pseudovector operator} gives the most generic definition for the spin of a moving particle:
\begin{equation}
 w^\mu = \frac1{2m}\epsilon^{\mu\nu\rho\sigma}P_\nu M_{\rho\sigma}.
\end{equation}
With a reference vector $e^\mu$ such that $e^2=-1$ and $e^\mu p_\mu=0$,
\begin{equation}
 e_\mu w^\mu
= e_\mu \frac1{2m} \epsilon^{\mu\nu\rho\sigma}p_\nu \frac{\ii}{4}[\gamma_\rho,\gamma_\sigma]
= \frac\ii{4m} \epsilon^{\mu\nu\rho\sigma}e_\mu p_\nu \gamma_\rho\gamma_\sigma
= \frac1{2m}\gamma_5 \slashed e \slashed p
\end{equation}
commutes with $(\slashed p\pm m)$ and thus $s=\pm$ may denote the eigenvalue of this operator $\pm1/2$.


\paragraph{Helicity operator}
A simpler option is the eigenvalues $\pm1$ of the helicity operator $h=\vc\sigma\cdot\vc p/\|\vc p\|=:\vc\sigma\cdot\vc n$.
This is in fact a special case of Pauli--\Lubanski operator with $e^\mu= (\|\vc p\|/m,p_0\vc n/m)$, which is verified by
\begin{align}
& h
=\epsilon^{0\nu\rho\sigma}(-n_\nu) S_{\rho\sigma}
=\frac{1}{2\ii}\epsilon^{0\nu\rho\sigma}n_\nu \gamma_\rho\gamma_\sigma
=\frac{\eta^{\mu0}}{2\ii}\epsilon_{\mu\nu\rho\sigma}n^\nu \gamma^\rho\gamma^\sigma
=\frac{\gamma_5}{2}\gamma_5[\slashed{n},\gamma^0]
=\frac{\gamma_5}{\|\vc p\|}({\{\slashed{p},\gamma^0\}}/{2}-\gamma^0\slashed{p});
\\&
w_\mu e^\mu 
=\frac{1}{2\|\vc p\|}\gamma_5(p^0-\gamma^0\slashed{p})\sim \frac{h}{2}.
\end{align}

% Low\[Epsilon][m_,n_,r_,s_]:=-LeviCivitaTensor[4][[m+1,n+1,r+1,s+1]]
% Low\[Epsilon][0,1,2,3]
% Sum[Low\[Epsilon][m,n,r,s]S[m]p[n]GammaMatrix[r,None] . GammaMatrix[s,None],{m,0,3},{n,0,3},{r,0,3},{s,0,3}];
% -2I LorSum[S[#]GammaMatrix[None,#]&] . LorSum[p[#]GammaMatrix[None,#]&] . GammaFive;
% Expand[%%-%]//.{p[0] S[0]->p[1]S[1]+p[2]S[2]+p[3]S[3]}//Simplify

\TODO{Discuss Poincare group before introducing PL operator.: \cite{WeinbergQFT1}, \cite{Kryuchkov:2015acw}}

\TODO{We have not defined the relation between $S_{\mu\nu}$ and $\sigma_i$ etc.; verify for more general case, including the proof of $\epsilon_{\mu\nu\rho\sigma}\gamma^\rho\gamma^\sigma=-\ii\gamma_5[\gamma_\mu,\gamma_\nu]$.}
\end{DownPara}

Projection operators for the subspaces spanned by $u^s$ ($v^s$) are given by
\begin{equation}
 P_u=\frac{m+\slashed p}{2m}, \quad P_v=\frac{m-\slashed p}{2m}.
\end{equation}
As $\{u^1, u^2, v^1, v^2\}$ is a basis of the vector space $\mathbb C^4$ with inner product $(\psi,\psi)=\overline\psi\psi$, we immediately have 
\begin{equation}
 \sum_{s=1,2}u^s(\vc p)\overline u^s(\vc p)=P_u \overline u^s(\vc p)u^s(\vc p) = \slashed p+m,\qquad
 \sum_{s=1,2}v^s(\vc p)\overline v^s(\vc p)=\slashed p-m.
\end{equation}
Similarly, using projection operator $P'_{\pm}=(1\pm\gamma_5\slashed e\slashed p/m)/2$,
\begin{align}
 &u^s(\vc p)\overline u^s(\vc p) = 2m P'_{\pm}P_u = \frac12(\slashed p+m)(1+s\gamma_5\slashed e),\\\
 &v^s(\vc p)\overline v^s(\vc p) = C(u^s\overline u^s)^\TT \beta^*C^\dagger\beta
=\frac12(m+h_C \slashed p)(1-s h_C \gamma_5\slashed e )C\beta^*C^\dagger \beta
\leadsto
\frac12(\slashed p-m)(1+s\gamma_5\slashed e)
\end{align}
for $s=\pm1$.

\subsection{Chiral notation}
In the chiral notation, $\beta=\gamma^0$ and $C=-\ii\gamma^2\gamma^0$, and $h_\beta=1$ and $h_C=-1$, the plain-wave spinors are given by
\begin{equation}
 u^s(\vc p) = \pmat{\sqrt{\sigma\cdot p}\xi^s\\\sqrt{\bar\sigma\cdot p}\xi^s},\quad
 v^s(\vc p) = \pmat{\sqrt{\sigma\cdot p}\eta^s\\-\sqrt{\bar\sigma\cdot p}\eta^s};\qquad
 \eta^s = \pmat{\eta^s_1\\\eta^s_2} \coloneq  \pmat{-(\xi^s_2)^*\\(\xi^s_1)^*},\quad
(\xi^s)^\dagger(\xi^t) = (\eta^s)^\dagger(\eta^t) = \delta^{st},
\end{equation}
where $\{\xi^1,\xi^2\}$ are the orthonormal basis, which fixes the definition of $\eta$, and\footnote{%
The notation is justified by
\begin{equation*}
\sqrt{\sigma\cdot p}\sqrt{\sigma\cdot p} ={\sigma\cdot p},\qquad
\sqrt{\bar\sigma\cdot p}\sqrt{\bar\sigma\cdot p} ={\bar\sigma\cdot p},\qquad
\sqrt{\sigma\cdot p}\sqrt{\bar\sigma\cdot p}=m.
\end{equation*}}
\begin{equation}
  \sqrt{\sigma\cdot p} \coloneq  \frac{m+p^\mu\sigma_\mu}{\sqrt{2(m+p^0)}},
\qquad
 \sqrt{\bar\sigma\cdot p} \coloneq  \frac{m+p^\mu\bar\sigma_\mu}{\sqrt{2(m+p^0)}}.
\end{equation}
\footnote{It is straightforward to verify the above-given equations:
\begin{align}
&
 \overline u u = 2m\delta^{st},\quad
 \overline v v = -2m\delta^{st},\quad
 \overline u v = \overline v u = 0,\quad
 u^\dagger u = v^\dagger v = 2p^0\delta^{st},\quad
\\&
 u^\dagger(\vc p)u(-\vc p)= v^\dagger(\vc p)v(-\vc p)=2m\delta^{st},\quad
 u^\dagger(\vc p)v(-\vc p)= v^\dagger(\vc p)u(-\vc p)=0,\quad
 \overline u(\vc p)u(-\vc p)= -\overline v(\vc p)v(-\vc p)=2p^0\delta^{st}.
\end{align}
}
So that the basis $\xi$ has eigenvalue $\pm1/2$ for the Pauli--\Lubanski operator, it should satisfy
\begin{equation}
 \frac12 \gamma_5\slashed{e} U \pmat{\xi^s\\\xi^s} = \left(\pm\frac12\right)U\pmat{\xi^s\\\xi^s};\qquad
\text{where~}
 U\coloneq \pmat{\sqrt{\sigma\cdot p} & 0 \\ 0 & \sqrt{\bar\sigma\cdot p}},\quad
 U^{-1}=\frac{1}{m}\pmat{\sqrt{\bar\sigma\cdot p} & 0 \\ 0 & \sqrt{\sigma\cdot p}}.
\end{equation}
Thus, by calculating the eigensystem of $U^{-1}\gamma_5\slashed {e} U$, one can get
\begin{equation}
 \xi^+ = \frac{1}{\sqrt{2(1-\phi^3)}}\pmat{\phi^1-\ii \phi^2 \\ 1-\phi^3},\quad
 \xi^- = \frac{1}{\sqrt{2(1-\phi^3)}}\pmat{1-\phi^3\\-\phi^1-\ii \phi^2};\qquad
 \phi^\mu = e^\mu-\frac{e^0p^\mu}{m+p^0}.
\end{equation}
If we use reference vector for the helicity, 
\begin{equation}
 \xi^+ = \frac1{\sqrt{2-2n^3}}\pmat{n^1-\ii n^2\\1-n^3}
       =\pmat{\ee^{-\ii\phi}\cos(\theta/2)\\\sin(\theta/2)},\quad
 \xi^- = \frac1{\sqrt{2-2n^3}}\pmat{1-n^3\\-n^1-\ii n^2}
       =\pmat{\sin(\theta/2)\\-\ee^{\ii\phi}\cos(\theta/2)}
\end{equation}
with $(n^1,n^2,n^3)=(\si\theta\co\phi,\si\theta\si\phi,\co\theta)$ is the direction of the momentum.





\end{document}
